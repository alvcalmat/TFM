\chapter{Design of the proposed solution}
\label{ch:DesignOfTheProposedSolution}
\lettrine[lraise=-0.1, lines=2, loversize=0.2]{L}{o}rem itsum
% El planificador desarrollado en esta tésis se compone a su vez de dos módulos bien diferenciados, el primero es el planificador de tareas propiamente dicho, encargado de la planificación de la misión y de su replanificación cuando fuera necesario, el segundo se encuentra a bordo de cada UAV y gestiona el comportamiento de cada equipo, ejecutando el plan que le ha asignado el primer módulo y reaccionando ante cualquier imprevisto de forma segura.

%%% Capi: En los capítulos 3 y 4 puedes coger texto del documento que tenemos hecho con Giuseppe, y del proyecto tuyo de tesis. 

\section{Node diagram}
\label{sec:NodeDiagram}
%%% Explicar las comunicaciones que hay entre el planner y el manager
%%% Explicar que se ha hecho de forma que toda la inteligencia y las decisiones estén y se tomen en el planner

\section{Centralized module: task planner}
\label{sec:Centralized module:TaskPlanner}

\section{Distributed module: behavior manager}
\label{sec:Distributed module: behavior manager}
%%% Explicar que se ha hecho con árboles de comportamiento en paralelo a algunos procesos de ros

\subsection{Main tree}
\label{sec:MainTree}
%%% Recalcar que se ha hecho de forma que toda la inteligencia y las decisiones estén y se tomen en el planner
%%% Hablar aqui dentro de como se gestionan las desconexiones, la batería y las replanificaciones

\subsection{Inspection task tree}
\label{sec:InspectionTaskTree}

\subsection{Monitoring task tree}
\label{sec:MonitoringTaskTree}

\subsection{Tool delivery task tree}
\label{sec:ToolDeliveryTaskTree}

\section{Lower and upper level modules faker}
\label{sec:LowerAndUpperLevelModulesFaker}
%%% GoToWP, Recharge, Monitoring, Inspection, ToolDelivery
%%% Battery sensor
