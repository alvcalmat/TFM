\chapter{Conclusions and future work}
\label{ch:ConclusionsAndFutureWork}
% Las conclusiones en formato:
    % Se ha hecho X, Y, y funciona muy bien.
    % Se ha visto que ocurre A, B, C

\section{Conclusions}
\label{sec:Conclusions}
% Optimalidad del planner, funcion de costes, aproximación realizada, simplificaciones realizadas
% Planificación de recargas: Se podría haber añadido una tarea de tipo Recharge para planificarlas. No planifica teniendo en cuenta las recargas ahora mismo, solo si hay batería suficiente. Eso se puede mejorar.
% El "fallo" que he encontrado en el Tool Delivery Task Tree. Devolve la herramienta a la base en caso de fallo.
% Separar el Recharge Action Node en la estructura típica (la de Inspection Task Tree): aproximación a la base por un lado y Recarga por otro.
% Importancia del battery faker para testear correctamente situaciones de emergencia.

\section{Future work}
\label{sec:FutureWork}
% Task ¿?
% Planificación de recargas: Se podría haber añadido una tarea de tipo Recharge para planificarlas. No planifica teniendo en cuenta las recargas ahora mismo, solo si hay batería suficiente. Eso se puede mejorar.

\subsection{Augmented reality}
\label{AugmentedReality}

\endinput
