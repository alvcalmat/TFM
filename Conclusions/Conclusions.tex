\chapter{Conclusions and future work}
\label{ch:ConclusionsAndFutureWork}
% Las conclusiones en formato:
    % Se ha hecho X, Y, y funciona muy bien.
    % Se ha visto que ocurre A, B, C

\section{Conclusions}
\label{sec:Conclusions}
%% Comentar los objetivos marcados:
% - Garantizar la utilización de los recursos y la ejecución eficiente de las tareas.
% - Cumplir con todos los requisitos de seguridad y garantizar la integridad de las plataformas aéreas y el éxito de la misión.
% - Ser capaz de replanificar en línea para reaccionar a eventos imprevistos.
% - Implementar la capa de software en el Sistema Operativo del Robot (ROS) y gestionar la comunicación con el resto de capas y módulos de software que conforman la arquitectura arquitectura.
% - Realizar simulaciones de Software In The Loop (SITL) para demostrar que el algoritmo es capaz de gobernar el comportamiento de la flota de forma eficiente y segura, y que es capaz de reaccionar ante imprevistos de forma dinámica, demostrando sus capacidades cognitivas.
% - Diseñar el planificador de tareas de forma que sea fácil de mantener, modificar o ampliar, procurando que sea modular y reutilizable para que pueda servir de base para la construcción de planificadores para otras aplicaciones.

% El BT se puede mejorar pero funciona muy bien y sienta las bases para programar comportamientos más complejos en el futuro. Servirá de ejemplo para la comunidad. Permite generar comportamientos complejos y numerosos estados sin que haya que preocuparse de las transiciones entre estados como pasa con las FSM, en las que este crece exponencialmente con el número de estados.

% Optimalidad del planner, funcion de costes, aproximación realizada, simplificaciones realizadas: decir que la aproximación del Planner en casos reales es aceptable


% Planificación de recargas: Se podría haber añadido una tarea de tipo Recharge para planificarlas. No planifica teniendo en cuenta las recargas ahora mismo, solo si hay batería suficiente. Eso se puede mejorar.

% Separar el Recharge Action Node en la estructura típica (la de Inspection Task Tree): aproximación a la base por un lado y Recarga por otro.

% El "fallo" que he encontrado en el Tool Delivery Task Tree. Devolver la herramienta a la base en caso de fallo.

% Importancia del battery faker para testear correctamente situaciones de emergencia.

\section{Future work}
\label{sec:FutureWork}
% Validar las técnicas desarrolladas en un entorno real

% Task ¿?
% Planificación de recargas: Se podría haber añadido una tarea de tipo Recharge para planificarlas. No planifica teniendo en cuenta las recargas ahora mismo, solo si hay batería suficiente. Eso se puede mejorar.

% Realidad aumentada

% Introducir algoritmos heuristicos aleatorios en el planificador para encontrar el plan óptimo de verdad.

% ¿Redes neuronales?

\endinput
