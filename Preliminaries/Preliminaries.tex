\chapter{Preliminaries}
\label{ch:Preliminaries}
\lettrine[lraise=-0.1, lines=2, loversize=0.2]{L}{o}rem itsum

% Poner en contexto las tecnologías que hay hoy día y demás.
\section{Current technology}
\label{sec:CurrentTechnology}

% Capi: quitar las subsecciones si veo que se puede integrar todo junto. Si veo que queda largo, dividir en subsecciones como tengo.
\subsection{UAVs}
\label{subsec:UAVs}

\subsection{Aerial co-workers}
\label{subsec:AerialCo-workers}

\subsection{Multi-drone teams}
\label{subsec:Multi-droneTeams}


% Related work: buscar artículos que tengan que ver con mi proyecto para poner en contexto lo que voy a aportar.
\section{Related work}
\label{sec:RelatedWork}

% Capi: quitar las subsecciones si veo que se puede integrar todo junto. Si veo que queda largo, dividir en subsecciones como tengo.
\subsection{Inspection applications with UAVs}
\label{subsec:InspectionApplicationsWithUAVs}

% Hablar de las formas existentes que hay para abordar el problema del reparto de tareas en equivos multi-UAV. Poner en contexto lo que voy a aportar con mi TFM.
\subsection{Task planning in multi-drone teams}
\label{subsec:TaskPlanning}

% Hablar de las formas existentes que hay para gestionar el comportamiento de un dron y su guiado en cada instante. Poner en contexto lo que voy a aportar con mi TFM. Hablar de las FSM y de los BT
\subsection{Drone behavior management}
\label{subsec:DroneBehaviorManagement}


% Estudio previo / Herramientas
% No hace falta explayarse mucho en ROS, Gazebo y Rviz, se da por supuesto en los que leen el documento, se cuenta brevemente en qué consisten esas herramientas y para qué se van a usar en el TFM
\section{Tools}
\label{sec:PreviousStudy}

\subsection{ROS}
\label{subsec:ROS}

\subsection{Gazebo}
\label{subsec:Gazebo}

\subsection{Rviz}
\label{subsec:Rviz}

\subsection{UAL}
\label{subsec:UAL}

\subsection{Behaviour Trees}
\label{subsec:BehaviourTrees}

\subsection{Groot}
\label{subsec:Groot}



\endinput
