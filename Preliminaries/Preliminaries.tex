\chapter{Preliminaries}
\label{ch:Preliminaries}
\lettrine[lraise=-0.1, lines=2, loversize=0.2]{T}{his} chapter focuses on the current state of the art of those technologies related to this project, as well as on the tools used for the development of the task planner as a software layer of a multi-layer architecture. In addition, the research work carried out on the state of the art in work related to the technologies and techniques used in this project is presented. 

% Poner en contexto las tecnologías que hay hoy día y demás.
\section{Current technology}
\label{sec:CurrentTechnology}
%% Origen de los drones (vehículos aéreos no tripulados, controlados por radiocontrol, no son autonomos, forma tipica de aeronave).
% Aunque en la última década el uso de UAVs se ha extendido a un gran número de aplicaciones [CivilAplications, UAVhandbook], el origen de esta tecnología se remonta a 1898 con la invención del radio control y la aparición de las primeras aeronaves no tripuladas, bautizadas con el nombre de dron. Estos no eran aun vehículos aéreos no manejados, y se empleaban principalmente para fines militares. 
Although in the last decade the use of \glspl{UAV} has spread to a large number of applications, the origin of this technology dates back to 1898 with the invention of radio control and the appearance of the first unmanned aircraft, baptised with the name of drone. These were not yet unmanned aerial vehicles, and were mainly used for military purposes.

\begin{figure}[htbp]
    \centering
    \includegraphics[width=0.6\linewidth]
    {Preliminaries/figures/Predator.jpg}
    \caption{General Atomics MQ-1 Predator. A \gls{RPA}. Source: \href{https://en.wikipedia.org/wiki/General_Atomics_MQ-1_Predator}{Wikipedia}}
    \label{fig:predator}
\end{figure}

%% Primeros UAV (drones autónomos). 
% Posteriormente, con el desarrollo de la tecnología, aparecieron los primeros ordenadores con el tamaño y la capacidad de cómputo suficiente para ejecutar el software necesario para operar un UAV de forma autónoma e incluso para controlar aeronaves con dinámicas más complejas e incluso inestables (planeadores [predator, BIGBLUE], dirigibles [AURORA], quadrotores [quadrotorsreview, mesicopter, pounds, miniquadrotor], multirotores [fullyactuated], alas batientes [COLIBRI, GRIFFING], etc.). A pesar de que la capacidad de cómputo seguía siendo insuficiente para algunas aplicaciones, desarrollar sistemas uav era posible gracias a la realzación de cálculos en tierra. Lo que se hacía era ejecutar los sistemas críticos y más importantes para el vuelo autónomo en el ordenador a bordo (controles, adquisición de datos, evitación de obstáculos, etc.), y ejecutar aquellos cálculos más demandantes y que no son necesarios a tiempo real sobre ordenadores en tierra [OffBoard, ollero2004control].
Later, with the development of technology, the first computers of sufficient size and computing power to run the software necessary to operate a \gls{UAV} autonomously and even to control aircraft with more complex and even unstable dynamics (gliders \cite{predator, BIGBLUE}, airships \cite{AURORA}, quadrotors \cite{quadrotorsreview, mesicopter, pounds, miniquadrotor}, multirotors \cite{fullyactuated}, flapping wings \cite{COLIBRI, GRIFFING, GRIFFIN2021}, etc.) appeared. Even though computational capacity was still insufficient for some applications, the development of \gls{UAV} systems was made possible by performing calculations on the ground. What was done was to run the critical and most important systems for autonomous flight on the on-board computer (controls, data acquisition, obstacle avoidance, etc.), and to run the more demanding calculations that are not necessary in real time on the ground computers \cite{OffBoard}.

\begin{figure}[htbp]
    \centering
    \includegraphics[width=0.6\linewidth]
    {Preliminaries/figures/GRIFFIN.png}
    \caption{GRIFFIN's flapping wing robot \cite{GRIFFIN2021}}
    \label{fig:predator}
\end{figure}

%% Adquisición de datos (para ser autónomos necesitan recolectar datos del entorno)
% Para que un vehículo aéreo pueda operar de forma autónoma es necesaria la adquisición de datos del entorno y su posterior procesado en tiempo real. En la literatura actual se pueden encontrar un gran número de configuraciones de sensores diferentes así como numerosas técnicas de adquisición y procesamiento de datos [buscar algun paper que haga un resumen de todas]. Citando algunos ejemplos, [,,,] descripción, [,,,] descripción, [,,,] descripción.
For an aerial vehicle to operate autonomously, it is necessary to acquire data from the environment and process it in real time. A large number of different sensor configurations as well as numerous data acquisition and processing techniques can be found in the current literature \cite{SenseAndAvoid, aasen2018quantitative, miningSensors}.

%% Aplicaciones para UAV existentes (que necesitan de un piloto humano que supervise, no son cognitivos)
%% Equipos multi-UAV (con supervisión humana) [multiUAVclassification]
%% Explosión de la tecnología.
% Una vez que la tecnología UAV alcanzó la capacidad y autonomía suficientes, comenzaron a aparecer las primeras aplicaciones para equipos tanto individuales [] como multi-UAV. Sobre estos últimos existe un gran interés, ya que se pueden configurar de formas diversas, recopilar y procesar datos de forma distribuida, incrementando la capacidad de cómputo del equipo, y generar comportamiento colectivos globales que emerjan de interacciones entre un gran número de UAVs que individualmente sean relativamente simples, lo que se conoce como enjambre. 
Once UAV technology reached sufficient capacity and autonomy, the first applications for both single \cite{nex2014uav, radoglou2020compilation, drummond2015uav} and multi-\gls{UAV} \cite{martinez2007multi, gu2018multiple, scherer2015autonomous} equipment began to appear. There is great interest in the latter, as they can be configured in different ways \cite{multiUAVclassification}, collect and process data in a distributed way, increasing the computational capacity of the equipment \cite{pascarella2015parallel, guo2021coded}, and generate global collective behaviour emerging from interactions between a large number of \glspl{UAV} that individually are relatively simple, known as swarming \cite{zhou2020uav, campion2018uav, chen2020sidr}. 

%% Esfuerzo para el desarrollo de aplicaciones sin presencia humana: capacidades cogitivas, entornos dinámicos. (Aerial co-workers)
% Las aplicaciones actuales suelen necesitar presencia humana para llevar a cabo ciertas decisiones, siendo el piloto humano el que supervisa que todo se desarrolle correctamente y quien aporta la capacidad cognitiva para analizar el entorno generalmente dinámico y reaccionar a situaciones imprevistas. Esto se debe a que dotar a un UAV de capacidad cognitiva suficiente para operar de forma totalmente autónoma en entornos dinámicos es una tarea muy complicada y que además requiere de gran capacidad de procesamiento. En los últimos años la tencología UAV esta ha evolucionado muy rápidamente beneficiandose de los avances que se están produciendo en computación y en inteligencia artificial. Gracias a que los procesadores son cada vez más potentes, eficientes y pequeños, los UAVs disponen cada vez de una mayor capacidad de cálculo sin aumentar con ello su peso y comprometer su autonomía. Con el aumento en el número de operaciones por segundo que estos equipos pueden realizar, se abre la posibilidad de emplear drones para aplicaciones antes impensables, aplicaciones que requieren de una gran cantidad de procesamiento y que, por lo general, hay que realizar en tiempo real. Al mismo tiempo, los avances en inteligencia artificial permiten que la percepción y la capacidad de análisis y fusión sensorial de los equipos UAV sean cada vez mejores. El avance de la tecnología está derribando una de las barreras que impiden a la tecnología UAV alcanzar ese nivel de autonomía, y con ello, se está destinando cada vez más esfuerzo de investigación para derribar la otra barrera, desarrollar software que permita a UAVs tener capacidad cognitiva.
Current applications often require human presence to carry out certain decisions, with the human pilot overseeing that everything runs smoothly and providing the cognitive capacity to analyse the generally dynamic environment and react to unforeseen situations \cite{kopeikin2012flight, }. This is because providing a \gls{UAV} with sufficient cognitive capacity to operate fully autonomously in dynamic environments is a very complicated task and requires a great deal of processing power. In recent years, \gls{UAV} technology has evolved rapidly, benefiting from advances in computing and artificial intelligence. As processors are becoming more powerful, efficient and smaller, \glspl{UAV} are becoming more and more powerful without increasing their weight or compromising their autonomy. With the increase in the number of operations per second that \glspl{UAV} can perform, this opens up the possibility of using drones for previously unthinkable applications, applications that require a large amount of processing and usually have to be performed in real time \cite{CivilAplications, shakeri2019design}. At the same time, advances in artificial intelligence mean that the perception, analysis and sensory fusion capabilities of \glspl{UAV} are getting better and better. Advances in technology are breaking down one of the barriers preventing \gls{UAV} technology from achieving this level of autonomy, and with it, more and more research effort is being devoted to breaking down the other barrier, developing software that enables \glspl{UAV} to have cognitive capabilities.

%% Investigaciones recientes para aplicaciones 100% autónomas
% Aerial Co-workers
% Mencionar proyectos en los que está involucrada la universidad de sevilla

%% Logros más recientes.
% Recientemente se ha completado el primer vuelo de un UAV en marte.

% Related work: buscar artículos que tengan que ver con mi proyecto para poner en contexto lo que voy a aportar.
\section{Related work}
\label{sec:RelatedWork}

% Capi: quitar las subsecciones si veo que se puede integrar todo junto. Si veo que queda largo, dividir en subsecciones como tengo.
\subsection{Inspection applications with UAVs}
\label{subsec:InspectionApplicationsWithUAVs}

% Hablar de las formas existentes que hay para abordar el problema del reparto de tareas en equivos multi-UAV. Poner en contexto lo que voy a aportar con mi TFM.
\subsection{Task planning in multi-drone teams}
\label{subsec:TaskPlanning}

% Hablar de las formas existentes que hay para gestionar el comportamiento de un dron y su guiado en cada instante. Poner en contexto lo que voy a aportar con mi TFM. Hablar de las FSM y de los BT
\subsection{Drone behavior management}
\label{subsec:DroneBehaviorManagement}


% Estudio previo / Herramientas
% No hace falta explayarse mucho en ROS, Gazebo y Rviz, se da por supuesto en los que leen el documento, se cuenta brevemente en qué consisten esas herramientas y para qué se van a usar en el TFM
\section{Tools}
\label{sec:PreviousStudy}

\subsection{ROS}
\label{subsec:ROS}

\subsection{Gazebo}
\label{subsec:Gazebo}

\subsection{Rviz}
\label{subsec:Rviz}

\subsection{UAL}
\label{subsec:UAL}

\subsection{Behaviour Trees}
\label{subsec:BehaviourTrees}

\subsection{Groot}
\label{subsec:Groot}



\endinput