\chapter{Preliminaries}
\label{ch:Preliminaries}
\lettrine[lraise=-0.1, lines=2, loversize=0.2]{T}{his} chapter focuses on the current state of the art of those technologies related to this project, as well as on the tools used for the development of the task planner as a software layer of a multi-layer architecture. In addition, the research work carried out on the state of the art in work related to the technologies and techniques used in this project is presented. 

% Poner en contexto las tecnologías que hay hoy día y demás.
\section{Current technology}
\label{sec:CurrentTechnology}
%% Origen de los drones (vehículos aéreos no tripulados, controlados por radiocontrol, no son autonomos, forma tipica de aeronave).
% Aunque en la última década el uso de UAVs se ha extendido a un gran número de aplicaciones [CivilAplications, UAVhandbook], el origen de esta tecnología se remonta a 1898 con la invención del radio control y la aparición de las primeras aeronaves no tripuladas, bautizadas con el nombre de dron. Estos no eran aun vehículos aéreos no manejados, y se empleaban principalmente para fines militares. 

%% Primeros UAV (drones autónomos). 

% Posteriormente, con el desarrollo de la tecnología, aparecieron los primeros ordenadores con el tamaño y la capacidad de cómputo suficiente para ejecutar el software necesario para operar un UAV de forma autónoma e incluso para controlar aeronaves con dinámicas más complejas e incluso inestables (planeadores [], quadrotores [], multirotores [], alas batientes [COLIBRI, GRIFFING], etc.). Lo que se hacía era ejecutar los sistemas críticos y esenciales para que el dron volase de forma autónoma en el ordenador a bordo (controles, adquisición de datos, evitación de obstáculos, etc.), y ejecutar aquellos cálculos que requieren una gran cantidad de recursos y que no son necesarios a tiempo real sobre ordenadores centralizados [buscar algun paper antiguo que haga esto (ollero2004control puede que sirva)].

%% Adquisición de datos (para ser autónomos necesitan recolectar datos del entorno)
% Para que un vehículo aéreo pueda operar de forma autónoma es necesaria la adquisición de datos del entorno y su posterior procesado en tiempo real. En la literatura actual se pueden encontrar un gran número de configuraciones de sensores diferentes así como numerosas técnicas de adquisición y procesamiento de datos [buscar algun paper que haga un resumen de todas]. Citando algunos ejemplos, [,,,] descripción, [,,,] descripción, [,,,] descripción.

%% Aplicaciones para UAV existentes (que necesitan de un piloto humano que supervise, no son cognitivos)


%% Equipos multi-UAV (con supervisión humana)

%% Explosión de la tecnología.
% La tecnología UAV ha evolucionado muy rápidamente en los últimos años beneficiandose de los avances que se están produciendo en computación. Gracias a que los procesadores son cada vez más potentes, eficientes y pequeños, los UAVs disponen cada vez de una mayor capacidad de cálculo sin aumentar con ello su peso y comprometer su autonomía. Con el aumento en el número de operaciones por segundo que estos equipos pueden realizar, se abre la posibilidad de emplear drones para aplicaciones antes impensables, aplicaciones que requieren de una gran cantidad de procesamiento y que, por lo general, hay que realizar en tiempo real.
\gls{UAV} technology has evolved very rapidly in recent years, benefiting from advances in computing. As processors are becoming more powerful, efficient and compact, \glspl{UAV} are increasingly capable of higher computing power without increasing their weight or compromising their autonomy. With the increase in the number of operations per second that \glspl{UAV} can perform, this opens up the possibility of using drones for previously unthinkable applications, applications that require a large amount of processing and usually have to be performed in real time.

%% Esfuerzo para el desarrollo de aplicaciones sin presencia humana: capacidades cogitivas, entornos dinámicos. (Aerial co-workers)
% Las aplicaciones actuales suelen necesitar presencia humana para llevar a cabo ciertas decisiones, siendo el piloto humano el que supervisa que todo se desarrolle correctamente y quien aporta la capacidad cognitiva para analizar el entorno generalmente dinámico y reaccionar a situaciones imprevistas. Esto se debe a que dotar a un UAV de capacidad cognitiva suficiente para operar de forma totalmente autónoma en entornos dinámicos es una tarea muy complicada y que además requiere de gran capacidad de procesamiento. El avance de la tecnología en este sentido está derribando una de las barreras que impiden a la tecnología UAV alcanzar ese nivel de autonomía, y con ello, se está destinando cada vez más esfuerzo de investigación para derribar la otra barrera, desarrollar software que permita a UAVs tener capacidad cognitiva.

%% Logros más recientes.
% Recientemente se ha completado el primer vuelo de un UAV en marte.

% Capi: quitar las subsecciones si veo que se puede integrar todo junto. Si veo que queda largo, dividir en subsecciones como tengo.
\subsection{UAVs}
\label{subsec:UAVs}

\subsection{Multi-drone teams}
\label{subsec:Multi-droneTeams}

\subsection{Aerial co-workers}
\label{subsec:AerialCo-workers}

% Related work: buscar artículos que tengan que ver con mi proyecto para poner en contexto lo que voy a aportar.
\section{Related work}
\label{sec:RelatedWork}

% Capi: quitar las subsecciones si veo que se puede integrar todo junto. Si veo que queda largo, dividir en subsecciones como tengo.
\subsection{Inspection applications with UAVs}
\label{subsec:InspectionApplicationsWithUAVs}

% Hablar de las formas existentes que hay para abordar el problema del reparto de tareas en equivos multi-UAV. Poner en contexto lo que voy a aportar con mi TFM.
\subsection{Task planning in multi-drone teams}
\label{subsec:TaskPlanning}

% Hablar de las formas existentes que hay para gestionar el comportamiento de un dron y su guiado en cada instante. Poner en contexto lo que voy a aportar con mi TFM. Hablar de las FSM y de los BT
\subsection{Drone behavior management}
\label{subsec:DroneBehaviorManagement}


% Estudio previo / Herramientas
% No hace falta explayarse mucho en ROS, Gazebo y Rviz, se da por supuesto en los que leen el documento, se cuenta brevemente en qué consisten esas herramientas y para qué se van a usar en el TFM
\section{Tools}
\label{sec:PreviousStudy}

\subsection{ROS}
\label{subsec:ROS}

\subsection{Gazebo}
\label{subsec:Gazebo}

\subsection{Rviz}
\label{subsec:Rviz}

\subsection{UAL}
\label{subsec:UAL}

\subsection{Behaviour Trees}
\label{subsec:BehaviourTrees}

\subsection{Groot}
\label{subsec:Groot}



\endinput
