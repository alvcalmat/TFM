\chapter*{Abstract}
\pagestyle{especial}
\chaptermark{Abstract}
\phantomsection
\addcontentsline{toc}{listasf}{Abstract}

%%% Hablar del problema que aborda el TFM.
% Este Trabajo de Fin de Máster ha afrontado problemas que surgen del reciente aumento de las aplicaciones de equipos cooperativos de UAV, los cuales son la autonomía para operar de forma prolongada en el tiempo con robustez ante posibles fallos, y la dificultad de aportar al equipo capacidades cognitivas para poder operar en entornos dinámicos con humanos. 
\lettrine[lraise=-0.1, lines=2, loversize=0.2]{T}{his} Master's Thesis has addressed problems arising from the recent increase in the applications of cooperative \glossary{UAV} teams, which are the autonomy to operate over a long period of time with robustness to possible failures, and the difficulty of providing the team with cognitive capabilities to be able to operate in dynamic environments with humans.

%%% Hablar de la importancia o del interés que hay por solucionar el problema.
% Muchas de estas aplicaciones están siendo ejecutadas actualmente por humanos, haciendo las actividaded mucho más costosas, lentas, e incluso en algunos casos, peligrosas. Es por eso que actualmente existe un gran interés y se están destinando muchos esfuerzos para desarrollar soluciones para los problemas planteados, ya que pueden suponer, además de un ahorro significativo para las empresas, una mejora drástica en la seguridad de los trabajadores en aquellos trabajos que sean de alto riesgo. Concretamente, la aplicación que en la que se ha centrado este trabajo es la asistencia a operarios humanos en tareas de inspección y mantenimiento en líneas eléctricas de alta tensión.
Many of these applications are currently being executed by humans, making the activities much more expensive, time-consuming, and in some cases even dangerous. This is why there is currently a great deal of interest and effort being put into developing solutions to the problems posed, as they can mean, in addition to significant savings for companies, a drastic improvement in the safety of workers in high-risk jobs. Specifically, the application on which this work has focused is the assistance to human operators in inspection and maintenance tasks on high-voltage power lines.

%%% Objetivos que se persiguen: ¿Por qué realizo esta investigación? ¿Qué se busca lograr? ¿Objetivo? ¿Hipótesis de partida?
% El objetivo del trabajo era desarrollar técnicas cognitvas de planificación para coordinar flotas de quadrotors que asistan a operarios humanos en tareas de inspección y mantenimiento en líneas eléctricas de alta tensión. Estas técnicas debían además extender la autonomía del sistema, garantizar que se cumplen los requisitos de seguridad entre drones y trabajadores humanos, y asegurar el éxito de la misión.
The aim of the work was to develop cognitive planning techniques for coordinating fleets of quadrotors to assist human operators in inspection and maintenance tasks on high-voltage power lines. These techniques should also extend the autonomy of the system, ensure that safety requirements between drones and human workers are met, and ensure the success of the mission.

%%% Descripción de la solución propuesta. ¿Cómo lo he hecho? ¿Técnicas utilizadas?
% Se ha propuesto una arquitectura de software basada en un planificador central, que se encarga de realizar la planificación de acciones en el tiempo y de controlar el estado de cada uno de los equipos conectados; y del gestor del comportamiento, distribuido a bordo de cada uno de los UAV, que se encarga de ejecutar el plan asignado por el módulo central. En el módulo distribuido se encuentra la mínima inteligencia que asegura el cumplimiento de los requisitos de seguridad, de forma que el resto de la inteligencia se pueda concentrar en el módulo centralizado. De esta forma se reduce lo máximo posible la carga computacional sobre los equipos aéreos, y por tanto, se alarga la vida de la batería. Se ha supuesto además que se dispone de alguna forma de recargar la batería durante la misión, por lo que entre las acciones de las que dispone el módulo central para planificar, se encuentra además la opción de recargar. Para llevar a cabo la planificación se ha definido un coste, que es calculado para cada tarea. Respetando entre otras cosas las prioridades de las tareas y su orden de llegada, cada tarea se asigna al UAV al que cueste menos su ejecución. Por el otro lado, para controlar el comportamiento de los drones y asegurar la seguridad de los equipos aéreos, se ha implementado un árbol de comportamiento.
A software architecture has been proposed based on a central planner, which is in charge of planning actions over time and monitoring the status of each of the connected equipment; and the behaviour manager, distributed on board each of the UAVs, which is in charge of executing the plan assigned by the central module. The distributed module contains the minimum intelligence that ensures compliance with safety requirements, so that the rest of the intelligence can be concentrated in the centralised module. This reduces the computational load on the aerial equipment as much as possible, thus extending battery life. It has also been assumed that there is some way of recharging the battery during the mission, so that among the actions available to the centralised module for planning, there is also the option of recharging. To carry out the planning, a cost has been defined, which is calculated for each task. Respecting, among other things, the priorities of the tasks and their order of arrival, each task is assigned to the \gls{UAV} that costs the least to execute. On the other hand, to control the behaviour of the drones and ensure the safety of the aerial equipment, a behaviour tree has been implemented.

%%% Resultados: Datos más importantes que respondan a las hipótesis y los objetivos marcados.
% Como resultado, se ha conseguido desarrollar una arquitectura de software capaz realizar la planificación de las misiones de forma dinámica asegurando mientras tanto la seguridad de los equipos involucrados. Gracias a la planificación, se consigue una mejor coordinación de los UAV y por tanto, un mejor aprovechamiento de la batería, alargrando así la autonomía de los equipos. El módulo central constituye una buena base que se puede adaptar fácilmente a otros proyectos que involucren equipos de múltiples UAV y de la cual se puede partir para desarrollar futuros planificadores más complejos. A su vez, el diseño de los módulos distribuidos, gracias al uso de árboles de comportamiento, permite una fácil reutilización y modificación. Comparado con la forma típica de implementación de este tipo de módulos, la cual involucra la creación de complejas máquinas de estados difíciles de leer para una persona, de reutilizar y de ampliar, el uso de árboles de comportamiento supone una gran mejora y permitirá la creación de comportamientos cada vez más complejos.
As a result, it has been possible to develop a software architecture capable of dynamically planning missions while ensuring the safety of the equipment involved. Thanks to the planning, a better coordination of the UAVs is achieved and therefore, a better use of the battery, thus extending the autonomy of the equipment. The core module provides a good base that can be easily adapted to other projects involving multi-UAV teams and from which more complex planners can be developed in the future. At the same time, the design of the distributed modules, thanks to the use of behaviour trees, allows for easy reuse and modification. Compared to the typical way of implementing such modules, which involves the creation of complex finite state machines that are difficult for a human to read, reuse and extend, the use of behaviour trees is a great improvement and will allow the creation of increasingly complex behaviours.

%%% Resumir la importancia de los resultados y de sus posibles aplicaciones.