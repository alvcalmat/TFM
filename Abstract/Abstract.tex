\chapter*{Abstract}
\pagestyle{especial}
\chaptermark{Abstract}
\phantomsection
\addcontentsline{toc}{listasf}{Abstract}

%%% Tal about the problem that TFM addresses.
\lettrine[lraise=-0.1, lines=2, loversize=0.2]{T}{his} Master's Thesis has addressed problems arising from the recent increase in the applications of cooperative \glossary{UAV} teams, which are the autonomy to operate over a long period of time with robustness to possible failures, and the difficulty of providing the team with cognitive capabilities to be able to operate in dynamic environments with humans.

%%% Talk about the importance or interest in solving the problem.
Many of these applications are currently being executed by humans, making the activities much more expensive, time-consuming, and in some cases even dangerous. This is why there is currently a great deal of interest and effort being put into developing solutions to the problems posed. 

%%% Objectives pursued: Why was this research carried on? What is the goal? ¿Objectives? Starting hypothesis?
The aim of the work was to develop cognitive planning techniques for coordinating fleets of quadrotors to assist human operators in inspection and maintenance tasks on high-voltage power lines. These techniques should also extend the autonomy of the system, ensure that safety requirements between drones and human workers are met, and ensure the success of the mission.

%%% Description of the proposed solution. How was it done? Used techniques?
A software architecture has been proposed based on a central planner and a distributed behaviour manager. To carry out the planning, a cost has been defined, which is calculated for each task. Thus, each one is assigned to the \gls{UAV} that costs the least to execute. On the other hand, to control the behaviour of the drones and ensure the safety of the aerial equipment, a behaviour tree has been implemented.

%%% Results: Most important data that respond to the objectives and hypothesis set.
As a result, it has been possible to develop a software architecture capable of dynamically planning missions while ensuring the safety of the equipment involved. This provides a good base that can be easily adapted and from which more complex planners can be developed in the future. Compared to the typical way of implementing behaviour managers, involving complex finite state machines that are difficult to read, reuse and extend, the use of behaviour trees is a great improvement and will allow the creation of increasingly complex behaviours.