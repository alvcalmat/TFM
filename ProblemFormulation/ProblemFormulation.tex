%% Capi: En los capítulos 3 y 4 puedes coger texto del documento que tenemos hecho con Giuseppe, y del proyecto tuyo de tesis.
\chapter{Problem Formulation}
\label{ch:ProblemFormulation}
\lettrine[lraise=-0.1, lines=2, loversize=0.2]{A}{s} mentioned in the chapter \ref{ch:Introduction}, the context around which this cognitive task planner is being developed is the inspection and maintenance of electrical networks. Although one of the objectives is to build a task planner whose characteristics allow its easy reuse and adaptation for other applications, it is relevant to state the problem for which it is being originally prepared. 

The AERIAL-CORE project (H2020-ICT-2019-871479) aims to develop different technologies for the use of multi-\gls{UAV} equipment in inspection and maintenance tasks in high-voltage electrical installations. In particular, one of the technologies proposed is the use of \glspl{ACW}, i.e. small teams of cooperative \glspl{UAV} to safely support maintenance workers while working at height on power lines. These systems would have to interact with humans (see Fig. \ref{fig:aerial_co_worker}) to inspect certain parts that are indicated to them, monitor worker safety during operation and deliver tools or other light equipment, in order to make the work more efficient and safer. In addition, to have a greater impact, the system would need to operate over extended periods of time, being able to autonomously deal with certain faults or recharges.

\begin{figure}[htbp]
    \centering
    \includegraphics[width=1\linewidth]
    {ProblemFormulation/figures/aerial_co_worker.jpeg}
    \caption{Multi-\gls{UAV} team supporting an operator. Source: \href{https://aerial-core.eu/}{Aerial-Core}}
    \label{fig:aerial_co_worker}
\end{figure}

Three types of \glspl{ACW} are referred, each intended to provide different functionality: \textit{Inspection-ACW}, \textit{Safety-ACW}, and \textit{Physical-ACW}. The use case scenarios can be summarized as follows: 

\begin{itemize}
    \item \textit{Inspection}, where a fleet of \glspl{ACW} (i.e., \textit{Inspection-ACWs}) carries out a detailed investigation of power equipment autonomously, helping the human workers to acquire views of the power tower that are not easily accessible (see Fig. \ref{fig:inspection_task});
    \item \textit{Safety}, where a formation of \glspl{ACW} (i.e., \textit{Safety-ACWs}) provides the supervising team with a view of the humans working on the power tower in order to monitor their status and to ensure their safety (see Fig. \ref{fig:monitor_task});
    \item \textit{Physical}, where an \gls{ACW} (i.e., \textit{Physical-ACW}) physically interacts with the human worker and provides physical assistance to it, i.e., while in contact with the human it flies stably, reliably, and accomplishes the required physical task (e.g., handover of a tool) without becoming harmful for the human worker (see Fig. \ref{fig:deliver_task}).
\end{itemize} 

Even if there is a specific type of \gls{ACW} for each of the tasks (inspection, monitoring and tool delivery), this does not mean that a \gls{UAV} can at any given time undertake a task for which it is not the best. It will therefore be the planner's task to take into account which \glspl{ACW} are best suited for each task, which are not but could perform it without problem, and which do not have the capacity to perform it at all. As a consequence, the number of ways in which the mission planning can be carried out multiplies, thus considerably increasing the difficulty of the problem that the task planner has to solve.

This mission planning problem with multiple \glspl{UAV} with battery constraints can be posed as an optimization problem, the solution of which indicates the most efficient way to allocate the different tasks and plan recharges. To react to possible failures, one of the most widespread options is to come up with dynamic methods that can replan in real time as certain events occur. Although there are many variants, most formulations for missions where multiple vehicles visit multiple locations to inspect or make deliveries give rise to NP-hard optimization problems and, therefore, the most widespread approach is to solve them using heuristic algorithms.

Uncertainty planning methods are appropriate for adding cognitive capabilities to a system that has to interact with humans in dynamic environments, as they allow optimizing plans by predicting the most likely intentions of humans and the outcomes of future actions. The main problem is their computational complexity, as the plan search space would grow exponentially with the number of \glspl{UAV} and with the future time horizon over which planning is to take place.

It is in this context and with these ideas in mind that the cognitive task planner was developed. As this is one of the software layers that make up the architecture that solve the problem, in order to present the details on which the planner has been designed it is necessary to at least talk about what information exchanges exist between the upper and lower layers of the software architecture, describe the interfaces by which this information travels and highlight when control is passed to lower levels of the software architecture. In the following section this information is presented by individually explaining the different tasks contemplated in the project. 

On the other hand, a review will be made of other important considerations that the planner must take into account such as battery recharges, connection losses and task rescheduling; analyzing the different situations in which each of them can occur and their different causes. 

\section{Description of tasks}
\label{sec:DescriptionOfTasks}
As mentioned above, three different types of tasks are envisaged in the project. These tasks are requested at any time by human workers through gestures. There will be a higher level software layer that processes the information contained in the gestures so that the planner receives an asynchronous communication from the upper layer containing the specifications of a new task. At this point, it is the planner's job to process the new information together with the information it already had in order to elaborate and implement a new plan. The same planner is also in charge of calling the low-level controllers when necessary and ensuring the safety of the equipment and the fulfillment of the mission. Each task is explained in detail below.

\subsection{Inspection tasks}
\label{subsec:InspectionTasks}
This task can be performed by all three types of \glspl{ACW}. It is the second highest priority task, with the tool delivery task being the only one that exceeds it. It consists of carrying out a detailed inspection of the specified areas of the power equipment. The layer immediately above the task planner is responsible for passing it a list of \glspl{WP} that define the inspection task, and the planner is responsible for deciding how many \glspl{ACW} it recruits to execute the task, which of the available \glspl{ACW} it assigns it to, and which subset of the list of \glspl{WP} it assigns to each one. In turn, once the planning is executed, the tasks of this type are transmitted to the lower level layers containing the subset of \glspl{WP} and the \glspl{ID} of the selected \glspl{UAV}.

All the above-mentioned communications will be carried out asynchronously, as the creation of the task by the workers, which triggers the whole sequence of actions, is done in this way.

\begin{figure}[htbp]
    \centering
    \includegraphics[width=1\linewidth]
    {ProblemFormulation/figures/inspection_task.pdf}
    \caption{\textit{Inspection-ACW} carrying out an inspection task}
    \label{fig:inspection_task}
\end{figure}

\subsection{Monitoring tasks}
\label{subsec:MonitoringTasks}
This task can also be executed by all three types of \glspl{ACW}. It is the lowest priority task. Monitor worker's safety consists of providing the supervisory team with a view of the people working in the power tower to monitor their status and ensure their safety. The layer immediately above the task planner communicates this time the \gls{ID} of the worker to be monitored, the number of \glspl{UAV} wanted and the distance they should keep from the worker. It is the task planner's responsability to decide once again which of the available \glspl{ACW} to assign this task to and the formation they should maintain during the flight. Once the planning has been executed, the tasks of this type are passed on to the lower level layers with both the original information and the information resulting from the planning.

The above-mentioned communications will also be carried out asynchronously for the same reason. 

\begin{figure}[htbp]
    \centering
    \includegraphics[width=0.5\linewidth]
    {ProblemFormulation/figures/monitor_task.pdf}
    \caption{\textit{Safety-ACW} carrying out a monitoring task}
    \label{fig:monitor_task}
\end{figure}

\subsection{Tool delivery tasks}
\label{subsec:ToolDeliveryTasks}
This task can be performed only by \textit{Physical-ACW} \glspl{UAV}, as special hardware is required to perform the physical interaction with the low-weight objects and the human. This is the highest priority task. Delivering a tool consists of picking up a tool and transporting it to the worker, with whom a physical interaction will take place through which the delivery of the tool will take place. Low-level controllers will have to be especially precise and careful not to hurt the worker. This time, the layer immediately above the task planner communicates the \gls{ID} of the worker to whom the tool is to be delivered and the \gls{ID} of the requested tool. Again, the task planner's mission is to decide which of the available \glspl{ACW} to give this task to. Once the planning is executed, tasks of this type are passed on to the lower level layers with the same information as originally.

The above-mentioned communications, once again, will be carried out asynchronously. 


\begin{figure}[htbp]
    \centering
    \includegraphics[width=1\linewidth]
    {ProblemFormulation/figures/deliver_task.png}
    \caption{\textit{Physical-ACW} carrying out a tool delivery task}
    \label{fig:deliver_task}
\end{figure}

% Otras consideraciones importantes a tener en cuenta: gestíón de la batería, desconexiones, imprevistos, prioridades, tipos de UAV.
% Sacar información del Informe de actividades
\section{Battery recharges}
\label{sec:BatteryRecharges}

\section{Connection losses}
\label{sec:ConnectionLosses}

\section{Task replanning situations}
\label{sec:TaskReplanningSituations}

