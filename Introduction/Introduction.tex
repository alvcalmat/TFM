\chapter{Introduction}
\label{ch:Introduction}
%%% Presentar el tema: Aerial co-workers: a task planning approach for multi-drone teams supporting inspection operations
% Definir el problema
\lettrine[lraise=-0.1, lines=2, loversize=0.2]{T}{he} use of \glspl{UAV} has grown considerably in recent years for numerous applications including real-time monitoring, search and rescue, providing wireless coverage, security and surveillance, precision agriculture, package delivery and infrastructure inspection \cite{CivilAplications}. With the rapidly developing technology in this area, and demonstrations of what \glspl{UAV} can do, there are increasing efforts to bring this technology to other applications. With the expected increase in applications for this technology, new problems and challenges arise, including autonomy, safety, obstacle avoidance and coordination of multi-\gls{UAV} teams. Developing the technology to solve these problems will be a major effort, but as \glspl{UAV} have proven to be critical in situations where humans are at high risk or highly inefficient and their capacity to evolve and develop even more potential in the short term, companies are investing in developing all sort of \gls{UAV}-based solutions.

\section{Motivation}
\label{sec:Motivation}
%%% Capi: motivación del problema:
% Con el incremento que ha sufrido la demanda eléctrica mundial, ha aparecido un reto para las compañías encargadas del suministro eléctrico relacionado con el mantenimiento y la reparación de las redes eléctricas de forma que se puedan minimizar la frecuencia de las averías. Según [PowerOutagesCauses], una de las principales causas de cortes eléctricos es el daño de las líneas de transmisión debido al mal tiempo o a campañas de inspección ineficientes.

% Por qué interesan los equipos multi-UAV para la inspección, principales barreras, etc. Puedes hablar del proyecto AERIAL-CORE como contexto del trabajo. Coje texto del paper que te pasé y del proyecto de tesis.

%%% Contexto y justificación del trabajo: 
% Dar razones de por qué es útil diseñar un planificador de tareas para equipos multi-UAV

\section{Objectives}
\label{sec:Objectives}
% Situación actual del ámbito investigado
% Antecedentes teóricos y teorías existentes (resultados de la revisión bibliográfica)
% Conceptos y definiciones clave

%%%: objetivos que se quieren alcanzar en tu TFM en concreto, dentro de todo el problema.
% Hablar en general del proyecto y de lo que quiero hacer.

%%% Preguntas de investigación e hipótesis: 
% Enumerar las hipótesis realizadas para diseñar el planificador

%\begin{hypothesis}\label{hyp:inicial}
%    "Dos \gls{ETSI} próximos entre sí provocarán patrones de error similares a la salida".
%\end{hypothesis}

\endinput

% Los UAV inteligentes son la siguiente gran revolución en esta tecnología, permitiendo su uso en aplicaciones como la inspección de infraestructuras de forma conjunta con humanos.