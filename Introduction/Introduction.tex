\chapter{Introduction}
\label{ch:Introduction}
\lettrine[lraise=-0.1, lines=2, loversize=0.2]{L}{o}rem itsum

% Hablar en general del proyecto y de lo que quiero hacer.

% Estudio teórico
% Programas usados, software empleado, entorno de programación
% Metodología de trabajo

% Dar razones de por qué es útil diseñar un planificador de tareas para equipos multi-UAV

% Enumerar las hipótesis realizadas para diseñar el planificador

\section{Motivation}
\label{sec:Motivation}
% Capi: motivación del problema: por qué interesan los equipos multi-UAV para la inspección, principales barreras, etc. Puedes hablar del proyecto AERIAL-CORE como contexto del trabajo. Coje texto del paper que te pasé y del proyecto de tesis.  

\section{Objectives}
\label{sec:Objectives}
% Capi: objetivos que se quieren alcanzar en tu TFM en concreto, dentro de todo el problema.

%\begin{hypothesis}\label{hyp:inicial}
%    "Dos \gls{ETSI} próximos entre sí provocarán patrones de error similares a la salida".
%\end{hypothesis}

\endinput
