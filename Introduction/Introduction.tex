\chapter{Introduction}
\label{ch:Introduction}

% Hablar en general del proyecto y de lo que quiero hacer.
\lettrine[lraise=-0.1, lines=2, loversize=0.2]{L}{a} primera vez que se observaron
los efectos de la radiación en satélites en órbita fue a mediados de la década de
1970. Desde entonces, los investigadores han estudiado sus efectos sobre
diferentes circuitos y tecnologías. La radiación puede ser un problema para los
circuitos destinados a trabajar en su presencia. Si esta es ionizante, puede dar
lugar a un \textit{\gls{SEE}}, o un efecto de evento singular, provocando un error 
en el circuito. Los daños que provoca la radiación se clasifican en dos grandes 
grupos: \textit{errores físicos ('hard errors')} y \textit{errores lógicos ('soft
errors')} \cite{TesisPoli}. Las \textit{conmutaciones por evento singular o 
\acrlong{SEU} (\acrshort{SEU})} son errores lógicos inducidos por radiación en el 
circuito que consisten en el cambio de valor de un biestable del mismo. No son 
daños permanentes, pero sí que pueden afectar al correcto funcionamiento del 
sistema.

% Importante decir que con la miniaturizacion de los circuitos esto también aplica
% en aviónica y a nivel del mar (referencia "Cosmic radiation comes to ASIC and SOC
% design" de Santarini y otras que encuentre)
Con la miniaturización de los circuitos, la dosis de radiación necesaria para 
provocar un \gls{SEU} es cada vez menor, con la consiguiente aparición de sus 
efectos cada vez a menor altitud \cite{EDN}. Esto acerca el problema de la 
radiación a aplicaciones más comunes como puede ser la aviación o las 
telecomunicaciones. A veces no importa, o es asumible, que un bit del circuito 
conmute a causa de radiación. Blindar un móvil frente a radiación o reforzar sus
circuitos con técnicas como la \textit{Redundancia Modular Triple o \gls{TMR}}
\cite{TMR} puede ser innecesario dado que el mayor riesgo al que nos exponemos es 
mínimo, pero esto no siempre es así. Cuando se trata de satélites, aviones, o 
incluso bases militares armadas, existen sistemas críticos cuyas misiones pueden 
ser el control orbital, la estabilización del vuelo o el lanzamiento de misiles, 
donde no son asumibles los errores que pueda provocar un \gls{SEU}.

% Dar razones de por qué puede ser interesante tener cierta capacidad de
% diagnístico de SEU
Diseñar circuitos resistentes a radiación puede ser un proceso costoso, complicado
y lento. Además, aplicar técnicas de refuerzo contra radiación a circuitos 
completos puede ser una pérdida de recursos \cite{SelectiveHardening}. Uno de los
pasos del diseño suele ser emplear una plataforma de inyección de fallos para
estudiar qué zonas del circuito son críticas y cuáles no necesitan ser reforzadas 
\cite{SelectiveHardening}. Si todo ha ido bien, uno de los últimos pasos suele ser
irradiar el circuito de forma real para verificar el diseño. Si se detectan 
irregularidades a la salida a causa de un \gls{SEU} ocurrido durante la prueba,
sería necesario rediseñar el circuito para reforzar aquellas zonas donde se hayan producido las conmutaciones.
Este proceso se vería enormemente beneficiado de una técnica que permita localizar
los \gls{SEU}, es decir, calcular el ciclo de reloj y biestable en el que ha
tenido lugar. Determinar la localización espacial y temporal de un \gls{SEU} se
denomina \textit{\gls{SEU} diagnosis} \cite{SEUDiagnosis}.

El problema al que nos enfrentamos al tratar de localizar un \gls{SEU} a partir de
la información de salida de un circuito crece exponencialmente con el tamaño del
circuito. Las escasas técnicas de diagnóstico  existentes hasta el momento 
requieren de diccionarios de fallos completos o exhaustivos, requisito que no
siempre es posible cumplir.

En la presente investigación hemos desarrollado una nueva técnica de diagnóstico
de \gls{SEU} basada en diccionarios de fallos incompletos o no exhaustivos. La
hipótesis de la que partimos es que: 

\begin{hypothesis}\label{hyp:inicial}
    "Dos \gls{SEU} próximos entre sí provocarán patrones de error similares a la 
    salida".
\end{hypothesis}

La mayor parte de la investigación se ha centrado en obtener métricas para 
examinar los patrones de salida desde distintas aproximaciones, ya que el parecido
o no de dos salidas depende mucho de cómo las observemos. En el desarrollo del
presente documento analizaremos y compararemos las métricas desarrolladas. Veremos
si alguna de ellas es mejor que otras, cómo podemos combinarlas en un único
algoritmo que realice el diagnóstico, si este mejora o empeora al prescindir
de alguna de las métricas fusionadas, y hasta qué punto podemos elevar la calidad
del diagnóstico empleando esta técnica.

\endinput
