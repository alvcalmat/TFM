\chapter{Introduction}
\label{ch:Introduction}
%%% Presentar el tema: Aerial co-workers: a task planning approach for multi-drone teams supporting inspection operations
% Definir el problema
\lettrine[lraise=-0.1, lines=2, loversize=0.2]{T}{he} use of \glspl{UAV} has grown considerably in recent years for numerous applications including real-time monitoring, search and rescue, providing wireless coverage, security and surveillance, precision agriculture, package delivery and infrastructure inspection \cite{CivilAplications}. With the rapidly developing technology in this area, and demonstrations of what \glspl{UAV} can do, there are increasing efforts to bring this technology to other applications. With the expected increase in applications for this technology, new problems and challenges arise, including autonomy, safety, obstacle avoidance and coordination of multi-\gls{UAV} teams. Developing the technology to solve these problems will be a major effort, but as \glspl{UAV} have proven to be critical in situations where humans are at high risk or highly inefficient and their capacity to evolve and develop even more potential in the short term, companies are investing in developing all sort of \gls{UAV}-based solutions.

\section{Motivation}
\label{sec:Motivation}
%%% Capi: motivación del problema:
With the increase in global electricity demand, a challenge has arisen for electricity supply companies to maintain and repair power grids in a way that minimizes the frequency of outages. According to \cite{PowerOutagesCauses}, one of the main causes of power outages is damage to transmission lines due to bad weather or inefficient inspection campaigns.

\begin{figure}[htbp]
    \centering
    \includegraphics[width=0.5\linewidth]
    {Introduction/figures/helicopter.jpg}
    \caption{Operators getting off the helicopter during a maintenance mission}
    \label{fig:helicopter}
\end{figure}

The strategy often used by electric companies to reduce power outages is to schedule periodic maintenance operations on active lines. This is the most suitable method if the correct functioning of the system is to be ensured and when replacing a circuit is unacceptable \cite{PowerOutagesCauses}. These maintenance missions are carried out by experienced crews on board helicopters and equipped with safety suits and harnesses among other things that prevent the operators from receiving an electric shock (see figure \ref{fig:helicopter}). The problem with this solution is that these activities are dangerous for the operators, as they are working at high altitude and on electrified lines, are extremely time-consuming and expensive (\$1500 per hour) and are subject to human error \cite{MaintenanceCost}.

% Por qué interesan los equipos multi-UAV para la inspección, principales barreras, etc. Puedes hablar del proyecto AERIAL-CORE como contexto del trabajo. Coje texto del paper que te pasé y del proyecto de tesis.
These are the reasons why distribution companies have the need to develop more efficient and safer maintenance methods. Multiple solutions have been proposed to automate this task \cite{MaintenanceSolutions}, but the best of them seems to be the use of \glspl{UAV} because of their flexibility and ability to inspect at different levels \cite{PowerOutagesCauses}. To achieve this, there are still some important barriers to overcome, such as the limited autonomy of these devices, the strong electromagnetic interference to which they would be subjected due to being close to power lines and the ability to detect and avoid obstacles of different nature that can be found in this type of environment \cite{MaintenanceCost}. Providing \glspl{UAV} with the cognitive capability to operate autonomously in such dynamic environments and with the presence of humans, and providing them with a rapid on-line planning method \cite{FastOnlinePlanning}, is key to address these complexities and to safely and successfully accomplish the assigned mission with \gls{UAV} fleets.

%%% Contexto y justificación del trabajo: 
% Dar razones de por qué es útil diseñar un planificador de tareas para equipos multi-UAV
A versatile and reliable software architecture will be essential to integrate and interconnect all the heterogeneous components that compose these cognitive multi-\gls{UAV} systems. 


In \cite{AerialCoreMulti-Layer} a multi-layer software architecture is presented that aims to carry out maintenance and inspection missions in high-voltage power lines in a cooperative way between human operators and a fleet of quadrotors. This architecture is part of the European Aerial-Core project, and this master's thesis has been responsible for the development of one of the stages that make up the software architecture presented.

\section{Objectives}
\label{sec:Objectives}
% Situación actual del ámbito investigado
% Antecedentes teóricos y teorías existentes (resultados de la revisión bibliográfica)
% Conceptos y definiciones clave

%%%: objetivos que se quieren alcanzar en tu TFM en concreto, dentro de todo el problema.
% Hablar en general del proyecto y de lo que quiero hacer.

%%% Preguntas de investigación e hipótesis: 
% Enumerar las hipótesis realizadas para diseñar el planificador

%\begin{hypothesis}\label{hyp:inicial}
%    "Dos \gls{ETSI} próximos entre sí provocarán patrones de error similares a la salida".
%\end{hypothesis}

\endinput

% Los UAV inteligentes son la siguiente gran revolución en esta tecnología, permitiendo su uso en aplicaciones como la inspección de infraestructuras de forma conjunta con humanos.