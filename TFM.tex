%%%%%%%%%%%%%%%%%%%%%%%%%%%%%%%%%%%%%%%%%%
%%% NORMALMENTE NO ES NECESARIO HACER 
%%% CAMBIOS EN ESTA PARTE DEL DOCUMENTO
%%%%%%%%%%%%%%%%%%%%%%%%%%%%%%%%%%%%%%%%%%


%:Clase del documento
\documentclass[fontsize=11pt, English=true, Myfinal=true, twoside, numbers=noenddot]{scrbook}
%Minion=true, English=true, Myfinal=true

%:Paquete de estilos propuesto
\usepackage{libroETSI}

%:Paquete específico para cargar tikz (y sus librerías) y pgfplots
\usepackage{dtsc-creafig}

%:Paquete para notaciones específicas
\usepackage{notacion}

%:Paquete para incorporar aspectos concretos de la edición
\usepackage{edicionPFC}

% Paquete para incluir epígrafes en los capítulos
\usepackage{epigraph}

% Paquete para incluir glosario
\usepackage{glossaries}

%:Para modificar fácilmente la fuente del texto.
\makeatletter
\ifdtsc@Minion % Queremos utilizar la fuente Minion y lo hemos declarado al principio
	\ifluatex
		\setmainfont[Renderer=Basic, Ligatures=TeX,	% Fuente del texto 
		Scale=1.01,
		]{Minion Pro}
   		% En este caso conviene modificar ligeramente el tamaño de las fuentes matemáticas
		\DeclareMathSizes{10}{10.5}{7.35}{5.25}
		\DeclareMathSizes{10.95}{11.55}{8.08}{5.77}
		\DeclareMathSizes{12}{12.6}{8.82}{6.3}
%		\setmainfont[Renderer=Basic, Ligatures=TeX,	% Fuente del texto 
%		]{Adobe Garamond Pro}
%		\setmainfont[Renderer=Basic, Ligatures=TeX,	% Fuente del texto 
%		]{Palatino LT Std}
	\fi
\else
	\ifluatex
		% Para utilizar la fuente Times New Roman, o alguna otra que se tenga instalada
		\setmainfont[Renderer=Basic, Ligatures=TeX,	% Fuente del texto 
		Scale=1.0,
		]{Times New Roman}
	\else
		\usepackage{tgtermes} 	%clone of Times
		%\usepackage[default]{droidserif}
		%\usepackage{anttor} 	
	\fi
\fi
\makeatother

% Formato A4
\geometry
{paperheight=297mm,%
paperwidth=210mm,%
top=25mm,%
headsep=8.5mm,%
includefoot, 
textheight=240mm, 
textwidth=150mm, 
bindingoffset=0mm, 
twoside}

\usepackage[a4,center]{crop}%para poner las cruces de esquina de página, poner la opción cross

%:Esquema de numeración por defecto
\setenumerate[1]{label=\normalfont\bfseries{\arabic*.}, leftmargin=*, labelindent=\parindent}
\setenumerate[2]{label=\normalfont\bfseries{\alph*}), leftmargin=*}
\setenumerate[3]{label=\normalfont\bfseries{\roman*.}, leftmargin=*}
\setlist{itemsep=.1em}
\setlength{\parindent}{1.0 em}

\setcounter{tocdepth}{4}						% El nivel hasta el que se muestra el índice 


%%%%%%%%%%%%%%%%%%%%%%%%%%%%%%%%%%%%%%%%%%
%%% A PARTIR DE AQUÍ HAY QUE EDITAR
%%%%%%%%%%%%%%%%%%%%%%%%%%%%%%%%%%%%%%%%%%

% Ejemplo de Glosario
\newacronym[type=main]{ETSI}{ETSI}{Escuela Técnica Superior de Ingeniería}
\newacronym[type=main]{US}{US}{Universidad de Sevilla}
%\newacronym[type=main]{}{}{}
\newacronym[type=main]{SEU}{SEU}{Single Event Upset}
\newacronym[type=main]{SEE}{SEE}{Single Event Effect}
\newacronym[type=main]{TMR}{TMR}{Triple Modular Redundancy}
\newacronym[type=main]{RBM}{RBM}{Restricted Boltzmann Machine}
\newacronym[type=main]{CUT}{CUT}{Circuit Under Test}
\newacronym[type=main]{FF}{FF}{flip-flop}
\newacronym[type=main]{ESA}{ESA}{European Space Agency}
\newacronym[type=main]{FPGA}{FPGA}{Field Programmable Gate Array}
\newacronym[type=main]{FFT}{FFT}{Fast Fourier Transform}
\newacronym[type=main]{FIFO}{FIFO}{First in, First out}
\newacronym[type=main]{FSM}{FSM}{Finite State Machine}
\newacronym[type=main]{UART}{UART}{Universal Asynchronous Receiver and Transmitter}


\makeindex
\makeglossaries %Si no se quiere el glosario, comentar esta línea.


%:Empieza el documento

\begin{document}


%PORTADA
%ver edicionPFC.sty para modificaciones

%:Para crear la portada y la portada interior (pagina titular)
\titulo{Aerial co-workers: a task planning approach for multi-drone teams supporting inspection operations} %\mbox evita que se divida una palabra al cambiar de línea
\autor{Álvaro Calvo Matos}
\director{Jesús Capitán Fernandez}
\titulodirector{Associate Professor}

\departamento{Dpto. Ingeniería de Sistemas y Automática}
%\departamento{Systems and Automation Engineering Department}
\centro{Escuela Técnica Superior de Ingeniería}
\universidad{Universidad de Sevilla}
%\universidad{University of Seville}
\titulacion{Máster en Ingeniería Electrónica, Robótica y Automática}
%\titulacion{Master in Electronic, Robotic and Automation Engineering}
\fecha{2021}
\nombretrabajo{Trabajo Fin de Máster} 


\hypersetup
	{
 	linkcolor=black, %Tocar para poner color en enlaces
	pdfauthor={\elautor},
	pdftitle={\nombretrabajo,\eltitulo}, 
	pdfkeywords={Latex, edición, formato de texto}	
	 }

%logo de la Universidad y logo del departamento, si lo hubiera. Para cambiar el pie de página con los logos, debe editarse el fichero ediciónPFC.sty
\portadaPFC{figuras/LogoUS.pdf}{figuras/LogoTSC.pdf} 
% Para incluir el logo del departamento hay que modificar el segundo parámetro de la linea anterior de este .tex, y
% hay que modificar las lineas 92 a 100 del fichero "edicionPFC.sty"

%Fin Portada

%:Todo lo que constituye la primera parte del libro que no es el cuerpo del libro en realidad
\frontmatter
\pagenumbering{Roman} %Pone la numeración en mayúscula (En español parece que es obligatorio)

%\include{dedicatoria/dedicatoria}%¿Comentar para proyectos/tesis?
\chapter*{Agradecimientos}
%\pagestyle{especial}
\pagestyle{empty}
%\chaptermark{Agradecimientos}
\phantomsection
%\addcontentsline{toc}{listasf}{Agradecimientos}
%\vspace{1cm}
%{\huge{Agradecimientos}}
%\vspace{1cm}

\lettrine[lraise=-0.1, lines=2, loversize=0.25]{}{}
Lorem itsum
% Tutor del proyecto:

% Compañeros del departamento

% Maestros de la carrera

% Compañeros de clase, por acompañarde durante todo el camino, en especial a 
% Damian por su apoyo y amistad en todo momento durante este último año.

% La familia

{\flushleft{\hfill \emph{Álvaro Calvo Matos}}}%
\vspace{-.3cm}
{\flushleft{\hfill \emph{Máster en Ingeniería Electrónica, Robótica y Automática}}}
{\flushleft{\hfill \emph{Sevilla, 2021}}}%


%PFC/PFM/TESIS
\chapter*{Abstract}
\pagestyle{especial}
\chaptermark{Abstract}
\phantomsection
\addcontentsline{toc}{listasf}{Abstract}
\lettrine[lraise=-0.1, lines=2, loversize=0.2]{L}{o}rem itsum
%%% Hablar del problema que aborda el TFM.
% Este Trabajo de Fin de Máster ha afrontado problemas que surgen del reciente aumento de las aplicaciones de equipos cooperativos de UAV, los cuales son la autonomía para operar de forma prolongada en el tiempo con robustez ante posibles fallos, y la dificultad de aportar al equipo capacidades cognitivas para poder operar en entornos dinámicos con humanos. 
This Master's Thesis has faced problems that arise from the recent increase in the applications of cooperative UAV teams, which are the autonomy to operate for a long time with robustness in the face of possible failures, and the difficulty of providing the team with capabilities cognitive skills to be able to operate in dynamic environments with humans. 

%%% Hablar de la importancia o del interés que hay por solucionar el problema.
% Muchas de estas aplicaciones están siendo ejecutadas actualmente por humanos, haciendo las actividaded mucho más costosas, lentas, e incluso en algunos casos, peligrosas. Es por eso que actualmente existe un gran interés y se están destinando muchos esfuerzos para desarrollar soluciones para los problemas planteados, ya que pueden suponer, además de un ahorro significativo para las empresas, una mejora drástica en la seguridad de los trabajadores en aquellos trabajos que sean de alto riesgo. Concretamente, la aplicación que en la que se ha centrado este trabajo es la asistencia a operarios humanos en tareas de inspección y mantenimiento en líneas eléctricas de alta tensión.

%%% Objetivos que se persiguen: ¿Por qué realizo esta investigación? ¿Qué se busca lograr? ¿Objetivo? ¿Hipótesis de partida?
% El objetivo del trabajo era desarrollar técnicas cognitvas de planificación para coordinar flotas de quadrotors que asistan a operarios humanos en tareas de inspección y mantenimiento en líneas eléctricas de alta tensión. Estas técnicas debían además extender la autonomía del sistema, garantizar que se cumplen los requisitos de seguridad entre drones y trabajadores humanos, y asegurar el éxito de la misión.

%%% Descripción de la solución propuesta. ¿Cómo lo he hecho? ¿Técnicas utilizadas?
% Se ha propuesto una arquitectura de software basada en un planificador central, que se encarga de realizar la planificación de acciones en el tiempo y de controlar el estado de cada uno de los equipos conectados; y del gestor del comportamiento, distribuido a bordo de cada uno de los UAV, que se encarga de ejecutar el plan asignado por el módulo central. En el módulo distribuido se encuentra la mínima inteligencia que asegura el cumplimiento de los requisitos de seguridad, de forma que el resto de la inteligencia se pueda concentrar en el módulo centralizado. De esta forma se reduce lo máximo posible la carga computacional sobre los equipos aéreos, y por tanto, se alarga la vida de la batería. Se ha supuesto además que se dispone de alguna forma de recargar la batería durante la misión, por lo que entre las acciones de las que dispone el módulo central para planificar, se encuentra además la opción de recargar. Para llevar a cabo la planificación se ha definido un coste, que es calculado para cada tarea. Respetando entre otras cosas las prioridades de las tareas y su orden de llegada, cada tarea se asigna al UAV al que cueste menos su ejecución. Por el otro lado, para controlar el comportamiento de los drones y asegurar la seguridad de los equipos aéreos, se ha implementado un árbol de comportamiento.

%%% Resultados: Datos más importantes que respondan a las hipótesis y los objetivos marcados.
% Como resultado, se ha conseguido desarrollar una arquitectura de software capaz realizar la planificación de las misiones de forma dinámica asegurando mientras tanto la seguridad de los equipos involucrados. Gracias a la planificación, se consigue una mejor coordinación de los UAV y por tanto, un mejor aprovechamiento de la batería, alargrando así la autonomía de los equipos. El módulo central constituye una buena base que se puede adaptar fácilmente a otros proyectos que involucren equipos de múltiples UAV y de la cual se puede partir para desarrollar futuros planificadores más complejos. A su vez, el diseño de los módulos distribuidos, gracias al uso de árboles de comportamiento, permite una fácil reutilización y modificación. Comparado con la forma típica de implementación de este tipo de módulos, la cual involucra la creación de complejas máquinas de estados difíciles de leer para una persona, de reutilizar y de ampliar, el uso de árboles de comportamiento supone una gran mejora y permitirá la creación de comportamientos cada vez más complejos.

%%% Resumir la importancia de los resultados y de sus posibles aplicaciones. 

% Índice abreviado 
% El índice abreviado se incluye también en algunos libros, con menor detalle que el completo. Descomentar las siguientes líneas.
\cleardoublepage
\phantomsection
\addcontentsline{toc}{listasf}{Abbreviated index}
\pagestyle{especial}
\shorttoc{Abbreviated index}{1}

%Índice normal, el completo
\cleardoublepage
\phantomsection
\pagestyle{especial}
\tableofcontents

%%%%%%%%%%%%%%%%%%%%%%%%%%%%%%%%%%%%%%%%%%%%%%%%%%%%%%%%%%%%%%%%%%%%%%%%%%%%%%%
%%%%%%% Descomentar la siguiente linea y editar notacion.tex si hiciera falta
%%%%%%% incluir notación en el TFG.
%\chapter*{\notationname}
\pagestyle{especial}
\chaptermark{\notationname}
\phantomsection
\addcontentsline{toc}{listasf}{\notationname}
%\section*{Notación}
%\begin{table}[htbp]
\begin{longtable}{p{3cm}p{8.5cm}}

%$\displaystyle D$ & Tasa de símbolos  (sim/s) \\
%$\displaystyle R_b$ & Tasa binaria (bit/s) \\
%$\displaystyle T$ & Tiempo de símbolo (s) \\
%$\displaystyle T_{b}$ & Tiempo de bit (s) \\
%$W\left( {t} \right)$ & Ruido blanco\\
%$w\left( {t} \right)$ & Función muestra de un ruido blanco\\
%$\displaystyle h_{c}\left( {t} \right)$ & Respuesta impulsiva de un canal LTI continuo en el tiempo\\
%$\displaystyle H_{c}\left( {\omega} \right)$ & Respuesta en frecuencia de un canal LTI continuo en el tiempo\\
%$\displaystyle h_{c}\left( {\tau;t} \right)$ & Respuesta impulsiva de un canal LTV continuo en el tiempo\\
%$\displaystyle H_{c}\left( {\omega;t} \right)$ & Respuesta en frecuencia de un canal LTV continuo en el tiempo\\
%$\displaystyle h_{c}\left( {n} \right)$ & Respuesta impulsiva de un canal LTI discreto en el tiempo\\
%$\displaystyle H_{c}\left( {\Omega} \right)$ & Respuesta en frecuencia de un canal LTI discreto en el tiempo\\
$\RR$ & Cuerpo de los números reales \\
$\CC$ & Cuerpo de los números complejos\\
$\left\| \vc{v} \right\|$ & Norma del vector $\vc{v}$ \\
$\left\langle {\vc{v}, \vc{w}} \right\rangle$ & Producto escalar de los vectores $\vc{v}$ y $\vc{w}$\\
$\left| {\vc{A}} \right|$ &Determinante de la matriz cuadrada $\vc{A}$\\
$\textrm{det}\left( {\vc{A}} \right)$ &Determinante de la matriz (cuadrada) $\vc{A}$\\
$\vc{A}\trs$ & Transpuesto de $\vc{A}$\\
$\vc{A}\inv$ & Inversa de la matriz $\vc{A}$\\
$\vc{A}{\psd}$ & Matriz pseudoinversa de la matriz $\vc{A}$\\
$\vc{A}\her$ & Transpuesto  y conjugado de $\vc{A}$\\
$\vc{A}\cnj$ & Conjugado\\
c.t.p. & En casi todos los puntos\\
c.q.d. & Como queríamos demostrar\\
\ensuremath{\blacksquare}& Como queríamos demostrar\\
\ensuremath{\square}& Fin de la solución\\
e.o.c. & En cualquier otro caso\\
$\e$ & número e\\
$\xp{x}$ & Exponencial compleja\\
$\xppi{x}$ & Exponencial compleja con $2\pi$\\
$\nxp{x}$ & Exponencial compleja negativa\\
$\nxppi{x}$ & Exponencial compleja negativa con $2\pi$\\
$\re$ & Parte real\\
$\im$ & Parte imaginaria\\
$\sen$ & Función seno \\
$\tg$ & Función tangente \\
$\arctg$ & Función arco tangente \\
$\sento{y}{x}$ & Función seno de $x$  elevado a $y$\\
$\costo{y}{x}$ & Función coseno de $x$  elevado a $y$\\
$\sa$ & Función sampling \\
$\sgn$ & Función signo \\
$\rect$ & Función rectángulo \\
$\sinc$ & Función sinc\\
$\pder{y}{x} $ & Derivada parcial de $y$ respecto a $x$\\
$x\gra$ & Notación de grado, $x$ grados.\\
%
%$C_{XY}$& covarianza  de dos variables aleatorias reales $X$ e $Y$\\
%$R_{XY}$& correlación  de dos variables aleatorias reales $X$ e $Y$\\
%$\rho_{XY}$ &Coeficiente de correlación de las variables aleatorias reales $X$  e $Y$\\
%$\vc{Z}$ & Vector aleatorio complejo\\
%$\displaystyle F_{X}\left( {\cdot} \right)$ & Función de distribución de la variable aleatoria $X$ \\
%$\displaystyle f_{X}\left( {\cdot} \right)$ & Función densidad de probabilidad de la variable aleatoria $X$ \\
%$p_{X}\left( {\cdot} \right)$ & Función masa de probabilidad de la variable aleatoria discreta $X$ \\
%
$\Pr\left( {A} \right)$ & Probabilidad del suceso $A$ \\
$\displaystyle E\left[ {X} \right]$ & Valor esperado de la variable aleatoria $X$ \\
$\si{X}$ & Varianza de la variable aleatoria $X$\\
$\sim f_{X}\left( {x} \right)$ & Distribuido siguiendo la función densidad de probabilidad $f_{X}\left( {x} \right)$\\
%
$\gauss{m_{X}}{\si{X}}$ &Distribución gaussiana para la variable aleatoria X, de media $m_{X}$ y varianza $\si{X}$ \\
$\id{n}$ & Matriz identidad de dimensión $n$\\
$\diag{\vc{x}}$ & Matriz diagonal a partir del vector $\vc{x}$\\
$\diag{\vc{A}}$ & Vector diagonal de la matriz $\vc{A}$\\
$\snr$& Signal-to-noise ratio \\
$\mse$ & Minimum square error\\
$\talq$ & Tal que \\
$\eqdef$ & Igual por definición \\
$\norm{\vc{x}}$ & Norma-2 del vector $\vc{x}$\\
$\card{\vc{{A}}}$ & Cardinal, número de elementos del conjunto $\vc{A}$\\
$\xyz{\vc{x}}{i}{n}$ & Elementos $i$, de 1 a $n$, del vector $\vc{x}$\\
%\newcommand{\xyz}[3]{\ensuremath{#1_{#2},#2=1,2,\ldots,#3}}
$\df{x}$& Diferencial de $x$\\
$\le$ & Menor o igual \\
$\ge$ & Mayor o igual \\
$\BL$ & Backslash \\
$\iff$ & Si y sólo si \\
$x=a+3\eqexpl{a=1} 4 $& Igual con explicación \\
$\tfrac{a}{b}$ & Fracción con estilo pequeño, $a/b$ \\
$\inc$ & Incremento \\
$b\ten{a}$ & Formato científico \\
$\tendsub{x}$ & Tiende, con x \\
$\ord$ & Orden\\
$\tm$ & Trade Mark\\
$\E[x]$ & Esperanza matemática de x\\
$\covm{\vc{x}}$ & Matriz de covarianza de $\vc{x}$\\
$\corrm{\vc{x}}$ & Matriz de correlación de $\vc{x}$\\
$\si{x}$ & Varianza de x \\


\end{longtable}
\newpage
%\end{table}
%


%\phantomsection
%\addcontentsline{toc}{listasf}{Acrónimos}
%\section*{Acrónimos}
%\begin{table}[htbp]
%\begin{tabular}{p{2cm}p{10cm}}
%Escuela Técnica Superior de In
%LTI & Lineal Invariante con el Tiempo \\
%LTV& Lineal Variable con el Tiempo\\
%AWGN& Ruido blanco gaussiano aditivo\\
%DMS& Fuente discreta sin memoria\\
%AEP& Propiedad de equipartición asintótica\\
%WLLN& Ley Débil de los Grandes Números\\
%DMC& Canal Discreto sin Memoria\\
%BSC& Canal Simétrico Binario\\
%BEC& Canal Binario con Borrado\\
%\end{tabular}
%\end{table}


%\nota{El libro de Lapidoth tiene una excelente recopilación.} %No incluir si no se quiere, comentándolo

%:Empieza el contenido del libro
\mainmatter

%:Página por defecto
\pagestyle{esitscCD}

%%%%%%%%%%%%%%%%%%%%%%%%%%%%%%%%%%%%%%%%%%%%%%%%%%%%%%%%%%%%%%%%%%%%%%%%%%%%%%%
%%%%%%% Incluir los diferentes capítulos del TFG en carpetas separadas.
%:Los diferentes capítulos, en carpetas separadas
%
\chapter{Introduction}
\label{ch:Introduction}
%%% Presentar el tema: Aerial co-workers: a task planning approach for multi-drone teams supporting inspection operations
% Definir el problema
\lettrine[lraise=-0.1, lines=2, loversize=0.2]{T}{he} use of \glspl{UAV} has grown considerably in recent years for numerous applications including real-time monitoring, search and rescue, providing wireless coverage, security and surveillance, precision agriculture, package delivery and infrastructure inspection \cite{CivilAplications}. With the rapidly developing technology in this area, and demonstrations of what \glspl{UAV} can do, there are increasing efforts to bring this technology to other applications. With the expected increase in applications for this technology, new problems and challenges arise, including autonomy, safety, obstacle avoidance and coordination of multi-\gls{UAV} teams. Developing the technology to solve these problems will be a major effort, but as \glspl{UAV} have proven to be critical in situations where humans are at high risk or highly inefficient and their capacity to evolve and develop even more potential in the short term, companies are investing in developing all sort of \gls{UAV}-based solutions.

\section{Motivation}
\label{sec:Motivation}
%%% Capi: motivación del problema:
% Con el incremento que ha sufrido la demanda eléctrica mundial, ha aparecido un reto para las compañías encargadas del suministro eléctrico relacionado con el mantenimiento y la reparación de las redes eléctricas de forma que se puedan minimizar la frecuencia de las averías. Según [PowerOutagesCauses], una de las principales causas de cortes eléctricos es el daño de las líneas de transmisión debido al mal tiempo o a campañas de inspección ineficientes.

% Por qué interesan los equipos multi-UAV para la inspección, principales barreras, etc. Puedes hablar del proyecto AERIAL-CORE como contexto del trabajo. Coje texto del paper que te pasé y del proyecto de tesis.

%%% Contexto y justificación del trabajo: 
% Dar razones de por qué es útil diseñar un planificador de tareas para equipos multi-UAV

\section{Objectives}
\label{sec:Objectives}
% Situación actual del ámbito investigado
% Antecedentes teóricos y teorías existentes (resultados de la revisión bibliográfica)
% Conceptos y definiciones clave

%%%: objetivos que se quieren alcanzar en tu TFM en concreto, dentro de todo el problema.
% Hablar en general del proyecto y de lo que quiero hacer.

%%% Preguntas de investigación e hipótesis: 
% Enumerar las hipótesis realizadas para diseñar el planificador

%\begin{hypothesis}\label{hyp:inicial}
%    "Dos \gls{ETSI} próximos entre sí provocarán patrones de error similares a la salida".
%\end{hypothesis}

\endinput

% Los UAV inteligentes son la siguiente gran revolución en esta tecnología, permitiendo su uso en aplicaciones como la inspección de infraestructuras de forma conjunta con humanos.
%
\chapter{State of the art}
\label{ch:StateOfTheArt}

% Poner en contexto las tecnologías que hay hoy día y demás
% Related work: buscar artículos que tengan que ver con mi proyecto para poner en contexto lo que voy a aportar.

\section{Detección de fallos (\textit{Fault Detection})}
\label{sec:FaultDetection}
Dado que no es posible realizar
un diagnóstico de \gls{SEU} sin detectarlo primero, numerosos estudios se centran
en desarrollar técnicas que permitan detectarlos a tiempo para suprimir sus 
efectos. Por ejemplo, en 2014, un equipo chino presentó una técnica de detección 
de \gls{SEU} basada en la \textit{Máquina de Boltzman Restringida o \gls{RBM}}, 
bloque fundamental en muchos algoritmos de \textit{Deep Learning} 
\cite{RBMSEUdetection}. En \cite{SCARA} abordan el problema de \textit{faul 
detection} por el modelo dinámico del sistema. Comparan las lecturas tomadas por
los sensores con los valores teóricos que se obtienen del modelo dinámico del
robot SCARA. De esta forma detectan anomalías debidas a radiación. En un estudio
más reciente, enfocado a sistemas embebidos, emplean programas de detección por
software. Multitud de hilos se ejecutan simultáneamente y se encargan de examinar
el circuito con el objetivo de detectar alguna irregularidad causada por
radiación \cite{DetectingSEUs}.


% En esta seccion contaría que lo que existe principalmente es para fallos de
% fabricación.
% FAULT LOCATION
\section{Diagnóstico de fallos o localización de fallos}
\label{sec:FaultLocation}
Hasta ahora, el diagnóstico de fallos ha sido poco estudiado, siendo los fallos de
fabricación a los que más esfuerzos de investigación se les ha dedicado
\cite{VLSI, EfficientSA0SA1, RepairSA0SA1, LargeComb, ANewRep, FILC, FDIRC}.
Estos no son el tipo de fallos que nos interesa diagnosticar en esta
investigación, ya que no son causados por radiación, si no que se producen, como
su nombre indica, en el momento de fabricación del circuito (\textit{stuck-at-0,
stuck-at-1}).

Las técnicas existentes para localización de fallos provocados por radiación se
basan principalmente en el uso de diccionarios de fallos, aunque también se
emplean vectores de prueba, listas de fallos, tabla de verdad de nodos
(\textit{"node truth table"}) y tabla de conexiones de pines (\textit{pin
connection table}) \cite{DiagnosisTechniques, LASAR, RTFDandD}. 

A excepción de contados estudios, la mayoría de los revisados modelan al 
circuito bajo prueba o \textit{\gls{CUT}} como una caja negra, es decir, el diseño
del circuito no se conoce y solo las salidas pueden ser monitorizadas.
Normalmente, el número de biestables del circuito es mucho mayor que el número de
salidas, por lo que es necesario observar el circuito el suficiente tiempo como
para detectar patrones que puedan ser asociados a la localización de un
determinado \gls{SEU} \cite{SEUDiagnosis}. Estas huellas son registradas y
almacenadas en un diccionario durante una fase previa al diagnóstico.

El diccionario de fallos se genera mediante inyección de fallos, en alguna
plataforma que lo permita \cite{FastFI, LeonFI, FTU}, y contiene información de 
la localización de los \gls{SEU} inyectados y el patrón de salidas que produce. 
Si el diccionario recoge todas las posibilidades, se habla de diccionario 
completo o exhaustivo, tomando el nombre de la campaña de inyección de fallos 
necesaria para generarlo (\textit{Campaña Exhaustiva}). En el caso contrario, es 
un diccionario incompleto o no exhaustivo, es decir, no todas las posibles 
combinaciones de (biestable, ciclo) han sido inyectadas y almacenadas en el 
diccionario. 

% La localización del \gls{SEU}, una vez detectado, se consigue comparando el 
% patrón de salida generado con la información contenida en los diccionarios.
Durante la fase de diagnóstico, para localizar un \gls{SEU} detectado, se compara
el patrón de error que genera en las salidas del circuito con los patrones
almacenados en el diccionario. Debido al largo tiempo de observación comentado, la
información a comparar puede tener un tamaño considerable, y por tanto el tiempo
necesario para procesar la comparación es alto. Una solución para reducir esta
cantidad de información y por tanto, el tiempo, permitiendo incluso localización
de \gls{SEU} en tiempo real, es comprimirla. Un ejemplo sería el uso de códigos 
HASH \cite{SEUDiagnosis}.

% This premise is pretty ambitious since supposes thatthere  is  an  unique
% signature  for  a  couple  of  clock  cycle  andFF  and  viceversa,  i.e.  this
% relation  is  univocal.  In  general,  itis not true, and several different
% injected faults can give placeto exactly the same signature at the outputs. In
% this scenario,it  is  also  obvious  that  the  greater  the  time  we  analyze
% theoutputs,  the  greater  the  possibility  to  get  different  signaturesfor
% different injected faults.

% Problema de que no se llegue a un único candidato, sino a una lista
% Problema de las colisiones
% There is another problem affecting injectivity related to theCUT  itself.  In  a
% fault  injection  campaign,  when  several  runsare  performed,  it  is
% possible  that  the  CUT  shows  exactly  thesame outputs for different fault
% injections, and not only for theGOLDEN outputs but also for wrong outputs. In
% such cases,the fault dictionary is no longer univocal or unambiguous.
Dada la gran cantidad de biestables existentes en comparación con el reducido
número de salidas, no es difícil imaginar la posibilidad de que dos \gls{SEU}
localizados en biestables y/o ciclos distintos produzcan exactamente el mismo
patrón de error a la salida, al menos durante el tiempo y test programados. 
Cuando esto ocurre, se habla de \textit{"Colisión"}. Además, es posible que un
\gls{SEU} no produzca error alguno a la salida durante el test, siendo
indistinguible de una situación libre de conmutaciones. Ante estas situaciones,
existirá más de una entada del diccionario que coincida con la buscada. Como
resultado del diagnóstico se obtienen no una si no una lista de posibles
localizaciones para el \gls{SEU} bajo diagnóstico.

% Explicar que el problema de la técnica basada en códigos hash es que necesita 
% de diccionarios exhaustivos, y generarlos es inviable para circuitos grandes.
Hasta ahora hemos hablado de diagnóstico empleando diccionarios de fallos
completos, pero si el \gls{CUT} es grande, obtener un diccionario exhaustivo es
una operación inviable, ya que la cantidad de combinaciones biestable-ciclo a
inyectar para ello se vuelve inabarcable. Si se intenta diagnosticar un \gls{SEU}
empleando un diccionario de fallos incompleto, aparecen nuevos problemas, ya que
puede ocurrir que la ubicación correcta no se haya inyectado durante la prueba, y
por tanto no se encuentre en el diccionario. Si además existe una colisión que sí
se ha inyectado, el diagnóstico concluirá con una localización única
aparentemente correcta que puede no se acerque nada a la real.


\endinput

% 
%\chapter{Teoric Approach}
\label{ch:TeoricApproach}

% Estudio teórico
% Programas usados, software empleado, entorno de programación
% Metodología de trabajo
% Hablar de las diferentes formas existentes de abordar los dos problemas a solucionar:
%     Formas de afrontar el reparto de tareas
%     Formas de afrontar el control del comportamiento de los Agentes (BT, FSM)

\lettrine[lraise=-0.1, lines=2, loversize=0.2]{L}{o}rem itsum

%
\chapter{Problem Description}
\label{ch:ProblemDescription}

% Descripcion del proyecto para el que se va a diseñar el planificador de tareas.

% Descripcion de la lista de tareas contempladas y explicación de cada una
% ¿? ¿Decir algo sobre los gestos? Se supone que esto es para Piloting y que ahí no hay gestos. De todas formas, mi parte es ajena a los gestos, le llegan las tareas ya procesadas.

% Otras consideraciones importantes a tener en cuenta: gestíón de la batería, desconexiones, imprevistos, prioridades, tipos de UAV.
\lettrine[lraise=-0.1, lines=2, loversize=0.2]{L}{o}rem itsum

%
\chapter{Design of the proposed solution}
\label{ch:DesignOfTheProposedSolution}
\lettrine[lraise=-0.1, lines=2, loversize=0.2]{T}{his} section provides more details about the implementation of the solution to the problem: node diagram, pseudocode and inter-module communications. All the code is available online\footnote{Human aware collaboration planner source code: \url{https://github.com/grvcTeam/aerialcore_planning}}, and was developed under the Ubuntu 18.04 operating system and ROS Melodic.

The solution proposed for the problem formulated in the previous section (see section \ref{ch:ProblemFormulation}) follows a hierarchical approach, with a high-level planner in charge of activating different low-level controllers. The high-level planner detects the tasks required by the operators, and distributes them from the ground in a centralised way among the available \glspl{ACW}, planning the necessary reloads throughout the mission. In addition, this planner reacts in real time to possible failures by reassigning tasks. The low-level planners are on board each \gls{UAV} and are responsible for executing contingency plans for these failures while the central planner calculates and communicates the new plan. They will also be in charge of controlling the movement of the \glspl{ACW} to execute the different tasks assigned by the higher-level module (e.g. flying to a location to be inspected or to the position of an operator waiting for a tool). From now on, the low-level module on board each \gls{UAV} will be called the \emph{Agent Behaviour Manager}, and the centralised module on the ground will be called the \emph{High-Level Planner}. Together, these modules have cognitive capabilities enough to interact with humans efficiently. 

%%%%%% ATENTION %%%%%%%%%%%%%
% (\emph{No se si quitar esta última frase, este párrafo es adaptado del proyecto de tesis})

% Hasta ahora he explicado de qué módulos se compone la solución y que hacen pero no cómo lo hace.
% En "sec:NodeDiagram" comenzaremos explicando el diagrama de nodos (con la fig:NodeDiagram por delante) y luego describiremos de forma general el proceso que siguen los módulos
% En "sec:Centralized module:TaskPlanner" y "sec:Distributed module: behavior manager" se explicará cómo hacen lo que hacen 

\section{Node diagram}
\label{sec:NodeDiagram}
%% Informe de actividades: Node Diagram
As stated in chapter \ref{ch:Introduction}, the developed task planner is part of a software architecture consisting of different layers, being the main cognitive block the central layer, the \emph{high-level cognitive task planner}. Figure \ref{fig:NodeDiagram} shows a schematic of the software architecture from the perspective of the module implemented in this project, including the nodes that form it and their interfaces. The part of the diagram painted in grey would be the complete software architecture, including from the high-level module in charge of analyzing the gestures made by the operators to extract the tasks from them; to the low-level controllers in charge of executing those tasks. The software layer corresponding to this thesis, in charge of high-level decision-making, is marked in blue-green. It is composed of the \emph{High-Level Planner}, which is centralised and runs on a ground station, colored in orange; and the \emph{Agent Behaviour Manager}, distributed on board each \gls{ACW}, painted in lime.

\begin{figure}[ht]
    \hspace{-1cm}
	\scalebox{0.7}{
		\begin{tikzpicture}
    		% WP7 block
    		\node (WP7-Box) at (8.35,0) [fill=gray!15,rounded corners, draw=black!70, densely dotted, minimum height=5cm, minimum width=19.5cm]{}; 

			% Task planner box
    		\node (TaskPlannerBox) at ($(WP7-Box)+(0,0)$) [fill=teal!15,rounded corners, draw=black!70, densely dotted, minimum height=4.5cm, minimum width=14cm]{};
    		
    		% Gesture Recognition
    		\node (GestureRecognition) at (0,0) [text centered, fill=white, draw, rectangle, minimum width=1.5cm, text width=5.5em]{Gesture\\Recognition};
    		
    		\draw[-latex] ($(GestureRecognition) - (1.8,0)$) -- (GestureRecognition);
 
    		% High-Level Planner
    		\node (HighLevelPlannerBox) at ($(GestureRecognition) + (3.5,0.25)$) [fill=orange!15,rounded corners, draw=black!70, densely dotted, minimum height=2cm, minimum width=2.5cm]{}; 
    		\node (HighLevelPlanner) at ($(HighLevelPlannerBox) + (0,-0.25)$) [text centered, fill=white, draw, rectangle, minimum width=1.5cm, text width=5.5em]{High-Level\\Planner};
    		\node (Centralized) at ($(HighLevelPlanner) + (0,0.75)$) [text centered]{\small Centralized};
    		
    		\draw[-latex] (GestureRecognition.east) -- node[above]{Task} (HighLevelPlanner);
    		
    		%%%%%%%%%%%%%%%%%%%
    		% UAV 1
    		\node (UAV1) at ($(HighLevelPlanner) + (6.75,1.25)$) [fill=lime!20,rounded corners, draw=black!70, densely dotted, minimum height=1.7cm, minimum width=5cm]{}; 
    		\node (AgentBehaviourManager1) at ($(UAV1) + (0,-0.25)$) [fill=white, draw, rectangle, text centered, text width=12em]{Agent Behaviour Manager};
    		\node (UAV1-Text) at ($(AgentBehaviourManager1) + (0,0.75)$) [text centered]{\small On board ACW-$1$};	

    		\draw[fill=black] ($ (HighLevelPlanner.east) + (1.715,0) $) arc(-180:180:0.05);
    		\draw[-latex] (HighLevelPlanner.east) -- ($ (HighLevelPlanner.east) + (1.75,0) $) -- ($ (HighLevelPlanner.east) + (1.75,1) $) -- node[above]{Task} node[below]{List} (AgentBehaviourManager1.west);
    		\draw[-latex] ($ (HighLevelPlanner.east) + (1.75,0) $) -- node[above]{Feedback} (HighLevelPlanner.east);
    		
    		%%%%%%%%%%%%%%%%%%%
    		
    		% Dots
    		\node (Dots2) at ($(UAV1) + (0,-1.25)$) [text centered]{\dots};
    		
    		%%%%%%%%%%%%%%%%%%%
    		
    		% UAV N
    		\node (UAVN) at ($(HighLevelPlanner) + (6.75,-1.25)$) [fill=lime!20,rounded corners, draw=black!70, densely dotted, minimum height=1.7cm, minimum width=5cm]{}; 
    		\node (AgentBehaviourManagerN) at ($(UAVN) + (0,-0.25)$) [fill=white, draw, rectangle, text centered, text width=12em]{Agent Behaviour Manager};
    		\node (UAVN-Text) at ($(AgentBehaviourManagerN) + (0,0.75)$) [text centered]{\small On board ACW-$N$};	

    		\draw[-latex] (HighLevelPlanner.east) -- ($ (HighLevelPlanner.east) + (1.75,0) $) -- ($ (HighLevelPlanner.east) + (1.75,-1.5) $) -- node[above]{Task} node[below]{List} (AgentBehaviourManagerN.west);
    		
    		%%%%%%%%%%%%%%%%%%%
    		
    		% Lower-Level Controllers
    		\node (LowerLevelControllers) at ($(HighLevelPlanner) + (13.25,0)$) [text centered, fill=white, draw, rectangle, minimum width=1.5cm, text width=5.5em]{Lower-Level\\Controllers};
    		
    		\draw[-latex] (LowerLevelControllers.east) -- ($(LowerLevelControllers) + (1.8,0)$);
    		
    		\draw[fill=black] ($ (LowerLevelControllers.west) + (-1.535,0) $) arc(-180:180:0.05);
    		\draw[-latex] (AgentBehaviourManager1.east) -- ($ (LowerLevelControllers.west) + (-1.5,1) $) -- ($ (LowerLevelControllers.west) + (-1.5,0) $) --  node[above]{Task}
    	    node[below]{Params}	(LowerLevelControllers.west);
    		\draw[-latex] (LowerLevelControllers.west) -- ($ (LowerLevelControllers.west) + (-1.5,0) $) -- ($ (LowerLevelControllers.west) + (-1.5,1) $) -- node[above]{Task}
    	    node[below]{Result}	(AgentBehaviourManager1.east);
    		\draw[-latex] (AgentBehaviourManagerN.east) -- ($ (LowerLevelControllers.west) + (-1.5,-1.5) $) -- ($ (LowerLevelControllers.west) + (-1.5,0) $) -- (LowerLevelControllers.west);
    		\draw[-latex] (LowerLevelControllers.west) -- ($ (LowerLevelControllers.west) + (-1.5,0) $) -- ($ (LowerLevelControllers.west) + (-1.5,-1.5) $) -- 
    		node[above]{Task}
    		node[below]{Result} (AgentBehaviourManagerN.east);
    		
    		%%%%%%%%%%%%%%%%%%%%%
    		
    		\node (RealUAVs) at ($(WP7-Box.south) + (2,-1.25)$) [text centered, fill=white, draw, rectangle, minimum width=1.5cm, text width=6em]{ACWs\\autopilot};
    		\node (Humans) at ($(WP7-Box.south) + (-2,-1.25)$) [text centered, fill=white, draw, rectangle, minimum width=1.5cm, text width=6em]{Humans\\Tracker};
    		
    		\draw[-latex] (RealUAVs.north) -- node[right]{Pose, Battery, State} ($(WP7-Box.south) + (2,0)$);
    		\draw[-latex] ($(WP7-Box.south) + (2,0)$) -- (RealUAVs.north);
    		\draw[-latex] (Humans.north) -- node[left]{Pose} ($(WP7-Box.south) + (-2,0)$);
		
	    \end{tikzpicture}}
	\caption{Software architecture: nodes and interfaces. Node diagram from the high-level cognitive task planner perspective}
	\label{fig:NodeDiagram}
\end{figure}

In the software architecture scheme, although some communications are bidirectional, it can be seen that there is a main flow of information. Starting with the information arriving at the node \emph{Gesture Recognition}, this propagates to the last layer, where the \emph{Lower-Level Controllers} use the already processed information to command the \glspl{ACW}. The table \ref{tab:interfaces} shows the type of data that each of the nodes in the figure \ref{fig:NodeDiagram} receives as input and the type of data that each of them emits as output. Additionally, table \ref{tab:shareddata} explains what each one of the data mentioned in the previous table consists of.

% Description of the data interfaces for each software module
\begin{table}[ht]
    \centering
    \caption{Description of the data interfaces for each software module}
    \label{tab:interfaces}
    \small
    \begin{tabular}{|p{0.25\columnwidth}|p{0.25\columnwidth}|p{0.4\columnwidth}|}
      \hline
      \multicolumn{1}{|c}{\textbf{Module Name}} & \multicolumn{1}{|c|}{\textbf{Input Data}} & \multicolumn{1}{c|}{\textbf{Output Data}}\\ \hline \hline
      Gesture Recognition & Images & \textbf{Task, defined by:} Task ID, Task Type, Monitoring Distance, Monitoring Number, WP List, Tool ID (some task parameters will be ignored depending on Task Type) \\ \hline
      
      High-Level Planner & Task, Feedback (Task Result, BatteryEnough, \gls{BT} info), Humans' Pose, \glspl{ACW}' Pose, Battery and State, and Agent Beacon & Task List adding to each one its extra parameters result of the planning (Formation and/or List of \glspl{ACW}' IDs) and Planner Beacon \\\hline
      
      Agent Behaviour Manager & Task List, Low-Level's Result, Human Pose, \glspl{ACW} Pose, Battery and State & Params needed by Low-Level Controllers (depending on Task Type), Feedback (Task Result, BatteryEnough, \gls{BT} info) and Agent Beacon \\ \hline
      
      Low-Level Controllers & Params (depending on Task Type) & Result \\ \hline
      
      Humans Tracker &  & Pose \\ \hline
      
      \gls{ACW} autopilot & Low-Level orders & Pose, Battery and State \\ \hline
      
    \end{tabular}
\end{table}

% Description of data types
\begin{table}[htb]
    \centering
    \caption{Description of data types}
    \label{tab:shareddata}
    \small
    \begin{tabular}{|p{0.2\columnwidth}|p{0.15\columnwidth}|p{0.55\columnwidth}|}
      \hline
      \multicolumn{1}{|c}{\textbf{Data name}} & \multicolumn{1}{|c|}{\textbf{Data type}} & \multicolumn{1}{c|}{\textbf{Comment}} \\ \hline \hline
      
      Task ID & String & Unique identifier of each task \\ \hline
      
      Task Type & Integer & Task type indicator: m/M, i/I or d/D \\ \hline
      
      Human Target ID & String & Unique identifier of each human worker. The position of the human target and other needed info is supposed to be known and accessible via its ID. \\ \hline
      
      Monitoring Distance & Float & Distance from which the \gls{ACW} surveil the worker during a safety monitoring task \\ \hline
      
      Monitoring Number & Integer & Number of \glspl{ACW} that are required in formation for a certain safety monitoring task \\ \hline
      
      WP List & List of $3$ float tuples ($x$, $y$, and $z$) & List of waypoints to be inspected \\ \hline
      
      List of \glspl{ACW}' IDs & List of Strings & List of the unique identifiers of the \glspl{ACW} that have been selected for a task that requires multiple \glspl{ACW} \\ \hline
      
      Formation & Integer & Indicates which of the predefined types of formations should be used for monitoring (e.g., circle, triangle) \\ \hline
      
      Tool ID & String & Unique identifier of the tool to be delivered \\ \hline
      
      \gls{ACW}'s Pose & geometry\_msgs /PoseStamped & \gls{ACW}'s Position and orientation \\ \hline
      
      \gls{ACW}'s Battery & sensors\_msgs /BatteryState & Percentage of battery in the \gls{ACW} \\ \hline

	  Task Result & String, Boolean & First one is the task unique \gls{ID} and second one its result once it's finished \\ \hline
      
      Battery Enough & Boolean & Result of computing if an \gls{ACW} will have enough battery for its current task \\ \hline

	  \gls{BT} info & String list & Status of each \gls{BT}'s node in its last execution (Running, IDLE, SUCCESS or FAILURE) \\ \hline
      
      Agent Beacon & String, String & First one is the \gls{ACW}'s unique ID while the second one defines \gls{ACW}'s type (SafetyACW, InspectACW, or PhysicalACW). It is used as heartbeat and to detect new \glspl{ACW} in Planner \\ \hline

	  Planner Beacon & Time & ROS::Time message containing the time when the beacon was sended. It is used to check the status of the connection from Agent's side. \\ \hline
      
      Lower-Level's Result & Boolean & Result of the Lower-Level Controllers once they have finished after being called \\ \hline
      
    \end{tabular}
\end{table}

The first node is constantly checking the images captured by the \glspl{UAV} for a gesture that is indicating a new task or the modification of an existing task. When this occurs, it asynchronously emits a task, which will be picked up by the centralised planner. As shown in the table \ref{tab:interfaces}, this communication includes the unique \gls{ID} that differentiates this task from the others, the type of task and the parameters that define it.

On the other hand, the \emph{High-Level Planner}, when it receives this information, proceeds to re-evaluate the optimal plan taking into account the task received, the information it receives from the \emph{\glspl{ACW}' autopilot}, and the position of the operators, which is periodically published by the \emph{Human Tracker}. The aforementioned data set constitutes the input data for the \emph{High-Level Planner} together with the feedback coming from each \emph{Agent Behaviour Manager}. Its output data being a list of tasks for each gls{ACW}.

On board each \gls{ACW} is the \emph{Agent Behaviour Manager}. This node is in charge of collecting the corresponding task list provided by the centralised planner. With this input information and the information coming from the \emph{Human Tracker} and the \emph{\gls{ACW}'s autopilot}, this module is in charge of calling the \emph{Lower-Level Controllers} to carry out the execution of the assigned plan. The information emitted by the \emph{\gls{ACW}'s autopilot} is also used to check that everything is working correctly and to execute the security protocols in case they are necessary. If this happens, the corresponding communication would be issued back to the \emph{High-Level Planner} node in order to calculate a new plan. This node also receives the \emph{Lower-Level Controllers}' result after calling each of them, and publish back to the \emph{High-Level Planner} some feedback.

In addition to these communications, the nodes \emph{High-Level Planner} and \emph{Agent Behaviour Manager} periodically exchange beacons that are used to detect both the connection of a new \gls{ACW} and its disconnection in case of failure. Finally, there is an asynchronous communication that is broadcast to all nodes indicating the end of the mission when this happens.

Finally, it is worth mentioning that the \emph{Gesture Recognition} node does not have a communication aimed at modifying parameters of a task already contemplated within the \emph{High-Level Planner}. However, this is possible because tasks have a unique identifier. Once a task has been delivered to the \emph{High-Level Planner}, in order to change any of its parameters, the \emph{Gesture Recognition} node just has to submit the task again, keeping the same task \gls{ID} and updating only the desired parameters. Thus, the \emph{High-Level Planner} would overwrite it and allocate it again with the new parameters.

\section{Centralized module: High-Level Planner}
\label{sec:Centralized module:TaskPlanner}
%% Protocolo de desconexión
%% Protocolo de pérdida de batería
%% Que ocurre cuando una tarea termina
%% Replanificaciones de tareas: restricciones a la hora de planificar o replanificar

As mentioned above, the \emph{High-Level Planner} is a centralised module running on a ground station and constitutes the main cognitive block of the software architecture of which this project is a part. Its purpose is to plan the mission in an optimal way, i.e., to distribute the pending tasks among the available \glspl{ACW} by specifying the order in which they are executed and taking into account the time it takes to complete each one, the type of each \glspl{UAV}, the distance each one will have to travel, the battery they have available, the task each one was executing, the priority of each task, the battery consumed by each task, the recharges that will be needed and when it is best to carry out the recharges.

%% Explicación y Pseudocódigo general de como se inicializa el nodo planner, el bucle while(ros::ok) y como se cierra cuando mission_over. Incluir escucha a imprevistos.
The general pseudocode for this node from launch to termination is contained in the code \ref{ps:GeneralPlanner}.

\begin{lstlisting}[caption={General operation of \emph{High-Level Planner}'s code}, breaklines=true, label=ps:GeneralPlanner]
	1. Read from a ros::param the address of the configuration file.
	2. Read from the configuration file all necessary information.
	3. Configure ROS communications (Publishers, Subscribers and ActionServers).
	4. Set the loop rate.
	5. Main "while" loop. While ros::ok() and not mission over do:
		5.1. Check the timeout of the Agents' beacons.
		5.2. Publish a new Planner beacon.
		5.3. Check for pending incoming communications (ros::spinOnce).
		5.4. Sleep the remaining time to send the next beacon.
	6. Wait until all UAVs have finished and disconnected. While there is any agent connected do:
		6.1. Check the timeout of the Agents' beacons.
		6.2. Check for pending incoming communications (ros::spinOnce).
		6.3. Sleep for a while.
\end{lstlisting}

%% Decir que todo funciona con callbacks y explicarlos. (batteryEnoughCB, taskResultCB, positionCallback, batteryCallback)(incomingTask, beaconCallback, missionOverCallback)
Since the environment in which the \glspl{UAV} operates is dynamic, this module has been programmed in such a way that it can react to unforeseen events and recalculate the optimal plan. As can be deduced from the \ref{ps:GeneralPlanner} pseudocode, everything works through callback functions. Every time a communication arrives from another node, a response is triggered on that node. The information contained in the message is analyzed and it is decided whether a replanning is necessary or not. The situations in which a replanning has been deemed necessary are listed in section \ref{sec:TaskReplanningSituations}. The communications summarized in the tables \ref{tab:interfaces} and \ref{tab:shareddata} and in the figure \ref{fig:NodeDiagram} are sufficient to detect these unforeseen events and to be able to respond to them in the best possible way.

\begin{lstlisting}[caption={Task callback pseudocode}, breaklines=true, label=ps:IncomingTask]
	1. Check if the task already exists and delete it in order to create it with the new parameters.
	2. Read the type of task and the parameters that apply to it.
	3. Add the new task to the pending task list.
	4. Perform a task planning.
\end{lstlisting}

Firstly, there is the callback that is executed when the node \emph{Gesture Recognition} sends a task, which always ends up calling the function in charge of calculating the optimal plan (see code \ref{ps:IncomingTask}); the mission over callback, whose only action is to change the value of a variable so that the node exits the main while loop; and finally the agent's beacon callback, which is executed every time a \gls{UAV} beacon is received and whose pseudocode is the code \ref{ps:AgentBeaconCallback}.

\begin{lstlisting}[caption={Agent's beacon callback}, breaklines=true, label=ps:AgentBeaconCallback]
	1. Read the information contained in the beacon.
	2. If it is a connection of a new UAV:
		2.1. Register it in the database.
		2.2. Perform a task planning.
	3. Else, if it is the heartbeat of an already known UAV:
		3.1. Reset the timeout timer.
\end{lstlisting}

The action carried out by the agent's beacon callback varies depending on whether it is the beacon of a new \gls{UAV} or the heartbeat of a known \gls{UAV}. For each agent there will be an object in the database that will contain another series of callbacks that will be in charge of receiving the messages coming from the \glspl{ACW} and respond accordingly.

\begin{lstlisting}[caption={Callback that runs when an \emph{Agent Behaviour Manager} sends battery feedback}, breaklines=true, label=ps:batteryEnoughCB]
	1. Update the value of the internal flag associated with the battery.
	2. Perform a task planning.
\end{lstlisting}

The \emph{Agent Behaviour Manager} node only sends communications messages indicating the battery status when it is due to an unplanned event. This event can be either an early battery depletion or a faster than expected recharge. In both cases, the callback function whose pseudocode is the code \ref{ps:batteryEnoughCB} consequently updates the value of an internal variable used during planning, and recalculates the optimal plan.

The other possible communication coming from a node of type \emph{Agent Behaviour Manager} with the ability to trigger a reaction in the planner is due to the termination of a task. When a task finishes successfully, it is simply removed from the list of pending tasks. In addition, this moment is used to re-evaluate the optimal plan. It is expected that the mission is still within the optimal plan, so in that case the planning result should be the same as the plan that was already being executed. If, on the other hand, conditions have changed since the last planning and a better plan now exists, it is at this point that the plan is updated. On the other hand, if the task ends with a failure, the callback action will depend on the causes of the failure (note that the interruption of a task will result in a failure). If the interruption is due to the \gls{UAV} battery, it may be planned, in which case no action is required, or it may be unexpected, in which case the corresponding actions are taken by the battery callback. Once it has been verified that the task has not finished due to the battery, a check is made to see if the task was at the beginning of the queue. If so, a failure has indeed occurred, so the operators are warned, the task is removed from the list and a replanning is executed. Otherwise the task in question would have been moved from the top of the queue due to a change of plans and therefore no action would have to be taken either. The pseudocode corresponding to what has just been explained is in code \ref{ps:taskResultCB}.

\begin{lstlisting}[caption={Callback that runs when an \emph{Agent Behaviour Manager} sends a task result}, breaklines=true, label=ps:taskResultCB]
	1. Read the information contained in the task result.
	2. If the task result is SUCCESS:
		2.1. Delete it from the pending tasks list.
		2.2. Perform a task planning.
	3. Else, if the task result is FAILURE:
		3.1. If the task has been halted because of not having battery enough:
			3.1.1. Return.
		3.2. Else, if the task is on the front of that ACW's task queue:
			3.2.1. Notify operators that a task has failed and is going to be deleted.
			3.2.2. Delete task from the pending tasks list.
			3.2.3. Perform a task planning.
		3.3. Else:
			3.3.1. Return.
\end{lstlisting}

The other two communications received by the \emph{High-Level Planner} from the \glspl{ACW} are sensor readings corresponding to the \glspl{UAV}' position and battery percentage. In both cases the only action of the corresponding callback is to update the information with the new values.

The last function that remains to be explained of those that can potentially request a replanning of the mission is the one in charge of checking the timeout of the agents' beacons. As shown in the code \ref{ps:GeneralPlanner}, this function is not a callback like the previous ones, instead it is executed periodically in the main while loop. Its operation is shown in the code \ref{ps:checkBeaconsTimeout}. Simply, for each agent connected, it checks that the timeout amount of time has not elapsed since its last beacon was received. If a timeout has occurred, that \gls{ACW} is considered disconnected and is removed from the centralised node data. If, after checking all the agents, the number of connected \glspl{UAV} has decreased, i.e. if any of the previously connected \glspl{UAV} has disconnected, a mission replanning is executed.

% Pseudocódigo de checkBeaconsTimeout
\begin{lstlisting}[caption={Beacons' timeout check function}, breaklines=true, label=ps:checkBeaconsTimeout]
	1. For each agent connected:
		1.1. If the elapsed time since the last beacon is grater than the timeout time:
			1.1.1. Add that agent's ID to the list of disconnected agents.
	2. While the list of disconnected agents is not empty:
		2.1. Take first ID from the list.
		2.2. Erase from the node's data all information related to that ID.
	3. If any agent has been disconnected:
		3.1. Perform a task planning.
\end{lstlisting}

%% Explicar como se realiza la planificación y poner psudocódigo de cómo se lleva a cabo. Explicar también como se calcula el coste.
The pseudocode that is executed when one of these functions deems it necessary to perform a new task planning is summarized in the code \ref{ps:performTaskAllocation}. It is important to remember that some tasks have a higher priority than others, and this depends only on the type of task. To simplify the process, it has been decided to allocate the tasks in order of arrival, assuming that between two tasks of the same type, the one that arrived first will be more urgent, and therefore will be given higher priority. Therefore, when a new task is received, it is stored both in the \emph{std::map} that contains all the pending tasks to facilitate the access to the information, and in the \emph{std::vector} of its task type, where the order of arrival is maintained. What this simplification allows is to assign tasks one at a time. By having a prioritized list of tasks and assuming that no task can be assigned before a task with a higher priority, the mission planning problem is reduced to calculating the cost of each task individually for each \gls{UAV} with the ability to execute it and assign it to the one with the lowest cost. For monitoring-type tasks, the selection of the required number of agents is strictly cost-based. The \emph{N} agents that cost the least to execute the task are selected. The same is a little more complex for the tasks of type inspect, where the number of agents to select is a parameter to be defined by the planner itself. This value is first set according to the number of points to be inspected. Up to three points, a single \gls{ACW} is selected; up to six points, two are selected; and from seven points onwards, three agents are selected, this being the maximum number imposed by the low-level controller. Moreover, as the low-level controller in charge of this task works, all the \glspl{ACW} selected for this task are required to start executing it simultaneously, so a second approximation of this number is made according to the number of idle \glspl{UAV}. Thus, if they are assigned this as the first task, they will start executing it simultaneously. Academically, this simplification seems to deviate from the optimal solution, but it must be remembered that this work is part of a software architecture that will operate in real situations. In such situations, it is not expected that there will be a large number of \glspl{UAV} connected simultaneously, nor a long list of pending tasks. In such a simple scenario, it makes sense to make this simplification without deviating too much from the optimal solution. Finally, the number of agents to be selected will be the smaller of the two above, being equal to one when there is no \gls{UAV} idle and zero in case there is no \gls{ACW} with enough battery. In the latter case, the task would be assigned after recharging. Once the number of agents to be selected has been defined, the agents that have the least cost to execute the task are selected from among those that meet the conditions described. Having selected the \gls{ACW} that will carry out the task, all that remains is to distribute among them the \glspl{WP} to be inspected. Although the algorithm in charge of performing the optimal distribution is in the low-level controller of this task, as the rest of the modules that make up the software architecture are not yet available, it has been necessary to program a distribution algorithm in order to be able to carry out the experiments. More details on this will be given in section \ref{sec:LowerAndUpperLevelModulesFaker}.

% Explicar como se calcula el coste.
The cost for each \gls{UAV} is calculated as the weighted sum of three different types of costs. A first cost assesses the type of \gls{ACW} and penalizes the assignment of tasks to those \glspl{UAV} designed for another type. It penalizes especially the assignment of lower priority tasks to agents designed to perform higher priority tasks. The second cost evaluates the total distance the \gls{UAV} will have to travel from where it is at the beginning of the task to where it is at the end of the task. This cost is an approximation of the expected battery consumption, although it does not take into account intermediate travel and hoovering times during the mission. The last cost penalizes the interruption of the task that was being executed according to the previous plan and rewards the assignment of the same task. This cost is intended to ensure that a task is preferentially assigned to an idle \gls{UAV}, to an \gls{UAV} that is executing a lower priority task, or even to an \gls{UAV} of a different type, rather than interrupting a task unnecessarily just because that \gls{ACW} has to travel a shorter distance, for example.

\begin{lstlisting}[caption={Simplified task planning function's pseudocode}, breaklines=true, label=ps:performTaskAllocation]
	1. If there is any agent connected:
		1.1. For each agent connected:
			1.1.1. Make a copy of the current task queue.
			1.1.2. Empty the task queue.
		1.2. For each Tool Delivery task:
			1.2.1. Compute the cost of the task for each PhysicalACW that has battery enough.
			1.2.2. Assign the task to the agent for who the task cost the least (from those who has battery enough).
			1.2.3. Add the task to that agent's task queue.
		1.3. For each Inspection task:
			1.3.1. Extract from the task parameters the list of WP to inspect.
			1.3.2. For each ACW(any type) tha has battery enough:
				1.3.2.1. Compute the cost of the task for that ACW. 
				1.3.2.2. Check if that ACW is still idle.
			1.3.3. Calculate the number of agents to select for the task based on the number of WP and the number of idle agents.
			1.3.4. If no agent has battery enough, continue.
			1.3.5. Else, if the number of agents to select is equal to zero, assign the task to the agent that cost the least.
			1.3.6. Else, select the calculated number of agents for whom the task costs the least.
			1.3.7. Divide the WP to inspect among the selected agents.
			1.3.8. For each selected agent:
				1.3.8.1. Set the remaining task parameters (List of selected ACWs' IDs and divided WP list).
				1.3.8.2. Add the task to the agent's task queue.
		1.4. For each Monitoring task:
			1.4.1. Compute the cost of the task for each ACW (any type) that has battery enough.
			1.4.2. If required number of ACWs for the task is zero:
				1.4.2.1. Warn operators that this parameter can not be zero.
				1.4.2.2. Delete task from pending tasks.
			1.4.3. Else, select the requested number of agents for whom the task costs the least.
			1.4.4. Set the remaining task parameter (List of selected ACWs' IDs)
			1.4.4. Add the task to each selected agent's task queue.
		1.5. For each ACW connected, send the new task queue to its Agent Behavior Manager.
	2. Else:
		2.1. Warn operators that any agent is connected.
\end{lstlisting}

Once the calculation of the mission plan has been completed, the new task queues are sent to the corresponding distributed modules. Each \emph{Agent Behaviour Manager} will react to this communication and will take care of executing the newly assigned plan. In the meantime, the \emph{High-Level Planner} node returns to the main while loop to continue waiting until an event that triggers a replanning occurs again.

\section{Distributed module: Agent Behavior Manager}
\label{sec:Distributed module: behavior manager}
%% Explicar qué es una drone behavior manager y cual es su función.
This node is in charge of executing the plan assigned by the \emph{High-Level Planner}, of checking the security of the equipment at all times, of detecting unforeseen events and of communicating them to the centralised node so that it can make a change of plans if it deems it necessary. The \emph{Agent Behavior Manager} will communicate with the low-level controllers, handing over control when necessary to complete the assigned plan.

%% Estructura general del nodo (pseudocodigo). Explicar que se ha hecho con árboles de comportamiento en paralelo a algunos procesos de ros
The general structure of this node is quite similar to that of the central node. The pseudocode is summarized in the code \ref{ps:GeneralAgent}. Upon initialization, the \emph{Agent Behaviour Manager} prepares the necessary information to start its operation, configures the necessary communications, declares and initializes the behaviour tree and, once the \gls{UAV} it controls has finished initializing, starts sending beacons to the central node to inform it of its joining the mission. Once the code finishes initializing and reaches the main while loop, the activity of the \emph{Agent Behaviour Manager} concentrates on the execution of callbacks in response to incoming messages, as in the \emph{High-Level Planner}, and on the execution of the behaviour tree, which directs and supervises the \gls{UAV} movement.

\begin{lstlisting}[caption={General operation of \emph{Agent Behaviour Manager}'s code}, breaklines=true, label=ps:GeneralAgent]
	1. Read from a ros::param the beacon's content (ACW's ID and type).
	2. Read from a ros::param the address of the configuration file.
	3. Read from the configuration file all necessary information.
	4. Configure ROS communications (Publishers, Subscribers and ActionServers).
	5. Set the loop rate.
	6. Declare the behaviour tree.
	7. Initialize each BT node.
	8. Start BT loggers to to facilitate debugging and monitoring of the node's performance.
	9. Wait until the ACW fully initializes.
	10. Main "while" loop. While ros::ok() and BT status is running:
		10.1. If a timeout of Planner's beacons has not ocurred:
			10.1.1. Publish a new Agent beacon.
		10.2. Check if battery is enough for the current task.
		10.3. Check for pending incoming communications (ros::spinOnce).
		10.4. Sleep the remaining time to send the next beacon.
\end{lstlisting}

%% Explicar lo que es un árbol de comportamiento y compararlo con una máquina de estados.
The \glspl{BT} are who governs the \glspl{ACW} to perform each of the assigned tasks. Each \gls{BT} monitors its \gls{ACW}'s battery and task status and reacts to any possible failure or unexpected event, requesting a new re-planning to the \emph{High-Level Planner} in case of need. BT can be defined as an improved \gls{FSM}. They are a more advanced mechanism to implement behaviors, especially because of their advantages in terms of scalability, modularity, readability and reusability, facilitating the creation of more complex behaviours with less effort.

%% Informe de actividades: Diseño del árbol de comportamiento.
Despite this, the process of designing a state machine is nothing like the process of designing a behaviour tree. Designing behaviour trees without ever having done it before is not a trivial task. Moreover, there will be more than one valid implementation to achieve the same behaviour, which makes it more complicated to design this type of solution when you do not yet have enough intuition to know which one is better. Taking advantage of the fact that the use of \gls{BT} is widespread in the videogame industry, information about them was gathered and studied to try to develop enough knowledge and intuition to design from scratch a \gls{BT} that meets the needs of the mission. The examples found in \cite{BT-CPP-doc, colledanchise2018behavior, BT-AI} were very useful.

Before proceeding with the explanation of the designed BT, the types of nodes that can be found in the selected C++ library and the functioning of each of them will be briefly discussed.

%% Informe de actividades: Tipos de nodos de BT
\begin{figure}[htbp]
    \centering
    \subfloat[]{%Fallback
		\label{subfig:Fallback}
        \begin{tikzpicture}
			\node (MainTree) at (0,0) [text centered, fill=white, draw, rectangle, minimum width=0.5cm, text width=0.5em]{\textbf{?}};
		\end{tikzpicture}}
    \hfill
    \subfloat[]{%Sequence
		\label{subfig:Sequence}
        \begin{tikzpicture}
			\node (MainTree) at (0,0) [text centered, fill=white, draw, rectangle, minimum width=1.5cm, text width=1.5em]{$\longrightarrow$};
		\end{tikzpicture}}
	\hfill
    \subfloat[]{%Reactive
		\label{subfig:Reactive}
        \begin{tikzpicture}
			\node (MainTree) at (0,0) [text centered, fill=magenta!5, draw=magenta, rectangle, minimum width=2cm, minimum height=0.75cm, text width=2em]{};
		\end{tikzpicture}}
    \hfill
    \subfloat[]{%Common
		\label{subfig:Common}
        \begin{tikzpicture}
			\node (MainTree) at (0,0) [text centered, fill=white, draw, rectangle, minimum width=2cm, minimum height=0.75cm, text width=2em]{};
		\end{tikzpicture}}
    \hfill
    \subfloat[]{%Decorator
		\label{subfig:Decorator}
        \begin{tikzpicture}
			\node (MainTree) at (0,0) [text centered, fill=orange!5, draw=orange, rectangle, minimum width=2cm, minimum height=0.75cm, text width=2em]{};
		\end{tikzpicture}}
    \hfill
    \subfloat[]{%Action
		\label{subfig:Action}
		\begin{tikzpicture}
			\node (MainTree) at (0,0) [text centered, fill=blue!5, draw=blue, rectangle, minimum width=2cm, minimum height=0.75cm, text width=2em]{};
		\end{tikzpicture}}
		\hfill
    \subfloat[]{%Condition
		\label{subfig:Condition}
        \begin{tikzpicture}
			\node (MainTree) at (0,0) [text centered, fill=blue!5, draw=blue, ellipse, minimum width=2cm, minimum height=0.75cm, text width=2em]{};
		\end{tikzpicture}}
    \caption{Different types of nodes that can be present in an \gls{BT}}
    \label{fig:CounterRunImagenes}
\end{figure}

%% Del WP7_Scenarios: Funcionamientos de los BT (va con la imagen anterior)
 Behaviour Trees are made up of \emph{Control} nodes, \emph{Decorator} nodes, and \emph{Leaf} nodes. \emph{Control} nodes could be either \emph{Fallback} nodes, represented with a question mark (see subfigure \ref{subfig:Fallback}), which try success calling one by one each of their children; or \emph{Sequence} nodes, represented with an arrow (see subfigure \ref{subfig:Sequence}), which call their children in order if the previous one have succeded. \emph{Fallback} nodes return \emph{SUCCESS} if one of its children does it, \emph{FAILURE} if none of them return \emph{SUCCESS}, and \emph{RUNNING} if one of its children returns \emph{RUNNING}. On the other hand, \emph{Sequence} nodes return \emph{SUCCESS} when all children have been called in order and have returned \emph{SUCCESS}. If any of them returns \emph{FAILURE}, the sequence is broken and the \emph{Sequence} node returns \emph{FAILURE} too. When a child returns \emph{RUNNING}, \emph{Sequence} node does it too. \emph{Control} nodes are represented in a black rectangular box when they are the standard ones (see subfigure \ref{subfig:Common}), but they could also be \emph{Reactive} control nodes, represented by a magenta box (see subfigure \ref{subfig:Reactive}), which means that its already called children will be called again in the next iteration. This is very useful for generating behaviours where an action is constantly reattempted, or where it is necessary to check that the necessary conditions are still met. A \emph{Child} node could be another \emph{Control} node, a \emph{Decorator} node, a \emph{Leaf} node or a whole sub-tree. A \emph{Decorator} node, represented in an orange box (see subfigure \ref{subfig:Decorator}), can only have one child (of any type) and its function is programmable (e.g., modifying its child result or retrying calling its child a number of times). \emph{Leaf} nodes, represented in blue, could be \emph{Condition} nodes, represented in a blue elliptical shaped box (see subfigure \ref{subfig:Condition}), that check a condition and return either \emph{SUCCESS} or \emph{FAILURE}; or \emph{Action} nodes, represented in a blue rectangular box (see subfigure \ref{subfig:Action}), that execute code that take longer to execute and therefore these nodes could also return \emph{RUNNING}.

\subsection{Main tree}
\label{sec:MainTree}
%%% Recalcar que se ha hecho de forma que toda la inteligencia y las decisiones estén y se tomen en el planner
%%% Hablar aqui dentro de como se gestionan las desconexiones, la batería y las replanificaciones
%% Protocolo de desconexión
%% Protocolo de pérdida de batería
%% Que ocurre cuando una tarea termina

%% Del WP7_Scenarios: main tree
\begin{figure}[ht]
	\begin{center}
		\scalebox{0.9}{
			\begin{tikzpicture}
        		\node (MainTree) at (0,0) [text centered, fill=white, draw, rectangle, minimum width=1.5cm, text width=5.5em]{Main Tree};

        		\node (RootFallback) at ($(MainTree) + (0,-1)$) [text centered, fill=magenta!5, draw=magenta, rectangle, minimum width=0.5cm, text width=0.5em]{\textbf{?}};
        		\draw[-latex] (MainTree.south) -- (RootFallback.north);

        		\node (MissionOverSequence) at ($(RootFallback) + (-3,-1.5)$) [text centered, fill=white, draw, rectangle, minimum width=1.5cm, text width=1.5em]{$\longrightarrow$};
        		\draw[-latex] (RootFallback.south) -- (MissionOverSequence.north);
        		\node (ForceRunning) at ($(RootFallback) + (3, -1.5)$) [text centered, fill=orange!5, draw=orange, rectangle, minimum width=1.5cm, text width=5.5em]{Force Running};
        		\draw[-latex] (RootFallback.south) -- (ForceRunning.north);
        		
        		\node (MissionOver) at ($(MissionOverSequence) + (-1.75,-1.5)$) [text centered, fill=blue!5, draw=blue, ellipse, minimum width=1.5cm, text width=5.5em]{Mission Over?};
        		\draw[-latex] (MissionOverSequence.south) -- (MissionOver.north);
        		\node (BackToStation) at ($(MissionOverSequence) + (1.75, -1.5)$) [text centered, fill=blue!5, draw=blue, rectangle, minimum width=1.5cm, text width=5.5em]{Back To Station};
        		\draw[-latex] (MissionOverSequence.south) -- (BackToStation.north);
        		\node (MissionFallback) at ($(ForceRunning) + (0,-1.5)$) [text centered, fill=magenta!5, draw=magenta, rectangle, minimum width=0.5cm, text width=0.5em]{\textbf{?}};
        		\draw[-latex] (ForceRunning.south) -- (MissionFallback.north);

        		\node (IdleSequence) at ($(MissionFallback) + (-4.25,-1.25)$) [text centered, fill=magenta!5, draw=magenta, rectangle, minimum width=1.5cm, text width=1.5em]{$\longrightarrow$};
        		\draw[-latex] (MissionFallback.south) -- (IdleSequence.north);
        		\node (WaitFallback) at ($(MissionFallback) + (4.25,-1.25)$) [text centered, fill=magenta!5, draw=magenta, rectangle, minimum width=0.5cm, text width=0.5em]{\textbf{?}};
        		\draw[-latex] (MissionFallback.south) -- (WaitFallback.north);
        		
        		\node (Inverter) at ($(IdleSequence) + (-3, -1.5)$) [text centered, fill=orange!5, draw=orange, rectangle, minimum width=1.5cm, text width=5.5em]{Inverter};
        		\draw[-latex] (IdleSequence.south) -- (Inverter.north);
        		\node (TaskSequence) at ($(IdleSequence) + (2.5,-1.5)$) [text centered, fill=magenta!5, draw=magenta, rectangle, minimum width=1.5cm, text width=1.5em]{$\longrightarrow$};
        		\draw[-latex] (IdleSequence.south) -- (TaskSequence.north);
        		\node (IsBatteryFull) at ($(WaitFallback) + (-1.75,-1.5)$) [text centered, fill=blue!5, draw=blue, ellipse, minimum width=1.5cm, text width=5.5em]{Is Battery Full?};
        		\draw[-latex] (WaitFallback.south) -- (IsBatteryFull.north);
        		\node (Recharge) at ($(WaitFallback) + (1.75, -1.5)$) [text centered, fill=blue!5, draw=blue, rectangle, minimum width=1.5cm, text width=6.5em]{Recharge};
        		\draw[-latex] (WaitFallback.south) -- (Recharge.north);

        		\node (Idle) at ($(Inverter) + (0,-1.5)$) [text centered, fill=blue!5, draw=blue, ellipse, minimum width=1.5cm, minimum height=1.35cm, text width=5.5em]{Idle?};
        		\draw[-latex] (Inverter.south) -- (Idle.north);
        		\node (IsBatteryEnough) at ($(TaskSequence) + (-1.75,-1.5)$) [text centered, fill=blue!5, draw=blue, ellipse, minimum width=1.5cm, text width=5.5em]{Is Battery Enough?};
        		\draw[-latex] (TaskSequence.south) -- (IsBatteryEnough.north);
        		\node (PerformTaskTree) at ($(TaskSequence) + (1.75, -1.5)$) [text centered, fill=white, draw, rectangle, minimum width=1.5cm, text width=6.5em]{Perform Task Tree};
        		\draw[-latex] (TaskSequence.south) -- (PerformTaskTree.north);
        		
        		%\draw[-latex] (.south) -- (.north);
		    \end{tikzpicture}}
		\caption{Behavior Tree: Main tree}
		\label{fig:MainTree}
	\end{center}
	\vspace{-1em}
\end{figure}
%% Informe de actividades: Main tree
% Esta es la base del árbol de comportamiento implementado. Primeramente se comprueba que la misión no haya terminado, de forma que el BT siga ejecutándose hasta que esta finalice. En caso contrario se ejecuta el resto del árbol. En caso de que el UAV sobre el cual se ejecuta el BT tenga asignada alguna tarea, se comprueba si se tiene batería suficiente y en caso afirmativo se ejecuta el sub-árbol para las tareas. Si el UAV no tiene ninguna tarea asignada, se comprueba el nivel de batería y se recarga. Esta estructura está diseñada de forma que, si el UAV se desconecta, o si el nodo “isBatteryEnough” devuelve “FAILURE”, se vacía la cola de tareas asignada como plan al UAV en cuestión y este, al detectar en la siguiente iteración que se encuentra ocioso, “Idle”, se dirigirá de forma segura a recargar. En ambos casos, será ejecutada una re-planificación de tareas.

%% Del WP7_Scenarios: main tree
Each Agent Behavior Manager implements several interconnected \gls{BT}. The \emph{Main tree} is depicted in Fig.~\ref{fig:MainTree}. This \gls{BT} checks whether the mission is over (a mission would represent the working session, not a single task, i.e., whether the \gls{ACW} is ready to be turned off) and otherwise, whether the \gls{ACW} has any task to perform. If so, the battery level is checked and, depending on the result, either the corresponding task is executed (sub-tree represented in Fig.~\ref{fig:PerformTasksTree}) or the \gls{ACW} is guided to a recharging station\footnote{Both Safety, Inspection and Physical-ACW provide an input interface to guide the \gls{ACW} to the charging station. In other words, among the low-level controller capabilities, there is a "reach this point". The location of the charging stations is known in advance or provided as input by the High-Level Planner/Behavior Tree.}. The \gls{BT} is also prepared to be safe against a loss of connection with the centralized module. Both unexpected events are managed flushing the task queue for the \gls{ACW} to recharging, while giving the High-Level Planner control to decide when it is the best time to stop recharging (the High-Level Planner just needs to assign tasks again so that the \gls{ACW} start working back).

%% Del WP7_Scenarios: perform task tree
\begin{figure}[ht]
	\begin{center}
		\scalebox{0.75}{
			\begin{tikzpicture}
			    \node (PerformTaskTree) at (0,0) [text centered, fill=white, draw, rectangle, minimum width=1.5cm, text width=5.5em]{Perform Task Tree};
        		
        		\node (TaskFallback) at ($(PerformTaskTree) + (0,-1.5)$) [text centered, fill=magenta!5, draw=magenta, rectangle, minimum width=0.5cm, text width=1.5em]{\textbf{?}};
        		\draw[-latex] (PerformTaskTree.south) -- (TaskFallback.north);
        		
        		\node (MonitorSequence) at ($(TaskFallback) + (-7,-1.5)$) [text centered, fill=magenta!5, draw=magenta, rectangle, minimum width=1.5cm, text width=1.5em]{$\longrightarrow$};
        		\draw[-latex] (TaskFallback.south) -- (MonitorSequence.north);
        		\node (InspectSequence) at ($(TaskFallback) + (0,-1.5)$) [text centered, fill=magenta!5, draw=magenta, rectangle, minimum width=1.5cm, text width=1.5em]{$\longrightarrow$};
        		\draw[-latex] (TaskFallback.south) -- (InspectSequence.north);
        		\node (DeliverSequence) at ($(TaskFallback) + (7,-1.5)$) [text centered, fill=magenta!5, draw=magenta, rectangle, minimum width=1.5cm, text width=1.5em]{$\longrightarrow$};
        		\draw[-latex] (TaskFallback.south) -- (DeliverSequence.north);
        		
        		\node (IsTaskMonitor) at ($(MonitorSequence) + (-1.75,-2)$) [text centered, fill=blue!5, draw=blue, ellipse, minimum width=1.5cm, text width=6em]{Is Task "Monitoring"?};
        		\draw[-latex] (MonitorSequence.south) -- (IsTaskMonitor.north);
        		\node (MonitorTree) at ($(MonitorSequence) + (1.75, -2)$) [text centered, fill=white, draw, rectangle, minimum width=1.5cm, text width=6.5em]{Monitoring Task Tree};
        		\draw[-latex] (MonitorSequence.south) -- (MonitorTree.north);
        		\node (IsTaskInspect) at ($(InspectSequence) + (-1.75,-2)$) [text centered, fill=blue!5, draw=blue, ellipse, minimum width=1.5cm, text width=5.5em]{Is Task "Inspection"?};
        		\draw[-latex] (InspectSequence.south) -- (IsTaskInspect.north);
        		\node (InspectTree) at ($(InspectSequence) + (1.75, -2)$) [text centered, fill=white, draw, rectangle, minimum width=1.5cm, text width=5.5em]{Inspection Task Tree};
        		\draw[-latex] (InspectSequence.south) -- (InspectTree.north);
        		\node (IsTaskDeliver) at ($(DeliverSequence) + (-1.75,-2)$) [text centered, fill=blue!5, draw=blue, ellipse, minimum width=1.5cm, text width=5.5em]{Is Task "Tool Delivery"?};
        		\draw[-latex] (DeliverSequence.south) -- (IsTaskDeliver.north);
        		\node (DeliverTree) at ($(DeliverSequence) + (1.5, -2)$) [text centered, fill=white, draw, rectangle, minimum width=1.5cm, text width=6.5em]{Tool Delivery Task Tree};
        		\draw[-latex] (DeliverSequence.south) -- (DeliverTree.north);
        		
		    \end{tikzpicture}}
		\caption{Behavior Tree: Perform Task Tree}
		\label{fig:PerformTasksTree}
	\end{center}
\end{figure}
%% Informe de actividades: perform task tree
% Este es el sub-árbol para comprobar qué tarea es la que se debe ejecutar y llamar consecuentemente a uno u otro sub-árbol. En este punto se podrían llamar directamente a los módulos del nivel inferior, pero se decidió que el control no se le pasaría a los módulos de niveles inferiores hasta que el UAV se encuentre cerca de la zona donde haya de realizar su tarea.

%% Del WP7_Scenarios: subtrees
Figures \ref{fig:MonitorTree}, \ref{fig:InspectTree} and \ref{fig:DeliverToolTree} represent the sub-trees that run Safety, Inspection, and Physical tasks, respectively. They all guide the \gls{ACW} close to where the tasks needs to be performed (e.g., close to a worker to monitor or a place to inspect) and then, the corresponding Lower-Level Controller is called. These Lower-Level Controllers run on board the corresponding \gls{ACW}s and must communicate their results (success or failure) asynchronously back to the Agent Behavior Manager, so that the Behaviour Tree could continue running.

%% Del WP7. Parrafo de unión con la descripción de las tareas. Habrá que reescribirlo porque no me gusta pero vale de inspiración.
In the following, a brief description on how to carry out each of the available tasks in the system. It is assumed that there are Low-Level Controllers running on the \glspl{ACW} to perform basic navigation actions, formation control for human worker monitoring, inspection operations, and physical interaction with the human worker (e.g., to pick or deliver a tool). These Low-Level Controllers operate in a known environment, represented by a map (i.e., an occupancy grid map) that also includes the position of obstacles and the power tower.

%% Informe de actividades: seguridad, emergencias y concentración de la inteligencia
% Destacar que este árbol de comportamiento no se encarga de realizar el reparto y la planificación de tareas a largo plazo como tal, sino que este se ejecuta sobre cada UAV y se encarga de llevar a cabo la ejecución del plan que se le ha asignado. Este BT es además el responsable de detectar, comunicar y actuar ante cualquier situación inesperada como un pronto agotamiento de la batería o una desconexión. A su vez, este submódulo se encarga de comunicar al planificador todos los detalles y los eventos que vayan sucediendo para que este decida si es necesario re-planificar toda la misión o no. El objetivo de este BT es conseguir que en los UAVs no haya toma de decisiones como tal, sino que toda la inteligencia se encuentre centralizada y que todos los comportamientos y decisiones que un UAV pueda llegar a tomar estén contenidas y precalculadas dentro de la estructura del propio BT, que está diseñado de forma que asegure el buen funcionamiento y la seguridad del equipo ante cualquier circunstancia.

% Information exchanges 
% Information interfaces/channels
% Takeovers
\subsection{Inspection task tree}
\label{sec:InspectionTaskTree}
% Del WP7_Scenarios
\textbf{High-Level Planner inputs}: Task ID, Task Type and Waypoint List (the rest will be ignored).
\textbf{Description}: the High-Level Planner, from the list of waypoints, will decide how many \glspl{ACW} are required and which part of the waypoint list is assigned to each one. Then, it would send to each corresponding Agent Behavior Manager the task parameters, including a list with the IDs of the selected~ glspl{ACW}. The same information is forwarded to the Lower-Level Controllers when the BT calls them (see Fig.~\ref{fig:InspectTree}). Basically, the list of waypoints to be inspected are covered by the assigned \glspl{ACW}, stopping at each of them to take images.  

\begin{figure}[ht]
	\begin{center}
		\scalebox{1}{
			\begin{tikzpicture}
			    \node (InspectTree) at (0,0) [text centered, fill=white, draw, rectangle, minimum width=1.5cm, text width=5.5em]{Inspection Task Tree};
        		
        		\node (InspectTaskSequence) at ($(InspectTree) + (0,-1.5)$) [text centered, fill=magenta!5, draw=magenta, rectangle, minimum width=1.5cm, text width=1.5em]{$\longrightarrow$};
        		\draw[-latex] (InspectTree.south) -- (InspectTaskSequence.north);
        		
        		\node (NearWPFallback) at ($(InspectTaskSequence) + (-1.5,-1.5)$) [text centered, fill=magenta!5, draw=magenta, rectangle, minimum width=0.5cm, text width=0.5em]{\textbf{?}};
        		\draw[-latex] (InspectTaskSequence.south) -- (NearWPFallback.north);
        		\node (Inspect) at ($(InspectTaskSequence) + (1.5,-1.5)$) [text centered, fill=blue!5, draw=blue, rectangle, minimum width=1.5cm, text width=5.5em]{Inspect};
        		\draw[-latex] (InspectTaskSequence.south) -- (Inspect.north);
        		
        		\node (IsUAVnearWP) at ($(NearWPFallback) + (-1.75,-2)$) [text centered, fill=blue!5, draw=blue, ellipse, minimum width=1.5cm, text width=5.5em]{Is \gls{ACW} near WP?};
        		\draw[-latex] (NearWPFallback.south) -- (IsUAVnearWP.north);
        		\node (GoNearWP) at ($(NearWPFallback) + (1.75, -2)$) [text centered, fill=blue!5, draw=blue, rectangle, minimum width=1.5cm, text width=6.5em]{Go near WP};
        		\draw[-latex] (NearWPFallback.south) -- (GoNearWP.north);
		    \end{tikzpicture}
		}
		\caption{Behavior Tree: sub-tree that controls the inspect tasks}
		\label{fig:InspectTree}
	\end{center}
\end{figure}

\subsection{Monitoring task tree}
\label{sec:MonitoringTaskTree}
% Del WP7_Scenarios
\textbf{High-Level Planner inputs}: Task ID, Task Type, Human Target ID, Monitoring Distance, and Monitoring Number (the rest will be ignored).
\textbf{Description}: the High-Level Planner will assign this task to as many Safety-ACWs as specified Monitoring Number. The formation will be chosen by the High-Level Planner from a set of fixed formations (to be listed) depending on the number of \gls{ACW}. Each selected \gls{ACW} will know a list with the IDs of the \glspl{ACW} selected for the task and the formation that they must take. As shown in Fig. \ref{fig:MonitorTree}, each Agent Behaviour Manager would individually navigate each \gls{ACW} near the human target and then, it would call the corresponding Lower-Level Controller for formation control. Extra \glspl{ACW} could even be added to the formation at any time, just updating the task parameters sending a new task from Gesture Recognition.

\begin{figure}[ht]
	\begin{center}
		\scalebox{1}{
			\begin{tikzpicture}
			    \node (MonitorTree) at (0,0) [text centered, fill=white, draw, rectangle, minimum width=1.5cm, text width=5.5em]{Monitoring Task Tree};
        		
        		\node (MonitorTaskSequence) at ($(MonitorTree) + (0,-1.5)$) [text centered, fill=magenta!5, draw=magenta, rectangle, minimum width=1.5cm, text width=1.5em]{$\longrightarrow$};
        		\draw[-latex] (MonitorTree.south) -- (MonitorTaskSequence.north);
        		
        		\node (NearHumanFallback) at ($(MonitorTaskSequence) + (-1.5,-1.5)$) [text centered, fill=magenta!5, draw=magenta, rectangle, minimum width=0.5cm, text width=1.5em]{\textbf{?}};
        		\draw[-latex] (MonitorTaskSequence.south) -- (NearHumanFallback.north);
        		\node (MonitorHumanTarget) at ($(MonitorTaskSequence) + (1.5,-1.5)$) [text centered, fill=blue!5, draw=blue, rectangle, minimum width=1.5cm, text width=5.5em]{Monitor};
        		\draw[-latex] (MonitorTaskSequence.south) -- (MonitorHumanTarget.north);
        		
        		\node (IsUAVnearHumanTarget) at ($(NearHumanFallback) + (-2,-2)$) [text centered, fill=blue!5, draw=blue, ellipse, minimum width=1.5cm, text width=6.5em]{Is \gls{ACW} near Human Target?};
        		\draw[-latex] (NearHumanFallback.south) -- (IsUAVnearHumanTarget.north);
        		\node (GoNearHuman) at ($(NearHumanFallback) + (1.55, -2)$) [text centered, fill=blue!5, draw=blue, rectangle, minimum width=1.5cm, text width=6.5em]{Go near Human Target};
        		\draw[-latex] (NearHumanFallback.south) -- (GoNearHuman.north);
		    \end{tikzpicture}}
		\caption{Behavior Tree: sub-tree that controls the safety monitoring tasks}
		\label{fig:MonitorTree}
	\end{center}
	\vspace{-1em}
\end{figure}

\subsection{Tool delivery task tree}
\label{sec:ToolDeliveryTaskTree}
% Del WP7_Scenarios
\textbf{High-Level Planner inputs}: Task ID, Task Type, Human Target ID and Tool ID (the rest will be ignored).
\textbf{Description}: After task allocation, the High-Level Planner will send the information to the corresponding Agent Behavior Manager and from there, the Lower-Level Controllers will be called sequentially as shown in Fig. \ref{fig:DeliverToolTree}. Basically, the \gls{ACW} needs to navigate to the station where the tool is, pick it up, navigate back to where the worker is, and start physical interaction to deliver the tool. 

\begin{figure}[ht]
	\begin{center}
		\scalebox{1}{
			\begin{tikzpicture}
			    \node (DeliverTree) at (0,0) [text centered, fill=white, draw, rectangle, minimum width=1.5cm, text width=6.5em]{Tool Delivery Task Tree};
			    
			    \node (DeliverTaskSequence) at ($(DeliverTree) + (0,-1)$) [text centered, fill=magenta!5, draw=magenta, rectangle, minimum width=1.5cm, text width=1.5em]{$\longrightarrow$};
        		\draw[-latex] (DeliverTree.south) -- (DeliverTaskSequence.north);

				\node (ToolFallback) at ($(DeliverTaskSequence) + (-6.5,-1.5)$) [text centered, fill=magenta!5, draw=magenta, rectangle, minimum width=0.5cm, text width=0.5em]{\textbf{?}};
        		\draw[-latex] (DeliverTaskSequence.south) -- (ToolFallback.north);
        		\node (HumanFallback) at ($(DeliverTaskSequence) + (0,-1.5)$) [text centered, fill=white, draw, rectangle, minimum width=0.5cm, text width=0.5em]{\textbf{?}};
        		\draw[-latex] (DeliverTaskSequence.south) -- (HumanFallback.north);
        		%\node (PermissionFallback) at ($(DeliverTaskSequence) + (6.5,-1.5)$) [text centered, fill=magenta!5, draw=magenta, rectangle, minimum width=0.5cm, text width=0.5em]{\textbf{?}};
        		%\draw[-latex] (DeliverTaskSequence.south) -- (PermissionFallback.north);
				\node (DeliverTool) at ($(DeliverTaskSequence) + (3.5,-1.5)$) [text centered, fill=blue!5, draw=blue, rectangle, minimum width=1.5cm, text width=6.5em]{Deliver Tool};
        		\draw[-latex] (DeliverTaskSequence.south) -- (DeliverTool.north);

				\node (hasACWtheTool) at ($(ToolFallback) + (-1.5,-2)$) [text centered, fill=blue!5, draw=blue, ellipse, minimum width=1.5cm, text width=5.5em]{Has \gls{ACW} the Tool?};
        		\draw[-latex] (ToolFallback.south) -- (hasACWtheTool.north);
        		\node (PickToolSequence) at ($(ToolFallback) + (1.5,-2)$) [text centered, fill=magenta!5, draw=magenta, rectangle, minimum width=1.5cm, text width=1.5em]{$\longrightarrow$};
        		\draw[-latex] (ToolFallback.south) -- (PickToolSequence.north);
        		\node (IsUAVnearHuman) at ($(HumanFallback) + (-1.75,-2)$) [text centered, fill=blue!5, draw=blue, ellipse, minimum width=1.5cm, text width=6.5em]{Is \gls{ACW} near Human Target?};
        		\draw[-latex] (HumanFallback.south) -- (IsUAVnearHuman.north);
        		\node (GoNearHuman) at ($(HumanFallback) + (1.75, -2)$) [text centered, fill=blue!5, draw=blue, rectangle, minimum width=1.5cm, text width=6.5em]{Go near Human Target};
        		\draw[-latex] (HumanFallback.south) -- (GoNearHuman.north);
        		%\node (DeliverSequence) at ($(PermissionFallback) + (-2.25,-2)$) [text centered, fill=white, draw, rectangle, minimum width=1.5cm, text width=1.5em]{$\longrightarrow$};
        		%\draw[-latex] (PermissionFallback.south) -- (DeliverSequence.north);
        		%\node (ForceFailure) at ($(PermissionFallback) + (2.25, -2)$) [text centered, fill=orange!5, draw=orange, rectangle, minimum width=1.5cm, text width=5.5em]{Force Failure};
        		%\draw[-latex] (PermissionFallback.south) -- (ForceFailure.north);


        		\node (StationFallback) at ($(PickToolSequence) + (-1.5,-2)$) [text centered, fill=white, draw, rectangle, minimum width=0.5cm, text width=0.5em]{\textbf{?}};
        		\draw[-latex] (PickToolSequence.south) -- (StationFallback.north);
        		\node (PickTool) at ($(PickToolSequence) + (1.5, -2)$) [text centered, fill=blue!5, draw=blue, rectangle, minimum width=1.5cm, text width=6.5em]{Pick Tool};
        		\draw[-latex] (PickToolSequence.south) -- (PickTool.north);
        		%\node (hasUAVpermission) at ($(DeliverSequence) + (-3.25,-2)$) [text centered, fill=white, draw, ellipse, minimum width=1.5cm, text width=5.5em]{Has  permission?};
        		%\draw[-latex] (DeliverSequence.south) -- (hasUAVpermission.north);
        		%\node (DeliverTool) at ($(DeliverSequence) + (0, -2)$) [text centered, fill=white, draw, rectangle, minimum width=1.5cm, text width=6.5em]{Tool Delivery};
        		%\draw[-latex] (DeliverSequence.south) -- (DeliverTool.north);
        		%\node (Retreat) at ($(DeliverSequence) + (3, -2)$) [text centered, fill=white, draw, rectangle, minimum width=1.5cm, text width=4.5em]{Retreat};
        		%\draw[-latex] (DeliverSequence.south) -- (Retreat.north);
        		%\node (WaitFallback) at ($(ForceFailure) + (0,-2)$) [text centered, fill=magenta!5, draw=magenta, rectangle, minimum width=0.5cm, text width=0.5em]{\textbf{?}};
        		%\draw[-latex] (ForceFailure.south) -- (WaitFallback.north);
        		
        		\node (IsUAVnearStation) at ($(StationFallback) + (-1.75,-2)$) [text centered, fill=blue!5, draw=blue, ellipse, minimum width=1.5cm, text width=5.5em]{Is \gls{ACW} near Station?};
        		\draw[-latex] (StationFallback.south) -- (IsUAVnearStation.north);
        		\node (GoNearStation) at ($(StationFallback) + (1.75, -2)$) [text centered, fill=blue!5, draw=blue, rectangle, minimum width=1.5cm, text width=6.5em]{Go near Station};
        		\draw[-latex] (StationFallback.south) -- (GoNearStation.north);
        		%\node (TimeoutSequence) at ($(WaitFallback) + (-1.5,-2)$) [text centered, fill=white, draw, rectangle, minimum width=1.5cm, text width=1.5em]{$\longrightarrow$};
        		%\draw[-latex] (WaitFallback.south) -- (TimeoutSequence.north);
        		%\node (Wait) at ($(WaitFallback) + (1.5, -2)$) [text centered, fill=white, draw, rectangle, minimum width=1.5cm, text width=6.5em]{Wait for Permission};
        		%\draw[-latex] (WaitFallback.south) -- (Wait.north);

        		%\node (Timeout) at ($(TimeoutSequence) + (-3.25,-2)$) [text centered, fill=white, draw, ellipse, minimum width=1.5cm, text width=5.5em]{Timeout?};
        		%\draw[-latex] (TimeoutSequence.south) -- (Timeout.north);
        		%\node (GoNearStation) at ($(TimeoutSequence) + (0, -2)$) [text centered, fill=white, draw, rectangle, minimum width=1.5cm, text width=6.5em]{Go near Station};
        		%\draw[-latex] (TimeoutSequence.south) -- (GoNearStation.north);
        		%\node (DropTool) at ($(TimeoutSequence) + (3.25, -2)$) [text centered, fill=white, draw, rectangle, minimum width=1.5cm, text width=6.5em]{Drop the Tool};
        		%\draw[-latex] (TimeoutSequence.south) -- (DropTool.north);
		    \end{tikzpicture}}
		\caption{Behavior Tree: sub-tree that controls the tool delivery tasks}
		\label{fig:DeliverToolTree}
	\end{center}
\end{figure}

\section{Lower and upper level modules faker}
\label{sec:LowerAndUpperLevelModulesFaker}
%%% GoToWP, Recharge, Monitoring, Inspection, ToolDelivery
%%% Battery sensor
%% Faker del reparto de WPs entre los ACWs seleccionados

% Informe de actividades: Battery faker
% Otra de las labores llevadas a cabo, esta de forma reciente, ha sido la de la elaboración de un nodo que simule ser la batería del UAV y comunique el nivel de batería falso a través de un tópico de ROS llamado “/mavros/battery_faker”. Esta implementación ha sido necesaria debido a que ni MAVROS ni UAL proporcionaban métodos para el control de la batería. Gracias a este nodo se pueden realizar ahora simulaciones en las que se controle la velocidad de carga y descarga de la batería, se fije la batería a un valor determinado y se controle en qué momentos se carga y se descarga la batería. Este nodo se está empleando en las simulaciones para probar el funcionamiento de la solución planteada. Concretamente, es de vital importancia asegurar que el BT detecta el estado de la batería correctamente y actúa en consecuencia, deteniendo la ejecución de la tarea actual de ser necesario y dirigiendo al UAV hacia la estación de recarga más cercana. Acompañando al nodo falseador de batería se ha creado un “Action” de ROS para controlar los modos de funcionamiento de este nodo, el nivel de la batería y las velocidades de carga y descarga de la misma.
%
% Diseño de los experimentos, simulaciones
% código, conclusiones, líneas futuras.
\chapter{Conclusions and future work}
\label{ch:ConclusionsAndFutureWork}
% Las conclusiones en formato:
    % Se ha hecho X, Y, y funciona muy bien.
    % Se ha visto que ocurre A, B, C

\section{Conclusions}
\label{sec:Conclusions}
% Optimalidad del planner, funcion de costes, aproximación realizada, simplificaciones realizadas: decir que la aproximación del Planner en casos reales es aceptable
% El BT se puede mejorar pero funciona muy bien y sienta las bases para programar comportamientos más complejos en el futuro. Servirá de ejemplo para la comunidad. Permite generar comportamientos complejos y numerosos estados sin que haya que preocuparse de las transiciones entre estados como pasa con las FSM, en las que este crece exponencialmente con el número de estados.
% Planificación de recargas: Se podría haber añadido una tarea de tipo Recharge para planificarlas. No planifica teniendo en cuenta las recargas ahora mismo, solo si hay batería suficiente. Eso se puede mejorar.
% El "fallo" que he encontrado en el Tool Delivery Task Tree. Devolve la herramienta a la base en caso de fallo.
% Separar el Recharge Action Node en la estructura típica (la de Inspection Task Tree): aproximación a la base por un lado y Recarga por otro.
% Importancia del battery faker para testear correctamente situaciones de emergencia.

\section{Future work}
\label{sec:FutureWork}
% Task ¿?
% Planificación de recargas: Se podría haber añadido una tarea de tipo Recharge para planificarlas. No planifica teniendo en cuenta las recargas ahora mismo, solo si hay batería suficiente. Eso se puede mejorar.
% Realidad aumentada
% Introducir algoritmos heuristicos aleatorios en el planificador para encontrar el plan óptimo de verdad.
% ¿Redes neuronales?

\endinput

%
%\chapter{References}
\label{ch:References}



\endinput

%

%%%%%%%%%%%%%%%%%%%%%%%%%%%%%%%%%%%%%%%%%%%%%%%%%%%%%%%%%%%%%%%%%%%%%%%%%%%%%%%
%%%%%%%%%%%%%%%%%%%%%%%%%%%%%%%%%%%%%%%%%%%%%%%%%%%%%%%%%%%%%%%%%%%%%%%%%%%%%%%
%%%%%%% Esto aún no lo he investigado, tengo que ver como va
%%%%%%%%%%%%%%%%%%%%%%%%%%%%%%%%%%%%%%%%%%%%%%%%%%%%%%%%%%%%%%%%%%%%%%%%%%%%%%%
%%%%%%%%%%%%%%%%%%%%%%%%%%%%%%%%%%%%%%%%%%%%%%%%%%%%%%%%%%%%%%%%%%%%%%%%%%%%%%%
%%%%%%% Apéndices
%%:Empezamos con los apéndices, que irían en uno o más ficheros. Es necesario incluir estos ficheros entre el entorno \begin{appendices}....\end{appendices} debido a que se ha deseado utilizar un formato diferente para el título de los apéndices, incluyendo la palabra apéndice, para la numeración de los apéndices, alfabético, y para las cabeceras de las páginas.
%
% \begin{appendices}
%
% % !TEX root =../LibroTipoETSI.tex



%APENDICE A
\chapter{Sobre  \LaTeX}\LABAPEN{ApA}
{Este es un ejemplo de apéndices, el texto es únicamente relleno, para que el lector pueda observar cómo se utiliza}
%%%%%%%%%%%%%%%%%
\section{Ventajas de \LaTeX}

El gusto por el \LaTeX\ depende de la forma de trabajar de cada uno. La principal virtud es la facilidad de formatear cualquier texto y la robustez. Incluir títulos, referencias es inmediato.
%\Blindtext
%\lipsum
Las ecuaciones quedan estupendamente, como puede verse en \EQ{Ap1}
\begin{equation}\LABEQ{Ap1}
x_{1}=x_{2}.
\end{equation}


\section{Inconvenientes}
%\Blindtext
El principal inconveniente de \LaTeX\ radica en la necesidad de aprender un conjunto de comandos para generar los elementos que queremos. Cuando se está acostumbrado a un entorno ``como lo escribo se obtiene'', a veces resulta difícil dar el salto a ``ver'' que es lo que se va a obtener con un determinado comando. 

Por otro lado, en general será muy complicado cambiar el formato para desviarnos de la idea original de sus creadores. No es imposible, pero sí muy difícil. Por ejemplo, con la sentencia siguiente:
 
\begin{lstlisting}[language=,caption={Escritura de una ecuación}, breaklines=true, label=prgA1-01]
\begin{equation}\LABEQ{Ap2}
x_{1}=x_{2}
\end{equation}
\end{lstlisting}
obtenemos:
\begin{equation}\LABEQ{Ap2}
x_{1}=x_{2}
\end{equation}
Esto será siempre así. Aunque, tal vez, esto podría ser una ventaja y no un incoonveniente.

Para una discusión similar sobre el Word\tsp{\textregistered}, ver \APEN{ApB}.
%\Blindtext


%%%%%%%%%%%%%%%%%%%%%%%%%%%%%%%%%%%%%%%
%APENDICE B
\chapter{Sobre Microsoft Word\tsp{\textregistered}}\LABAPEN{ApB}

\section{Ventajas del Word\tsp{\textregistered}}
La ventaja mayor del Word\tsp{\textregistered} es que permite configurar el formato muy fácilmente. Para las ecuaciones,
\begin{equation}
x_{1}=x_{2},
\end{equation}
tradicionalmente ha proporcionado pésima presentación. Sin embargo, el software adicional Mathtype\tsp{\textregistered} solventó este problema, incluyendo una apariencia muy profesional y cuidada. Incluso permitía utilizar un estilo similar al \LaTeX\xspace. Además, aunque el Word\tsp{\textregistered} incluye sus propios atajos para escribir ecuaciones,  Mathtype\tsp{\textregistered} admite también escritura \LaTeX\xspace. En las últimas versiones de Word\tsp{\textregistered}, sin embargo, el formato de ecuaciones está muy cuidado, con un aspecto similar al de \LaTeX.


\section{Inconvenientes de Word\tsp{\textregistered}}
Trabajar con títulos, referencias cruzadas e índices es un engorro, por no decir nada sobre la creación de una tabla de contenidos. Resulta muy frecuente que alguna referencia quede pérdida o huérfana y aparezca un mensaje en negrita indicando que  no se encuentra. 

Los estilos permiten trabajar bien definiendo la apariencia, pero también puede desembocar en un descontrolado incremento de los mismos. Además, es muy probable que Word\tsp{\textregistered} se quede colgado, sobre todo al trabajar con copiar y pegar de otros textos y cuando se utilizan ficheros de gran extensión, como es el caso de un libro.

%\end{equation}
 
%
% \end{appendices}

%%%%%%%%%%%%%%%%%%%%%%%%%%%%%%%%%%%%%%
%%%%%%%%%%%%%%%%%%%%%%%%%%%%%%%%%%%%%%
%:Empieza todo lo que no constituye el cuerpo en si del libro. Todo lo que va detrás
\backmatter

%:Indice de figuras, coméntese las siguientes líneas si no se desea
\cleardoublepage
\phantomsection

%:Para añadir una línea en blanco en el TOC y separar esta lista
\addtocontents{toc}{\protect\mbox{}\protect\hspace*{0pt}\par}
\addcontentsline{toc}{listasb}{\listfigurename}
\pagestyle{especial}
\listoffigures

%:Indice de tablas, coméntese las siguientes líneas si no se desea
\cleardoublepage
\phantomsection
\addcontentsline{toc}{listasb}{\listtablename}
\pagestyle{especial}
\listoftables

%:Indice de Programas
\cleardoublepage
\phantomsection
\addcontentsline{toc}{listasb}{\lstlistlistingname}
\pagestyle{especial}
\lstlistoflistings

%%%%%%%%%%%%%%%%%%%%%%%%%%%%%%%%%%%%%%%%%%%%%%%%%%%%%%%%%%%%%%%%%%%%%%%%%%%%%%%
%:Bibliografía con biblatex
\nocite{*}
\cleardoublepage
\phantomsection
\addcontentsline{toc}{listasb}{\bibname}
\pagestyle{especial}

\bibliographystyle{IEEEtran}
%\bibliographystyle{amsplain} %flexbib amsplain alpha

%:Fichero con la bibliografía, BIBTEX
\bibliography{bibliography}

% Este fichero .bib se puede generar usando algún gestor de bibliografías. Se recomiendan dos:
% - Zotero
% - Mendeley (con licencia de la US)

%:Índice alfabético de palabras
\cleardoublepage
\phantomsection
\addcontentsline{toc}{listasb}{\indexname}
\chaptermark{\indexname}
\printindex


%:Acrónimos
\cleardoublepage
\phantomsection
\addcontentsline{toc}{listasb}{\glossaryname}
\chaptermark{\glossaryname}
\printglossaries


\end{document}
