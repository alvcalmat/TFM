%%%%%%%%%%%%%%%%%%%%%%%%%%%%%%%%%%%%%%%%%%
%%% NORMALMENTE NO ES NECESARIO HACER 
%%% CAMBIOS EN ESTA PARTE DEL DOCUMENTO
%%%%%%%%%%%%%%%%%%%%%%%%%%%%%%%%%%%%%%%%%%


%:Clase del documento
\documentclass[fontsize=11pt, English=true, Myfinal=true, twoside, numbers=noenddot]{scrbook}
%Minion=true, English=true, Myfinal=true

%:Paquete de estilos propuesto
\usepackage{libroETSI}

%:Paquete específico para cargar tikz (y sus librerías) y pgfplots
\usepackage{dtsc-creafig}

%:Paquete para notaciones específicas
\usepackage{notacion}

%:Paquete para incorporar aspectos concretos de la edición
\usepackage{edicionPFC}

% Paquete para incluir epígrafes en los capítulos
\usepackage{epigraph}

% Paquete para incluir glosario
\usepackage{glossaries}

%:Para modificar fácilmente la fuente del texto.
\makeatletter
\ifdtsc@Minion % Queremos utilizar la fuente Minion y lo hemos declarado al principio
	\ifluatex
		\setmainfont[Renderer=Basic, Ligatures=TeX,	% Fuente del texto 
		Scale=1.01,
		]{Minion Pro}
   		% En este caso conviene modificar ligeramente el tamaño de las fuentes matemáticas
		\DeclareMathSizes{10}{10.5}{7.35}{5.25}
		\DeclareMathSizes{10.95}{11.55}{8.08}{5.77}
		\DeclareMathSizes{12}{12.6}{8.82}{6.3}
%		\setmainfont[Renderer=Basic, Ligatures=TeX,	% Fuente del texto 
%		]{Adobe Garamond Pro}
%		\setmainfont[Renderer=Basic, Ligatures=TeX,	% Fuente del texto 
%		]{Palatino LT Std}
	\fi
\else
	\ifluatex
		% Para utilizar la fuente Times New Roman, o alguna otra que se tenga instalada
		\setmainfont[Renderer=Basic, Ligatures=TeX,	% Fuente del texto 
		Scale=1.0,
		]{Times New Roman}
	\else
		\usepackage{tgtermes} 	%clone of Times
		%\usepackage[default]{droidserif}
		%\usepackage{anttor} 	
	\fi
\fi
\makeatother

% Formato A4
\geometry
{paperheight=297mm,%
paperwidth=210mm,%
top=25mm,%
headsep=8.5mm,%
includefoot, 
textheight=240mm, 
textwidth=150mm, 
bindingoffset=0mm, 
twoside}

\usepackage[a4,center]{crop}%para poner las cruces de esquina de página, poner la opción cross

%:Esquema de numeración por defecto
\setenumerate[1]{label=\normalfont\bfseries{\arabic*.}, leftmargin=*, labelindent=\parindent}
\setenumerate[2]{label=\normalfont\bfseries{\alph*}), leftmargin=*}
\setenumerate[3]{label=\normalfont\bfseries{\roman*.}, leftmargin=*}
\setlist{itemsep=.1em}
\setlength{\parindent}{1.0 em}

\setcounter{tocdepth}{4}						% El nivel hasta el que se muestra el índice 


%%%%%%%%%%%%%%%%%%%%%%%%%%%%%%%%%%%%%%%%%%
%%% A PARTIR DE AQUÍ HAY QUE EDITAR
%%%%%%%%%%%%%%%%%%%%%%%%%%%%%%%%%%%%%%%%%%

% Ejemplo de Glosario
\newacronym[type=main]{ETSI}{ETSI}{Escuela Técnica Superior de Ingeniería}
\newacronym[type=main]{US}{US}{Universidad de Sevilla}
%\newacronym[type=main]{}{}{}
\newacronym[type=main]{SEU}{SEU}{Single Event Upset}
\newacronym[type=main]{SEE}{SEE}{Single Event Effect}
\newacronym[type=main]{TMR}{TMR}{Triple Modular Redundancy}
\newacronym[type=main]{RBM}{RBM}{Restricted Boltzmann Machine}
\newacronym[type=main]{CUT}{CUT}{Circuit Under Test}
\newacronym[type=main]{FF}{FF}{flip-flop}
\newacronym[type=main]{ESA}{ESA}{European Space Agency}
\newacronym[type=main]{FPGA}{FPGA}{Field Programmable Gate Array}
\newacronym[type=main]{FFT}{FFT}{Fast Fourier Transform}
\newacronym[type=main]{FIFO}{FIFO}{First in, First out}
\newacronym[type=main]{FSM}{FSM}{Finite State Machine}
\newacronym[type=main]{UART}{UART}{Universal Asynchronous Receiver and Transmitter}


\makeindex
\makeglossaries %Si no se quiere el glosario, comentar esta línea.


%:Empieza el documento

\begin{document}


%PORTADA
%ver edicionPFC.sty para modificaciones

%:Para crear la portada y la portada interior (pagina titular)
\titulo{Aerial co-workers: a task planning approach for multi-drone teams supporting inspection operations} %\mbox evita que se divida una palabra al cambiar de línea
\autor{Álvaro Calvo Matos}
\director{Jesús Capitán Fernandez}
\titulodirector{Associate Professor}

\departamento{Dpto. Ingeniería de Sistemas y Automática}
%\departamento{Systems and Automation Engineering Department}
\centro{Escuela Técnica Superior de Ingeniería}
\universidad{Universidad de Sevilla}
%\universidad{University of Seville}
\titulacion{Máster en Ingeniería Electrónica, Robótica y Automática}
%\titulacion{Master in Electronic, Robotic and Automation Engineering}
\fecha{2021}
\nombretrabajo{Trabajo Fin de Máster} 


\hypersetup
	{
 	linkcolor=black, %Tocar para poner color en enlaces
	pdfauthor={\elautor},
	pdftitle={\nombretrabajo,\eltitulo}, 
	pdfkeywords={Latex, edición, formato de texto}	
	 }

%logo de la Universidad y logo del departamento, si lo hubiera. Para cambiar el pie de página con los logos, debe editarse el fichero ediciónPFC.sty
\portadaPFC{figuras/LogoUS.pdf}{figuras/LogoTSC.pdf} 
% Para incluir el logo del departamento hay que modificar el segundo parámetro de la linea anterior de este .tex, y
% hay que modificar las lineas 92 a 100 del fichero "edicionPFC.sty"

%Fin Portada

%:Todo lo que constituye la primera parte del libro que no es el cuerpo del libro en realidad
\frontmatter
\pagenumbering{Roman} %Pone la numeración en mayúscula (En español parece que es obligatorio)

%\include{dedicatoria/dedicatoria}%¿Comentar para proyectos/tesis?
\chapter*{Agradecimientos}
%\pagestyle{especial}
\pagestyle{empty}
%\chaptermark{Agradecimientos}
\phantomsection
%\addcontentsline{toc}{listasf}{Agradecimientos}
%\vspace{1cm}
%{\huge{Agradecimientos}}
%\vspace{1cm}

\lettrine[lraise=-0.1, lines=2, loversize=0.25]{}{}
Lorem itsum
% Tutor del proyecto:

% Compañeros del departamento

% Maestros de la carrera

% Compañeros de clase, por acompañarde durante todo el camino, en especial a 
% Damian por su apoyo y amistad en todo momento durante este último año.

% La familia

{\flushleft{\hfill \emph{Álvaro Calvo Matos}}}%
\vspace{-.3cm}
{\flushleft{\hfill \emph{Máster en Ingeniería Electrónica, Robótica y Automática}}}
{\flushleft{\hfill \emph{Sevilla, 2021}}}%


%PFC/PFM/TESIS
\chapter*{Abstract}
\pagestyle{especial}
\chaptermark{Abstract}
\phantomsection
\addcontentsline{toc}{listasf}{Abstract}
\lettrine[lraise=-0.1, lines=2, loversize=0.2]{L}{o}rem itsum
%%% Hablar del problema que aborda el TFM.
% Este Trabajo de Fin de Máster ha afrontado problemas que surgen del reciente aumento de las aplicaciones de equipos cooperativos de UAV, los cuales son la autonomía para operar de forma prolongada en el tiempo con robustez ante posibles fallos, y la dificultad de aportar al equipo capacidades cognitivas para poder operar en entornos dinámicos con humanos. 
This Master's Thesis has faced problems that arise from the recent increase in the applications of cooperative UAV teams, which are the autonomy to operate for a long time with robustness in the face of possible failures, and the difficulty of providing the team with capabilities cognitive skills to be able to operate in dynamic environments with humans. 

%%% Hablar de la importancia o del interés que hay por solucionar el problema.
% Muchas de estas aplicaciones están siendo ejecutadas actualmente por humanos, haciendo las actividaded mucho más costosas, lentas, e incluso en algunos casos, peligrosas. Es por eso que actualmente existe un gran interés y se están destinando muchos esfuerzos para desarrollar soluciones para los problemas planteados, ya que pueden suponer, además de un ahorro significativo para las empresas, una mejora drástica en la seguridad de los trabajadores en aquellos trabajos que sean de alto riesgo. Concretamente, la aplicación que en la que se ha centrado este trabajo es la asistencia a operarios humanos en tareas de inspección y mantenimiento en líneas eléctricas de alta tensión.

%%% Objetivos que se persiguen: ¿Por qué realizo esta investigación? ¿Qué se busca lograr? ¿Objetivo? ¿Hipótesis de partida?
% El objetivo del trabajo era desarrollar técnicas cognitvas de planificación para coordinar flotas de quadrotors que asistan a operarios humanos en tareas de inspección y mantenimiento en líneas eléctricas de alta tensión. Estas técnicas debían además extender la autonomía del sistema, garantizar que se cumplen los requisitos de seguridad entre drones y trabajadores humanos, y asegurar el éxito de la misión.

%%% Descripción de la solución propuesta. ¿Cómo lo he hecho? ¿Técnicas utilizadas?
% Se ha propuesto una arquitectura de software basada en un planificador central, que se encarga de realizar la planificación de acciones en el tiempo y de controlar el estado de cada uno de los equipos conectados; y del gestor del comportamiento, distribuido a bordo de cada uno de los UAV, que se encarga de ejecutar el plan asignado por el módulo central. En el módulo distribuido se encuentra la mínima inteligencia que asegura el cumplimiento de los requisitos de seguridad, de forma que el resto de la inteligencia se pueda concentrar en el módulo centralizado. De esta forma se reduce lo máximo posible la carga computacional sobre los equipos aéreos, y por tanto, se alarga la vida de la batería. Se ha supuesto además que se dispone de alguna forma de recargar la batería durante la misión, por lo que entre las acciones de las que dispone el módulo central para planificar, se encuentra además la opción de recargar. Para llevar a cabo la planificación se ha definido un coste, que es calculado para cada tarea. Respetando entre otras cosas las prioridades de las tareas y su orden de llegada, cada tarea se asigna al UAV al que cueste menos su ejecución. Por el otro lado, para controlar el comportamiento de los drones y asegurar la seguridad de los equipos aéreos, se ha implementado un árbol de comportamiento.

%%% Resultados: Datos más importantes que respondan a las hipótesis y los objetivos marcados.
% Como resultado, se ha conseguido desarrollar una arquitectura de software capaz realizar la planificación de las misiones de forma dinámica asegurando mientras tanto la seguridad de los equipos involucrados. Gracias a la planificación, se consigue una mejor coordinación de los UAV y por tanto, un mejor aprovechamiento de la batería, alargrando así la autonomía de los equipos. El módulo central constituye una buena base que se puede adaptar fácilmente a otros proyectos que involucren equipos de múltiples UAV y de la cual se puede partir para desarrollar futuros planificadores más complejos. A su vez, el diseño de los módulos distribuidos, gracias al uso de árboles de comportamiento, permite una fácil reutilización y modificación. Comparado con la forma típica de implementación de este tipo de módulos, la cual involucra la creación de complejas máquinas de estados difíciles de leer para una persona, de reutilizar y de ampliar, el uso de árboles de comportamiento supone una gran mejora y permitirá la creación de comportamientos cada vez más complejos.

%%% Resumir la importancia de los resultados y de sus posibles aplicaciones. 

% Índice abreviado 
% El índice abreviado se incluye también en algunos libros, con menor detalle que el completo. Descomentar las siguientes líneas.
\cleardoublepage
\phantomsection
\addcontentsline{toc}{listasf}{Abbreviated index}
\pagestyle{especial}
\shorttoc{Abbreviated index}{1}

%Índice normal, el completo
\cleardoublepage
\phantomsection
\pagestyle{especial}
\tableofcontents

%%%%%%%%%%%%%%%%%%%%%%%%%%%%%%%%%%%%%%%%%%%%%%%%%%%%%%%%%%%%%%%%%%%%%%%%%%%%%%%
%%%%%%% Descomentar la siguiente linea y editar notacion.tex si hiciera falta
%%%%%%% incluir notación en el TFG.
%\chapter*{\notationname}
\pagestyle{especial}
\chaptermark{\notationname}
\phantomsection
\addcontentsline{toc}{listasf}{\notationname}
%\section*{Notación}
%\begin{table}[htbp]
\begin{longtable}{p{3cm}p{8.5cm}}

%$\displaystyle D$ & Tasa de símbolos  (sim/s) \\
%$\displaystyle R_b$ & Tasa binaria (bit/s) \\
%$\displaystyle T$ & Tiempo de símbolo (s) \\
%$\displaystyle T_{b}$ & Tiempo de bit (s) \\
%$W\left( {t} \right)$ & Ruido blanco\\
%$w\left( {t} \right)$ & Función muestra de un ruido blanco\\
%$\displaystyle h_{c}\left( {t} \right)$ & Respuesta impulsiva de un canal LTI continuo en el tiempo\\
%$\displaystyle H_{c}\left( {\omega} \right)$ & Respuesta en frecuencia de un canal LTI continuo en el tiempo\\
%$\displaystyle h_{c}\left( {\tau;t} \right)$ & Respuesta impulsiva de un canal LTV continuo en el tiempo\\
%$\displaystyle H_{c}\left( {\omega;t} \right)$ & Respuesta en frecuencia de un canal LTV continuo en el tiempo\\
%$\displaystyle h_{c}\left( {n} \right)$ & Respuesta impulsiva de un canal LTI discreto en el tiempo\\
%$\displaystyle H_{c}\left( {\Omega} \right)$ & Respuesta en frecuencia de un canal LTI discreto en el tiempo\\
$\RR$ & Cuerpo de los números reales \\
$\CC$ & Cuerpo de los números complejos\\
$\left\| \vc{v} \right\|$ & Norma del vector $\vc{v}$ \\
$\left\langle {\vc{v}, \vc{w}} \right\rangle$ & Producto escalar de los vectores $\vc{v}$ y $\vc{w}$\\
$\left| {\vc{A}} \right|$ &Determinante de la matriz cuadrada $\vc{A}$\\
$\textrm{det}\left( {\vc{A}} \right)$ &Determinante de la matriz (cuadrada) $\vc{A}$\\
$\vc{A}\trs$ & Transpuesto de $\vc{A}$\\
$\vc{A}\inv$ & Inversa de la matriz $\vc{A}$\\
$\vc{A}{\psd}$ & Matriz pseudoinversa de la matriz $\vc{A}$\\
$\vc{A}\her$ & Transpuesto  y conjugado de $\vc{A}$\\
$\vc{A}\cnj$ & Conjugado\\
c.t.p. & En casi todos los puntos\\
c.q.d. & Como queríamos demostrar\\
\ensuremath{\blacksquare}& Como queríamos demostrar\\
\ensuremath{\square}& Fin de la solución\\
e.o.c. & En cualquier otro caso\\
$\e$ & número e\\
$\xp{x}$ & Exponencial compleja\\
$\xppi{x}$ & Exponencial compleja con $2\pi$\\
$\nxp{x}$ & Exponencial compleja negativa\\
$\nxppi{x}$ & Exponencial compleja negativa con $2\pi$\\
$\re$ & Parte real\\
$\im$ & Parte imaginaria\\
$\sen$ & Función seno \\
$\tg$ & Función tangente \\
$\arctg$ & Función arco tangente \\
$\sento{y}{x}$ & Función seno de $x$  elevado a $y$\\
$\costo{y}{x}$ & Función coseno de $x$  elevado a $y$\\
$\sa$ & Función sampling \\
$\sgn$ & Función signo \\
$\rect$ & Función rectángulo \\
$\sinc$ & Función sinc\\
$\pder{y}{x} $ & Derivada parcial de $y$ respecto a $x$\\
$x\gra$ & Notación de grado, $x$ grados.\\
%
%$C_{XY}$& covarianza  de dos variables aleatorias reales $X$ e $Y$\\
%$R_{XY}$& correlación  de dos variables aleatorias reales $X$ e $Y$\\
%$\rho_{XY}$ &Coeficiente de correlación de las variables aleatorias reales $X$  e $Y$\\
%$\vc{Z}$ & Vector aleatorio complejo\\
%$\displaystyle F_{X}\left( {\cdot} \right)$ & Función de distribución de la variable aleatoria $X$ \\
%$\displaystyle f_{X}\left( {\cdot} \right)$ & Función densidad de probabilidad de la variable aleatoria $X$ \\
%$p_{X}\left( {\cdot} \right)$ & Función masa de probabilidad de la variable aleatoria discreta $X$ \\
%
$\Pr\left( {A} \right)$ & Probabilidad del suceso $A$ \\
$\displaystyle E\left[ {X} \right]$ & Valor esperado de la variable aleatoria $X$ \\
$\si{X}$ & Varianza de la variable aleatoria $X$\\
$\sim f_{X}\left( {x} \right)$ & Distribuido siguiendo la función densidad de probabilidad $f_{X}\left( {x} \right)$\\
%
$\gauss{m_{X}}{\si{X}}$ &Distribución gaussiana para la variable aleatoria X, de media $m_{X}$ y varianza $\si{X}$ \\
$\id{n}$ & Matriz identidad de dimensión $n$\\
$\diag{\vc{x}}$ & Matriz diagonal a partir del vector $\vc{x}$\\
$\diag{\vc{A}}$ & Vector diagonal de la matriz $\vc{A}$\\
$\snr$& Signal-to-noise ratio \\
$\mse$ & Minimum square error\\
$\talq$ & Tal que \\
$\eqdef$ & Igual por definición \\
$\norm{\vc{x}}$ & Norma-2 del vector $\vc{x}$\\
$\card{\vc{{A}}}$ & Cardinal, número de elementos del conjunto $\vc{A}$\\
$\xyz{\vc{x}}{i}{n}$ & Elementos $i$, de 1 a $n$, del vector $\vc{x}$\\
%\newcommand{\xyz}[3]{\ensuremath{#1_{#2},#2=1,2,\ldots,#3}}
$\df{x}$& Diferencial de $x$\\
$\le$ & Menor o igual \\
$\ge$ & Mayor o igual \\
$\BL$ & Backslash \\
$\iff$ & Si y sólo si \\
$x=a+3\eqexpl{a=1} 4 $& Igual con explicación \\
$\tfrac{a}{b}$ & Fracción con estilo pequeño, $a/b$ \\
$\inc$ & Incremento \\
$b\ten{a}$ & Formato científico \\
$\tendsub{x}$ & Tiende, con x \\
$\ord$ & Orden\\
$\tm$ & Trade Mark\\
$\E[x]$ & Esperanza matemática de x\\
$\covm{\vc{x}}$ & Matriz de covarianza de $\vc{x}$\\
$\corrm{\vc{x}}$ & Matriz de correlación de $\vc{x}$\\
$\si{x}$ & Varianza de x \\


\end{longtable}
\newpage
%\end{table}
%


%\phantomsection
%\addcontentsline{toc}{listasf}{Acrónimos}
%\section*{Acrónimos}
%\begin{table}[htbp]
%\begin{tabular}{p{2cm}p{10cm}}
%Escuela Técnica Superior de In
%LTI & Lineal Invariante con el Tiempo \\
%LTV& Lineal Variable con el Tiempo\\
%AWGN& Ruido blanco gaussiano aditivo\\
%DMS& Fuente discreta sin memoria\\
%AEP& Propiedad de equipartición asintótica\\
%WLLN& Ley Débil de los Grandes Números\\
%DMC& Canal Discreto sin Memoria\\
%BSC& Canal Simétrico Binario\\
%BEC& Canal Binario con Borrado\\
%\end{tabular}
%\end{table}


%\nota{El libro de Lapidoth tiene una excelente recopilación.} %No incluir si no se quiere, comentándolo

%:Empieza el contenido del libro
\mainmatter

%:Página por defecto
\pagestyle{esitscCD}

%%%%%%%%%%%%%%%%%%%%%%%%%%%%%%%%%%%%%%%%%%%%%%%%%%%%%%%%%%%%%%%%%%%%%%%%%%%%%%%
%%%%%%% Incluir los diferentes capítulos del TFG en carpetas separadas.
%:Los diferentes capítulos, en carpetas separadas
%
\chapter{Introduction}
\label{ch:Introduction}
%%% Presentar el tema: Aerial co-workers: a task planning approach for multi-drone teams supporting inspection operations
% Definir el problema
\lettrine[lraise=-0.1, lines=2, loversize=0.2]{T}{he} use of \glspl{UAV} has grown considerably in recent years for numerous applications including real-time monitoring, search and rescue, providing wireless coverage, security and surveillance, precision agriculture, package delivery and infrastructure inspection \cite{CivilAplications}. With the rapidly developing technology in this area, and demonstrations of what \glspl{UAV} can do, there are increasing efforts to bring this technology to other applications. With the expected increase in applications for this technology, new problems and challenges arise, including autonomy, safety, obstacle avoidance and coordination of multi-\gls{UAV} teams. Developing the technology to solve these problems will be a major effort, but as \glspl{UAV} have proven to be critical in situations where humans are at high risk or highly inefficient and their capacity to evolve and develop even more potential in the short term, companies are investing in developing all sort of \gls{UAV}-based solutions.

\section{Motivation}
\label{sec:Motivation}
%%% Capi: motivación del problema:
% Con el incremento que ha sufrido la demanda eléctrica mundial, ha aparecido un reto para las compañías encargadas del suministro eléctrico relacionado con el mantenimiento y la reparación de las redes eléctricas de forma que se puedan minimizar la frecuencia de las averías. Según [PowerOutagesCauses], una de las principales causas de cortes eléctricos es el daño de las líneas de transmisión debido al mal tiempo o a campañas de inspección ineficientes.

% Por qué interesan los equipos multi-UAV para la inspección, principales barreras, etc. Puedes hablar del proyecto AERIAL-CORE como contexto del trabajo. Coje texto del paper que te pasé y del proyecto de tesis.

%%% Contexto y justificación del trabajo: 
% Dar razones de por qué es útil diseñar un planificador de tareas para equipos multi-UAV

\section{Objectives}
\label{sec:Objectives}
% Situación actual del ámbito investigado
% Antecedentes teóricos y teorías existentes (resultados de la revisión bibliográfica)
% Conceptos y definiciones clave

%%%: objetivos que se quieren alcanzar en tu TFM en concreto, dentro de todo el problema.
% Hablar en general del proyecto y de lo que quiero hacer.

%%% Preguntas de investigación e hipótesis: 
% Enumerar las hipótesis realizadas para diseñar el planificador

%\begin{hypothesis}\label{hyp:inicial}
%    "Dos \gls{ETSI} próximos entre sí provocarán patrones de error similares a la salida".
%\end{hypothesis}

\endinput

% Los UAV inteligentes son la siguiente gran revolución en esta tecnología, permitiendo su uso en aplicaciones como la inspección de infraestructuras de forma conjunta con humanos.
%
\chapter{State of the art}
\label{ch:StateOfTheArt}

% Poner en contexto las tecnologías que hay hoy día y demás
% Related work: buscar artículos que tengan que ver con mi proyecto para poner en contexto lo que voy a aportar.

\section{Detección de fallos (\textit{Fault Detection})}
\label{sec:FaultDetection}
Dado que no es posible realizar
un diagnóstico de \gls{SEU} sin detectarlo primero, numerosos estudios se centran
en desarrollar técnicas que permitan detectarlos a tiempo para suprimir sus 
efectos. Por ejemplo, en 2014, un equipo chino presentó una técnica de detección 
de \gls{SEU} basada en la \textit{Máquina de Boltzman Restringida o \gls{RBM}}, 
bloque fundamental en muchos algoritmos de \textit{Deep Learning} 
\cite{RBMSEUdetection}. En \cite{SCARA} abordan el problema de \textit{faul 
detection} por el modelo dinámico del sistema. Comparan las lecturas tomadas por
los sensores con los valores teóricos que se obtienen del modelo dinámico del
robot SCARA. De esta forma detectan anomalías debidas a radiación. En un estudio
más reciente, enfocado a sistemas embebidos, emplean programas de detección por
software. Multitud de hilos se ejecutan simultáneamente y se encargan de examinar
el circuito con el objetivo de detectar alguna irregularidad causada por
radiación \cite{DetectingSEUs}.


% En esta seccion contaría que lo que existe principalmente es para fallos de
% fabricación.
% FAULT LOCATION
\section{Diagnóstico de fallos o localización de fallos}
\label{sec:FaultLocation}
Hasta ahora, el diagnóstico de fallos ha sido poco estudiado, siendo los fallos de
fabricación a los que más esfuerzos de investigación se les ha dedicado
\cite{VLSI, EfficientSA0SA1, RepairSA0SA1, LargeComb, ANewRep, FILC, FDIRC}.
Estos no son el tipo de fallos que nos interesa diagnosticar en esta
investigación, ya que no son causados por radiación, si no que se producen, como
su nombre indica, en el momento de fabricación del circuito (\textit{stuck-at-0,
stuck-at-1}).

Las técnicas existentes para localización de fallos provocados por radiación se
basan principalmente en el uso de diccionarios de fallos, aunque también se
emplean vectores de prueba, listas de fallos, tabla de verdad de nodos
(\textit{"node truth table"}) y tabla de conexiones de pines (\textit{pin
connection table}) \cite{DiagnosisTechniques, LASAR, RTFDandD}. 

A excepción de contados estudios, la mayoría de los revisados modelan al 
circuito bajo prueba o \textit{\gls{CUT}} como una caja negra, es decir, el diseño
del circuito no se conoce y solo las salidas pueden ser monitorizadas.
Normalmente, el número de biestables del circuito es mucho mayor que el número de
salidas, por lo que es necesario observar el circuito el suficiente tiempo como
para detectar patrones que puedan ser asociados a la localización de un
determinado \gls{SEU} \cite{SEUDiagnosis}. Estas huellas son registradas y
almacenadas en un diccionario durante una fase previa al diagnóstico.

El diccionario de fallos se genera mediante inyección de fallos, en alguna
plataforma que lo permita \cite{FastFI, LeonFI, FTU}, y contiene información de 
la localización de los \gls{SEU} inyectados y el patrón de salidas que produce. 
Si el diccionario recoge todas las posibilidades, se habla de diccionario 
completo o exhaustivo, tomando el nombre de la campaña de inyección de fallos 
necesaria para generarlo (\textit{Campaña Exhaustiva}). En el caso contrario, es 
un diccionario incompleto o no exhaustivo, es decir, no todas las posibles 
combinaciones de (biestable, ciclo) han sido inyectadas y almacenadas en el 
diccionario. 

% La localización del \gls{SEU}, una vez detectado, se consigue comparando el 
% patrón de salida generado con la información contenida en los diccionarios.
Durante la fase de diagnóstico, para localizar un \gls{SEU} detectado, se compara
el patrón de error que genera en las salidas del circuito con los patrones
almacenados en el diccionario. Debido al largo tiempo de observación comentado, la
información a comparar puede tener un tamaño considerable, y por tanto el tiempo
necesario para procesar la comparación es alto. Una solución para reducir esta
cantidad de información y por tanto, el tiempo, permitiendo incluso localización
de \gls{SEU} en tiempo real, es comprimirla. Un ejemplo sería el uso de códigos 
HASH \cite{SEUDiagnosis}.

% This premise is pretty ambitious since supposes thatthere  is  an  unique
% signature  for  a  couple  of  clock  cycle  andFF  and  viceversa,  i.e.  this
% relation  is  univocal.  In  general,  itis not true, and several different
% injected faults can give placeto exactly the same signature at the outputs. In
% this scenario,it  is  also  obvious  that  the  greater  the  time  we  analyze
% theoutputs,  the  greater  the  possibility  to  get  different  signaturesfor
% different injected faults.

% Problema de que no se llegue a un único candidato, sino a una lista
% Problema de las colisiones
% There is another problem affecting injectivity related to theCUT  itself.  In  a
% fault  injection  campaign,  when  several  runsare  performed,  it  is
% possible  that  the  CUT  shows  exactly  thesame outputs for different fault
% injections, and not only for theGOLDEN outputs but also for wrong outputs. In
% such cases,the fault dictionary is no longer univocal or unambiguous.
Dada la gran cantidad de biestables existentes en comparación con el reducido
número de salidas, no es difícil imaginar la posibilidad de que dos \gls{SEU}
localizados en biestables y/o ciclos distintos produzcan exactamente el mismo
patrón de error a la salida, al menos durante el tiempo y test programados. 
Cuando esto ocurre, se habla de \textit{"Colisión"}. Además, es posible que un
\gls{SEU} no produzca error alguno a la salida durante el test, siendo
indistinguible de una situación libre de conmutaciones. Ante estas situaciones,
existirá más de una entada del diccionario que coincida con la buscada. Como
resultado del diagnóstico se obtienen no una si no una lista de posibles
localizaciones para el \gls{SEU} bajo diagnóstico.

% Explicar que el problema de la técnica basada en códigos hash es que necesita 
% de diccionarios exhaustivos, y generarlos es inviable para circuitos grandes.
Hasta ahora hemos hablado de diagnóstico empleando diccionarios de fallos
completos, pero si el \gls{CUT} es grande, obtener un diccionario exhaustivo es
una operación inviable, ya que la cantidad de combinaciones biestable-ciclo a
inyectar para ello se vuelve inabarcable. Si se intenta diagnosticar un \gls{SEU}
empleando un diccionario de fallos incompleto, aparecen nuevos problemas, ya que
puede ocurrir que la ubicación correcta no se haya inyectado durante la prueba, y
por tanto no se encuentre en el diccionario. Si además existe una colisión que sí
se ha inyectado, el diagnóstico concluirá con una localización única
aparentemente correcta que puede no se acerque nada a la real.


\endinput

%
% Estudio teórico, programas usados, software empleado, entorno de programación, metodología de trabajo, hablar de los
% BT y de las maquinas de estados, experimentos, simulaciones, código, conclusiones, líneas futuras.
%\include{/}
%
\chapter{Conclusions and future work}
\label{ch:ConclusionsAndFutureWork}
% Las conclusiones en formato:
    % Se ha hecho X, Y, y funciona muy bien.
    % Se ha visto que ocurre A, B, C

\section{Conclusions}
\label{sec:Conclusions}
% Optimalidad del planner, funcion de costes, aproximación realizada, simplificaciones realizadas: decir que la aproximación del Planner en casos reales es aceptable
% El BT se puede mejorar pero funciona muy bien y sienta las bases para programar comportamientos más complejos en el futuro. Servirá de ejemplo para la comunidad. Permite generar comportamientos complejos y numerosos estados sin que haya que preocuparse de las transiciones entre estados como pasa con las FSM, en las que este crece exponencialmente con el número de estados.
% Planificación de recargas: Se podría haber añadido una tarea de tipo Recharge para planificarlas. No planifica teniendo en cuenta las recargas ahora mismo, solo si hay batería suficiente. Eso se puede mejorar.
% El "fallo" que he encontrado en el Tool Delivery Task Tree. Devolve la herramienta a la base en caso de fallo.
% Separar el Recharge Action Node en la estructura típica (la de Inspection Task Tree): aproximación a la base por un lado y Recarga por otro.
% Importancia del battery faker para testear correctamente situaciones de emergencia.

\section{Future work}
\label{sec:FutureWork}
% Task ¿?
% Planificación de recargas: Se podría haber añadido una tarea de tipo Recharge para planificarlas. No planifica teniendo en cuenta las recargas ahora mismo, solo si hay batería suficiente. Eso se puede mejorar.
% Realidad aumentada
% Introducir algoritmos heuristicos aleatorios en el planificador para encontrar el plan óptimo de verdad.
% ¿Redes neuronales?

\endinput

%
%\chapter{References}
\label{ch:References}



\endinput

%

%%%%%%%%%%%%%%%%%%%%%%%%%%%%%%%%%%%%%%%%%%%%%%%%%%%%%%%%%%%%%%%%%%%%%%%%%%%%%%%
%%%%%%%%%%%%%%%%%%%%%%%%%%%%%%%%%%%%%%%%%%%%%%%%%%%%%%%%%%%%%%%%%%%%%%%%%%%%%%%
%%%%%%% Esto aún no lo he investigado, tengo que ver como va
%%%%%%%%%%%%%%%%%%%%%%%%%%%%%%%%%%%%%%%%%%%%%%%%%%%%%%%%%%%%%%%%%%%%%%%%%%%%%%%
%%%%%%%%%%%%%%%%%%%%%%%%%%%%%%%%%%%%%%%%%%%%%%%%%%%%%%%%%%%%%%%%%%%%%%%%%%%%%%%
%%%%%%% Apéndices
%%:Empezamos con los apéndices, que irían en uno o más ficheros. Es necesario incluir estos ficheros entre el entorno \begin{appendices}....\end{appendices} debido a que se ha deseado utilizar un formato diferente para el título de los apéndices, incluyendo la palabra apéndice, para la numeración de los apéndices, alfabético, y para las cabeceras de las páginas.
%
% \begin{appendices}
%
% % !TEX root =../LibroTipoETSI.tex



%APENDICE A
\chapter{Sobre  \LaTeX}\LABAPEN{ApA}
{Este es un ejemplo de apéndices, el texto es únicamente relleno, para que el lector pueda observar cómo se utiliza}
%%%%%%%%%%%%%%%%%
\section{Ventajas de \LaTeX}

El gusto por el \LaTeX\ depende de la forma de trabajar de cada uno. La principal virtud es la facilidad de formatear cualquier texto y la robustez. Incluir títulos, referencias es inmediato.
%\Blindtext
%\lipsum
Las ecuaciones quedan estupendamente, como puede verse en \EQ{Ap1}
\begin{equation}\LABEQ{Ap1}
x_{1}=x_{2}.
\end{equation}


\section{Inconvenientes}
%\Blindtext
El principal inconveniente de \LaTeX\ radica en la necesidad de aprender un conjunto de comandos para generar los elementos que queremos. Cuando se está acostumbrado a un entorno ``como lo escribo se obtiene'', a veces resulta difícil dar el salto a ``ver'' que es lo que se va a obtener con un determinado comando. 

Por otro lado, en general será muy complicado cambiar el formato para desviarnos de la idea original de sus creadores. No es imposible, pero sí muy difícil. Por ejemplo, con la sentencia siguiente:
 
\begin{lstlisting}[language=,caption={Escritura de una ecuación}, breaklines=true, label=prgA1-01]
\begin{equation}\LABEQ{Ap2}
x_{1}=x_{2}
\end{equation}
\end{lstlisting}
obtenemos:
\begin{equation}\LABEQ{Ap2}
x_{1}=x_{2}
\end{equation}
Esto será siempre así. Aunque, tal vez, esto podría ser una ventaja y no un incoonveniente.

Para una discusión similar sobre el Word\tsp{\textregistered}, ver \APEN{ApB}.
%\Blindtext


%%%%%%%%%%%%%%%%%%%%%%%%%%%%%%%%%%%%%%%
%APENDICE B
\chapter{Sobre Microsoft Word\tsp{\textregistered}}\LABAPEN{ApB}

\section{Ventajas del Word\tsp{\textregistered}}
La ventaja mayor del Word\tsp{\textregistered} es que permite configurar el formato muy fácilmente. Para las ecuaciones,
\begin{equation}
x_{1}=x_{2},
\end{equation}
tradicionalmente ha proporcionado pésima presentación. Sin embargo, el software adicional Mathtype\tsp{\textregistered} solventó este problema, incluyendo una apariencia muy profesional y cuidada. Incluso permitía utilizar un estilo similar al \LaTeX\xspace. Además, aunque el Word\tsp{\textregistered} incluye sus propios atajos para escribir ecuaciones,  Mathtype\tsp{\textregistered} admite también escritura \LaTeX\xspace. En las últimas versiones de Word\tsp{\textregistered}, sin embargo, el formato de ecuaciones está muy cuidado, con un aspecto similar al de \LaTeX.


\section{Inconvenientes de Word\tsp{\textregistered}}
Trabajar con títulos, referencias cruzadas e índices es un engorro, por no decir nada sobre la creación de una tabla de contenidos. Resulta muy frecuente que alguna referencia quede pérdida o huérfana y aparezca un mensaje en negrita indicando que  no se encuentra. 

Los estilos permiten trabajar bien definiendo la apariencia, pero también puede desembocar en un descontrolado incremento de los mismos. Además, es muy probable que Word\tsp{\textregistered} se quede colgado, sobre todo al trabajar con copiar y pegar de otros textos y cuando se utilizan ficheros de gran extensión, como es el caso de un libro.

%\end{equation}
 
%
% \end{appendices}

%%%%%%%%%%%%%%%%%%%%%%%%%%%%%%%%%%%%%%
%%%%%%%%%%%%%%%%%%%%%%%%%%%%%%%%%%%%%%
%:Empieza todo lo que no constituye el cuerpo en si del libro. Todo lo que va detrás
\backmatter

%:Indice de figuras, coméntese las siguientes líneas si no se desea
\cleardoublepage
\phantomsection

%:Para añadir una línea en blanco en el TOC y separar esta lista
\addtocontents{toc}{\protect\mbox{}\protect\hspace*{0pt}\par}
\addcontentsline{toc}{listasb}{\listfigurename}
\pagestyle{especial}
\listoffigures

%:Indice de tablas, coméntese las siguientes líneas si no se desea
\cleardoublepage
\phantomsection
\addcontentsline{toc}{listasb}{\listtablename}
\pagestyle{especial}
\listoftables

%:Indice de Programas
\cleardoublepage
\phantomsection
\addcontentsline{toc}{listasb}{\lstlistlistingname}
\pagestyle{especial}
\lstlistoflistings

%%%%%%%%%%%%%%%%%%%%%%%%%%%%%%%%%%%%%%%%%%%%%%%%%%%%%%%%%%%%%%%%%%%%%%%%%%%%%%%
%:Bibliografía con biblatex
\nocite{*}
\cleardoublepage
\phantomsection
\addcontentsline{toc}{listasb}{\bibname}
\pagestyle{especial}

\bibliographystyle{IEEEtran}
%\bibliographystyle{amsplain} %flexbib amsplain alpha

%:Fichero con la bibliografía, BIBTEX
\bibliography{bibliography}

% Este fichero .bib se puede generar usando algún gestor de bibliografías. Se recomiendan dos:
% - Zotero
% - Mendeley (con licencia de la US)

%:Índice alfabético de palabras
\cleardoublepage
\phantomsection
\addcontentsline{toc}{listasb}{\indexname}
\chaptermark{\indexname}
\printindex


%:Acrónimos
\cleardoublepage
\phantomsection
\addcontentsline{toc}{listasb}{\glossaryname}
\chaptermark{\glossaryname}
\printglossaries


\end{document}
