%%%%%%%%%%%%%%%%%%%%%%%%%%%%%%%%%%%%%%%%%%
%%% NORMALMENTE NO ES NECESARIO HACER 
%%% CAMBIOS EN ESTA PARTE DEL DOCUMENTO
%%%%%%%%%%%%%%%%%%%%%%%%%%%%%%%%%%%%%%%%%%


%:Clase del documento
\documentclass[fontsize=11pt, English=true, Myfinal=true, twoside, numbers=noenddot]{scrbook}
%Minion=true, English=true, Myfinal=true

%:Paquete de estilos propuesto
\usepackage{libroETSI}

%:Paquete específico para cargar tikz (y sus librerías) y pgfplots
\usepackage{dtsc-creafig}

%:Paquete para notaciones específicas
\usepackage{notacion}

%:Paquete para incorporar aspectos concretos de la edición
\usepackage{edicionPFC}

% Paquete para incluir epígrafes en los capítulos
\usepackage{epigraph}

% Paquete para incluir glosario
\usepackage{glossaries}

\newcommand{\JC}[1]{{\color{red} {JC: #1}}}
\newcommand{\AC}[1]{{\color{red} {AC: #1}}}

% === Tikz Packages ==================
\usepackage{tikz}
\pgfdeclarelayer{foreground}
\pgfsetlayers{background, main, foreground}
\usetikzlibrary{quotes, angles, backgrounds, arrows, automata, shapes, positioning, calc, through, spy, decorations.pathreplacing, decorations.markings, arrows.meta, automata, petri}

\tikzset{
    imglabel/.style={
      rectangle,
      inner sep=2pt,
      % rounded corners=.1em,
      text=black,
      minimum height=1em,
      text centered,
      fill=white,
      fill opacity=1.0,
      text opacity=1,
      anchor=south west,
    },
  }

\tikzset{
	state/.style={
		rectangle,
		draw=black, very thick,
		minimum height=1.0em,
		text centered,
	},
}

\tikzset{
  % style to apply some styles to each segment of a path
  on each segment/.style={
    decorate,
    decoration={
      show path construction,
      moveto code={},
      lineto code={
        \path [#1]
        (\tikzinputsegmentfirst) -- (\tikzinputsegmentlast);
      },
      curveto code={
        \path [#1] (\tikzinputsegmentfirst)
        .. controls
        (\tikzinputsegmentsupporta) and (\tikzinputsegmentsupportb)
        ..
        (\tikzinputsegmentlast);
      },
      closepath code={
        \path [#1]
        (\tikzinputsegmentfirst) -- (\tikzinputsegmentlast);
      },
    },
  },
  % style to add an arrow in the middle of a path
  mid arrow/.style={postaction={decorate,decoration={
        markings,
        mark=at position .5 with {\arrow[#1]{stealth}}
      }}},
}
% ===== End of tikz packages ============

%:Para modificar fácilmente la fuente del texto.
\makeatletter
\ifdtsc@Minion % Queremos utilizar la fuente Minion y lo hemos declarado al principio
	\ifluatex
		\setmainfont[Renderer=Basic, Ligatures=TeX,	% Fuente del texto 
		Scale=1.01,
		]{Minion Pro}
   		% En este caso conviene modificar ligeramente el tamaño de las fuentes matemáticas
		\DeclareMathSizes{10}{10.5}{7.35}{5.25}
		\DeclareMathSizes{10.95}{11.55}{8.08}{5.77}
		\DeclareMathSizes{12}{12.6}{8.82}{6.3}
%		\setmainfont[Renderer=Basic, Ligatures=TeX,	% Fuente del texto 
%		]{Adobe Garamond Pro}
%		\setmainfont[Renderer=Basic, Ligatures=TeX,	% Fuente del texto 
%		]{Palatino LT Std}
	\fi
\else
	\ifluatex
		% Para utilizar la fuente Times New Roman, o alguna otra que se tenga instalada
		\setmainfont[Renderer=Basic, Ligatures=TeX,	% Fuente del texto 
		Scale=1.0,
		]{Times New Roman}
	\else
		\usepackage{tgtermes} 	%clone of Times
		%\usepackage[default]{droidserif}
		%\usepackage{anttor} 	
	\fi
\fi
\makeatother

% Formato A4
\geometry
{paperheight=297mm,%
paperwidth=210mm,%
top=25mm,%
headsep=8.5mm,%
includefoot, 
textheight=240mm, 
textwidth=150mm, 
bindingoffset=0mm, 
twoside}

\usepackage[a4,center]{crop}%para poner las cruces de esquina de página, poner la opción cross

%:Esquema de numeración por defecto
\setenumerate[1]{label=\normalfont\bfseries{\arabic*.}, leftmargin=*, labelindent=\parindent}
\setenumerate[2]{label=\normalfont\bfseries{\alph*}), leftmargin=*}
\setenumerate[3]{label=\normalfont\bfseries{\roman*.}, leftmargin=*}
\setlist{itemsep=.1em}
\setlength{\parindent}{1.0 em}

\setcounter{tocdepth}{4}						% El nivel hasta el que se muestra el índice 


%%%%%%%%%%%%%%%%%%%%%%%%%%%%%%%%%%%%%%%%%%
%%% A PARTIR DE AQUÍ HAY QUE EDITAR
%%%%%%%%%%%%%%%%%%%%%%%%%%%%%%%%%%%%%%%%%%

% Ejemplo de Glosario
\newacronym[type=main]{ETSI}{ETSI}{Escuela Técnica Superior de Ingeniería}
\newacronym[type=main]{US}{US}{Universidad de Sevilla}
\newacronym[type=main, plural=UAVs, firstplural=Unmanned Aerial Vehicles (UAVs)]{UAV}{UAV}{Unmanned Aerial Vehicle}
\newacronym[type=main]{ROS}{ROS}{Robot Operating System}
\newacronym[type=main]{SITL}{SITL}{Software In The Loop}
\newacronym[type=main, plural=RPAs, firstplural=Remotely Piloted Aircraft]{RPA}{RPA}{Remotely Piloted Aircraft}
\newacronym[type=main]{NASA}{NASA}{National Aeronautics and Space Administration}
\newacronym[type=main]{PDDL}{PDDL}{Planning Domain Description Language}
\newacronym[type=main]{ASP}{ASP}{Answer Set Programming}
\newacronym[type=main]{RDDL}{RDDL}{Relational Dynamic Influence Diagram Language}
\newacronym[type=main, plural=FSM, firstplural=Finite State Machines (FSM)]{FSM}{FSM}{Finite State Machine}
\newacronym[type=main]{PHFSM}{PHFSM}{Parallel Hierarchical Finite State Machine}
\newacronym[type=main, plural=BTs, firstplural=Behaviour Trees (BTs)]{BT}{BT}{Behaviour Tree}
\newacronym[type=main]{UAL}{UAL}{UAV Abstraction Layer}
\newacronym[type=main, plural=ACWs, firstplural=Aerial Co-Worker]{ACW}{ACW}{Aerial Co-Worker}
\newacronym[type=main, plural=WPs, firstplural=waypoints (WPs)]{WP}{WP}{waypoint}
\newacronym[type=main, plural=IDs, firstplural=identifications (IDs)]{ID}{ID}{identification}
\newacronym[type=main]{RTPS}{RTPS}{Real-Time Positioning System}
%\newacronym[type=main, plural=, firstplural=]{}{}{}
%\newacronym[type=main]{}{}{}


\makeindex
\makeglossaries %Si no se quiere el glosario, comentar esta línea.


%:Empieza el documento

\begin{document}


%PORTADA
%ver edicionPFC.sty para modificaciones

%:Para crear la portada y la portada interior (pagina titular)
\titulo{Aerial co-workers: a task planning approach for multi-drone teams supporting inspection operations} %\mbox evita que se divida una palabra al cambiar de línea
\autor{Álvaro Calvo Matos}
\director{Jesús Capitán Fernandez}
\titulodirector{Associate Professor}

\departamento{Dpto. Ingeniería de Sistemas y Automática}
%\departamento{Systems and Automation Engineering Department}
\centro{Escuela Técnica Superior de Ingeniería}
\universidad{Universidad de Sevilla}
%\universidad{University of Seville}
\titulacion{Máster en Ingeniería Electrónica, Robótica y \mbox{Automática}}
%\titulacion{Master in Electronic, Robotic and Automation Engineering}
\fecha{2021}
\nombretrabajo{Trabajo Fin de Máster} 


\hypersetup
	{
 	linkcolor=black, %Tocar para poner color en enlaces
	pdfauthor={\elautor},
	pdftitle={\nombretrabajo,\eltitulo}, 
	pdfkeywords={Latex, edición, formato de texto}	
	 }

%logo de la Universidad y logo del departamento, si lo hubiera. Para cambiar el pie de página con los logos, debe editarse el fichero ediciónPFC.sty
\portadaPFC{figuras/LogoUS.pdf}{figuras/LogoTSC.pdf} 
% Para incluir el logo del departamento hay que modificar el segundo parámetro de la linea anterior de este .tex, y
% hay que modificar las lineas 92 a 100 del fichero "edicionPFC.sty"

%Fin Portada

%:Todo lo que constituye la primera parte del libro que no es el cuerpo del libro en realidad
\frontmatter
\pagenumbering{Roman} %Pone la numeración en mayúscula (En español parece que es obligatorio)

%\include{dedicatoria/dedicatoria}%¿Comentar para proyectos/tesis?
\chapter*{Agradecimientos}
%\pagestyle{especial}
\pagestyle{empty}
%\chaptermark{Agradecimientos}
\phantomsection
%\addcontentsline{toc}{listasf}{Agradecimientos}
%\vspace{1cm}
%{\huge{Agradecimientos}}
%\vspace{1cm}

\lettrine[lraise=-0.1, lines=2, loversize=0.25]{}{}
Lorem itsum
% Tutor del proyecto:

% Compañeros del departamento

% Maestros de la carrera

% Compañeros de clase, por acompañarde durante todo el camino, en especial a 
% Damian por su apoyo y amistad en todo momento durante este último año.

% La familia

{\flushleft{\hfill \emph{Álvaro Calvo Matos}}}%
\vspace{-.3cm}
{\flushleft{\hfill \emph{Máster en Ingeniería Electrónica, Robótica y Automática}}}
{\flushleft{\hfill \emph{Sevilla, 2021}}}%


%PFC/PFM/TESIS
\chapter*{Abstract}
\pagestyle{especial}
\chaptermark{Abstract}
\phantomsection
\addcontentsline{toc}{listasf}{Abstract}

%%% Tal about the problem that TFM addresses.
\lettrine[lraise=-0.1, lines=2, loversize=0.2]{T}{his} master's thesis has addressed problems steaming from the recent increase in the applications of cooperative \gls{UAV} teams, which are their autonomy to operate over a long period of time with robustness to possible failures, and the ability to enhance the team with cognitive capabilities so that they are able to operate in dynamic environments with humans.

%%% Talk about the importance or interest in solving the problem.
Many of these applications are currently being executed by humans, making the activities much more expensive, time-consuming and, in some cases, even dangerous. This is why there is currently a great deal of interest and effort being put into developing solutions to the problems posed. 

%%% Objectives pursued: Why was this research carried on? What is the goal? ¿Objectives? Starting hypothesis?
The aim of the work in this thesis was to develop cognitive planning techniques for coordinating fleets of quadrotors to assist human operators in inspection and maintenance tasks on high-voltage power lines. These techniques should also extend the autonomy of the system, ensure that safety requirements between UAVs and human workers are met, and ensure the success of the mission.

%%% Description of the proposed solution. How was it done? Used techniques?
A software architecture has been proposed based on a central planner and a distributed behaviour manager. To carry out mission planning, cost functions for each incoming task have been defined. Thus, tasks are assigned to UAVs efficiently taking into account their battery constraints. Moreover, to control the behaviour of the UAVs and ensure the safety of the aerial equipment, behaviour trees have been implemented.

%%% Results: Most important data that respond to the objectives and hypothesis set.
As a result, it has been possible to develop a software architecture capable of dynamically planning missions while ensuring the safety of the equipment involved. This provides a good base that can be easily adapted and from which more complex planners could be developed in the future. Compared to the typical way of implementing behaviour managers, involving complex finite state machines that are difficult to read, reuse and extend, the use of behaviour trees is a great improvement and will allow the creation of increasingly complex behaviours.

\chapter*{Resumen}
\pagestyle{especial}
\chaptermark{Resumen}
\phantomsection
\addcontentsline{toc}{listasf}{Resumen}
%%% Hablar del problema que aborda el TFM.
\lettrine[lraise=-0.1, lines=2, loversize=0.2]{E}{ste} Trabajo de Fin de Máster ha afrontado problemas que surgen del reciente aumento de las aplicaciones de equipos cooperativos de \gls{UAV}, los cuales son la autonomía para operar de forma prolongada en el tiempo con robustez ante posibles fallos, y la dificultad de aportar al equipo capacidades cognitivas para poder operar en entornos dinámicos con humanos. 

%%% Hablar de la importancia o del interés que hay por solucionar el problema.
Muchas de estas aplicaciones están siendo ejecutadas actualmente por humanos, haciendo las actividaded mucho más costosas, lentas, e incluso en algunos casos, peligrosas. Es por eso que actualmente existe un gran interés y se están destinando muchos esfuerzos para desarrollar soluciones para los problemas planteados.

%%% Objetivos que se persiguen: ¿Por qué realizo esta investigación? ¿Qué se busca lograr? ¿Objetivo? ¿Hipótesis de partida?
El objetivo del trabajo en este TFM es desarrollar técnicas cognitvas de planificación para coordinar flotas de UAVs que asistan a operarios humanos en tareas de inspección y mantenimiento en líneas eléctricas de alta tensión. Estas técnicas deben además extender la autonomía del sistema, garantizar que se cumplan los requisitos de seguridad entre UAVs y trabajadores humanos, y asegurar el éxito de la misión.

%%% Descripción de la solución propuesta. ¿Cómo lo he hecho? ¿Técnicas utilizadas?
Se ha propuesto una arquitectura de software basada en un planificador central y un gestor de comportamientos distribuido. Para llevar a cabo la planificación se han definido costes para las distintas tareas existentes. De esta forma, se asignan a los distintos UAVs de manera eficiente, teniendo en cuenta sus restricciones de batería. Por el otro lado, para controlar el comportamiento de los UAVs y asegurar la seguridad de los equipos aéreos, se han implementado diferentes árboles de comportamiento.

%%% Resultados: Datos más importantes que respondan a las hipótesis y los objetivos marcados.
Como resultado, se ha conseguido desarrollar una arquitectura de software capaz de realizar la planificación de las misiones de forma dinámica asegurando mientras tanto la seguridad de los equipos involucrados. Esto constituye una buena base que se puede adaptar fácilmente y a partir de la cual se pueden desarrollar futuros planificadores más complejos. Comparado con la forma típica de implementar gestores de comportamiento, ivolucrando complejas máquinas de estados finitas difíciles de leer, reutilizar y ampliar, el uso de árboles de comportamiento supone una gran mejora y permitirá la creación de comportamientos cada vez más complejos.
 

% Índice abreviado 
% El índice abreviado se incluye también en algunos libros, con menor detalle que el completo. Descomentar las siguientes líneas.
\cleardoublepage
\phantomsection
\addcontentsline{toc}{listasf}{Short Outline}
\pagestyle{especial}
\shorttoc{Short Outline}{1}

%Índice normal, el completo
\cleardoublepage
\phantomsection
\pagestyle{especial}
\tableofcontents

%%%%%%%%%%%%%%%%%%%%%%%%%%%%%%%%%%%%%%%%%%%%%%%%%%%%%%%%%%%%%%%%%%%%%%%%%%%%%%%
%%%%%%% Descomentar la siguiente linea y editar notacion.tex si hiciera falta
%%%%%%% incluir notación en el TFG.
%\include{notacion/notacion} %No incluir si no se quiere, comentándolo

%:Empieza el contenido del libro
\mainmatter

%:Página por defecto
\pagestyle{esitscCD}

%%%%%%%%%%%%%%%%%%%%%%%%%%%%%%%%%%%%%%%%%%%%%%%%%%%%%%%%%%%%%%%%%%%%%%%%%%%%%%%
%%%%%%% Incluir los diferentes capítulos del TFG en carpetas separadas.
%:Los diferentes capítulos, en carpetas separadas
%
\chapter{Introduction}
\label{ch:Introduction}
\lettrine[lraise=-0.1, lines=2, loversize=0.2]{L}{o}rem itsum

% Hablar en general del proyecto y de lo que quiero hacer.

% Estudio teórico
% Programas usados, software empleado, entorno de programación
% Metodología de trabajo

% Dar razones de por qué es útil diseñar un planificador de tareas para equipos multi-UAV

% Enumerar las hipótesis realizadas para diseñar el planificador

\section{Motivation}
\label{sec:Motivation}
% Capi: motivación del problema: por qué interesan los equipos multi-UAV para la inspección, principales barreras, etc. Puedes hablar del proyecto AERIAL-CORE como contexto del trabajo. Coje texto del paper que te pasé y del proyecto de tesis.  

\section{Objectives}
\label{sec:Objectives}
% Capi: bjetivos que se quieren alcanzar en tu TFM en concreto, dentro de todo el problema.

%\begin{hypothesis}\label{hyp:inicial}
%    "Dos \gls{ETSI} próximos entre sí provocarán patrones de error similares a la salida".
%\end{hypothesis}

\endinput

%
\chapter{Preliminaries}
\label{ch:Preliminaries}
\lettrine[lraise=-0.1, lines=2, loversize=0.2]{L}{o}rem itsum

% Poner en contexto las tecnologías que hay hoy día y demás.
\section{Current technology}
\label{sec:CurrentTechnology}

\subsection{UAVs}
\label{subsec:UAVs}

\subsection{Aerial co-workers}
\label{subsec:AerialCo-workers}

\subsection{Multi-drone teams}
\label{subsec:Multi-droneTeams}


% Related work: buscar artículos que tengan que ver con mi proyecto para poner en contexto lo que voy a aportar.
\section{Related work}
\label{sec:RelatedWork}

\subsection{Inspection applications with UAVs}
\label{subsec:InspectionApplicationsWithUAVs}

% Hablar de las formas existentes que hay para abordar el problema del reparto de tareas en equivos multi-UAV. Poner en contexto lo que voy a aportar con mi TFM.
\subsection{Task planning in multi-drone teams}
\label{subsec:TaskPlanning}

% Hablar de las formas existentes que hay para gestionar el comportamiento de un dron y su guiado en cada instante. Poner en contexto lo que voy a aportar con mi TFM. Hablar de las FSM y de los BT
\subsection{Drone behavior management}
\label{subsec:DroneBehaviorManagement}


% Estudio previo
\section{Previous study}
\label{sec:PreviousStudy}

\subsection{ROS}
\label{subsec:ROS}

\subsection{Gazebo}
\label{subsec:Gazebo}

\subsection{Behaviour Trees}
\label{subsec:BehaviourTrees}

\subsection{Groot}
\label{subsec:Groot}

\subsection{Rviz}
\label{subsec:Rviz}

\endinput

% 
%\chapter{Teoric Approach}
\label{ch:TeoricApproach}

% Estudio teórico
% Programas usados, software empleado, entorno de programación
% Metodología de trabajo
% Hablar de las diferentes formas existentes de abordar los dos problemas a solucionar:
%     Formas de afrontar el reparto de tareas
%     Formas de afrontar el control del comportamiento de los Agentes (BT, FSM)

\lettrine[lraise=-0.1, lines=2, loversize=0.2]{L}{o}rem itsum

%
\chapter{Problem Formulation}
\label{ch:ProblemDFormulation}
\lettrine[lraise=-0.1, lines=2, loversize=0.2]{L}{o}rem itsum

% Descripcion del proyecto para el que se va a diseñar el planificador de tareas.

% Descripcion de la lista de tareas contempladas y explicación de cada una
% ¿? ¿Decir algo sobre los gestos? Se supone que esto es para Piloting y que ahí no hay gestos. De todas formas, mi parte es ajena a los gestos, le llegan las tareas ya procesadas.
\section{Description of tasks}
\label{sec:DescriptionOfTasks}

\subsection{Inspection tasks}
\label{subsec:InspectionTasks}

\subsection{Monitoring tasks}
\label{subsec:MonitoringTasks}

\subsection{Tool delivery tasks}
\label{subsec:ToolDeliveryTasks}


% Otras consideraciones importantes a tener en cuenta: gestíón de la batería, desconexiones, imprevistos, prioridades, tipos de UAV.
\section{Battery recharges}
\label{sec:BatteryRecharges}

\section{Connection losses}
\label{sec:ConnectionLosses}

\section{Task replanning situations}
\label{sec:TaskReplanningSituations}


%
\chapter{Design of the proposed solution}
\label{ch:DesignOfTheProposedSolution}
\lettrine[lraise=-0.1, lines=2, loversize=0.2]{L}{o}rem itsum
% El planificador desarrollado en esta tésis se compone a su vez de dos módulos bien diferenciados, el primero es el planificador de tareas propiamente dicho, encargado de la planificación de la misión y de su replanificación cuando fuera necesario, el segundo se encuentra a bordo de cada UAV y gestiona el comportamiento de cada equipo, ejecutando el plan que le ha asignado el primer módulo y reaccionando ante cualquier imprevisto de forma segura.

%%% Capi: En los capítulos 3 y 4 puedes coger texto del documento que tenemos hecho con Giuseppe, y del proyecto tuyo de tesis. 

% Se utilizará una aproximación jerárquica, con un planificador de alto nivel encargado de activar distintos controladores de bajo nivel. El planificador de alto nivel detectará las tareas requeridas por los operarios, y las distribuirá de manera centralizada entre los UAVs, planificando las recargas necesarias. Además, este planificador reaccionará en tiempo real ante posibles fallos reasignando tareas y ejecutando planes de contingencia. También tendrá capacidades cognitivas para interaccionar con humanos de manera eficiente. Los planificadores de bajo nivel se encargarán de controlar el movimiento de los UAVs para ejecutar las distintas tareas, por ejemplo volar hacia un lugar a inspeccionar o a la posición de un operario esperando una herramienta. La investigación de la tesis estará centrada en la planificación de alto nivel, y se utilizarán algoritmos del estado del arte para los controladores de bajo nivel.

\section{Node diagram}
\label{sec:NodeDiagram}
%%% Explicar las comunicaciones que hay entre el planner y el manager
%%% Explicar que se ha hecho de forma que toda la inteligencia y las decisiones estén y se tomen en el planner

\section{Centralized module: task planner}
\label{sec:Centralized module:TaskPlanner}
%% Protocolo de desconexión
%% Protocolo de pérdida de batería
%% Que ocurre cuando una tarea termina
%% Replanificaciones de tareas: restricciones a la hora de planificar o replanificar

\section{Distributed module: behavior manager}
\label{sec:Distributed module: behavior manager}
%%% Explicar que se ha hecho con árboles de comportamiento en paralelo a algunos procesos de ros

\subsection{Main tree}
\label{sec:MainTree}
%%% Recalcar que se ha hecho de forma que toda la inteligencia y las decisiones estén y se tomen en el planner
%%% Hablar aqui dentro de como se gestionan las desconexiones, la batería y las replanificaciones
%% Protocolo de desconexión
%% Protocolo de pérdida de batería
%% Que ocurre cuando una tarea termina


% Information exchanges 
% Information interfaces/channels
% Takeovers
\subsection{Inspection task tree}
\label{sec:InspectionTaskTree}

\subsection{Monitoring task tree}
\label{sec:MonitoringTaskTree}

\subsection{Tool delivery task tree}
\label{sec:ToolDeliveryTaskTree}

\section{Lower and upper level modules faker}
\label{sec:LowerAndUpperLevelModulesFaker}
%%% GoToWP, Recharge, Monitoring, Inspection, ToolDelivery
%%% Battery sensor

%
\chapter{Results}
\label{ch:Results}

\lettrine[lraise=-0.1, lines=2, loversize=0.2]{L}{o}rem itsum

\section{Task planning}
\label{sec:TaskPlanning}

\subsection{Battery}
\label{subsec:Battery}

\subsection{Connection lost}
\label{subsec:ConnectionLost}

\subsection{Replanning}
\label{subsec:Replanning}


\section{Drone behaviour manager results}
\label{sec:DroneBehavioutManagerResults}

\subsection{Battery management}
\label{subsec:BatteryManagement}

\subsection{Connection lost management}
\label{subsec:ConnectionLostManagement}

\subsection{Replanning management}
\label{subsec:ReplanningManagement}


\section{Simulations}
\label{sec:Simulations}

\subsection{One drone simulations}
\label{subsec:OneDroneSimulations}

\subsection{Multi-drone simulations}
\label{subsec:Multi-droneSimulations}


%
% \chapter{Conclusions and future work}
\label{ch:ConclusionsAndFutureWork}

\section{Conclusions}
\label{sec:Conclusions}
%% Comentar los objetivos marcados:
In this work, a task planning approach has been developed with the capability to perform mission planning for multi-\gls{UAV} teams. The system has sufficient cognitive capability to control multiple \glspl{UAV} operating as co-workers in dynamic environments safely. Simulations have demonstrated the system's ability to detect emergency situations and act in a safe way by executing contingency plans autonomously while calculating a new plan to follow that takes into account the unforeseen events that have occurred. The design of the system proposed two blocks: a centralised block on the ground in charge of optimal mission planning; and distributed blocks on board each of the \glspl{ACW} to allow the system to be robust to failures and have enough cognitive capacity to react to unforeseen events by recalculating the optimal plan. In this way, an efficient execution of tasks and a better use of resources is achieved, which translates into greater combined autonomy for the \gls{UAV} team.

The system has been designed in \gls{ROS} and the communications between the different software layers and the different blocks of each layer have been carried out using the tools offered by \gls{ROS}. This facilitates the integration of the system developed in other robotics projects that require a task planning system with these or similar characteristics. The use of \glspl{BT} for the design of the \gls{UAV} behaviour manager has great advantages over conventional \glspl{FSM}. This technique makes it possible to generate complex behaviours with numerous states without having to worry about taking into account each of the transitions between these states, as happens with \glspl{FSM}, in which the number of transitions grows exponentially with the number of states. The characteristics of the library used to program this part of the system make it easy to maintain, modify or extend. Moreover, thanks to its modular nature, this block can be reused in parts or in its entirety in any other project. The designed \gls{BT}, although it can be improved, has demonstrated in the simulations that works fairly well, laying the foundations for programming more complex behaviours in the future and serving as an example for the aerial robotics community, which can use it as a starting point for other applications. 

With respect to the block in charge of mission planning, the \emph{High-Level Planner}, it has so far demonstrated the ability to generate coherent plans in the conditions in which it has been tested and has also shown itself capable of recalculating these plans online in reaction to unforeseen events of different natures. The achieved solution is able to plan the mission taking into account imposed constraints such as the type of each \gls{ACW}, the priority of each of the tasks and the battery level of each of the \glspl{UAV}, being able to calculate plans for missions consisting of an indefinite number of tasks and \glspl{ACW}. Regarding the optimality of the plans generated by this block, it should be noticed that it has not been implemented any solution to approximate the optimal plan, but a heuristic solution based on a cost function that is calculated for each of the \glspl{ACW} with each of the tasks. However, this type of solution should be enough for initial tests in the targeted scenarios, are composed of few tasks and \glspl{ACW}. The solution reached, in this context, is a valid approximation towards a planning algorithm that generates a close-to-optimal plan.

\section{Future work}
\label{sec:FutureWork}
As part of the future work, the techniques developed in this work will be validated in a real environment with real \glspl{UAV}. In addition, the system will be used as a starting point for a PhD thesis in which it will be attempted to refine and improve the design of the \emph{Agent Behaviour Manager} block, as well as to develop a planning algorithm that generates a real approximation to the optimal plan for each situation. To this end, probabilistic decision-making algorithms will be introduced into the system, as well as the capacity to learn in real time some characteristics such as the \glspl{UAV}' battery consumption or human's intentions, thus anticipating unforeseen events and applying contingency plans. This would provide a greater robustness against failures and highly dynamic environments.

A first improvement for the planner with respect to the current version could be the incorporation of \emph{reload} tasks that, instead of being requested by human operators like the rest of the tasks, would be incorporated by the \emph{High-Level Planner} into the task queue, thus separating emergency reloads (or reloads that are executed when an agent is idle) from reloads carried out as part of the plan. Implementing this change in the \gls{BT} would mean modifying the \emph{Perform Task} tree to contemplate this new task in the design of the tree, a change that could be carried out by reusing and slightly adapting the trees used for the \emph{Inspection} and \emph{Safety Monitoring} tasks, taking advantage of the reusability of the \glspl{BT}.

In addition, in future work, it is intended to investigate the use of mixed reality technologies also for inspection applications with multi-\gls{UAV} teams, combining views taken from different points to recreate more complete visual environments for the operator, and improving the human-machine interaction of the system during collaborative tasks.

\endinput

\chapter{Conclusions and future work}
\label{ch:ConclusionsAndFutureWork}
\section{Conclusions}
\label{sec:Conclusions}
En este trabajo se ha desarrolado un planificador de tareas con capacidad para realizar la planificación de misiones para equipos multi-UAV. El sistema tiene la capacidad cognitiva suficiente para controlar a múltiples UAVs que funcionen como co-trabajadores en entornos dinámicos de forma segura. Las simulaciones realizadas han demostrado la capacidad que tiene el sistema para detectar situaciones de emergencia y actuar de forma segura ejecutando planes de contingencia de forma autónoma mientras se calcula un nuevo plan a seguir que tenga en cuenta los imprevistos que hayan acontecido. El diseño del sistema dividido en un bloque centralizado en tierra encargado de realizar la planificación óptima de la misión y de bloques distribuidos a bordo de cada uno de los ACWs permite que el sistema presente robustez ante fallos y capacidad cognitiva suficiente para reaccionar ante eventos imprevistos recalculando el plan óptimo. De esta forma, se consigue una ejecución eficiente de las tareas y un mejor aprovechamiento de los recursos, que se traduce en una mayor autonomía conjunta del equipo de UAVs.

El sistema se ha diseñado en ROS y las comunicacioens entre las diferentes capas de software y los diferentes bloques de cada capa se han realizado empleando las herramientas que este ofrece. Esto facilita la integración del sistema desarrollado en otros proyectos de robótica que necesiten de un planificador de tareas con estas características o similares. El uso de BT para el diseño del UAV behaviour manager presenta grandes ventajas frente a las FSM convencionales. Esta técnica permite generar comportamientos complejos con numerosos estados sin que haya que preocuparse por tener en cuenta cada una de las transiciones entre esos estados como pasa con las FSM, en las que el número de transiciones crece exponencialmente con el número de estados. Las características de la librería empleada para programar esta parte del sistema hacen que sea facil de mantener, de modificar o de amplicar. Además, gracias a su caracter modular, este bloque puede ser reutilizado tanto a partes como en su totalidad en cualquier otro proyecto. El BT diseñado, aunque mejorable, ha demostrado en las simulaciones realizadas que funciona bastante bien, sentando las bases para programar comportamientos más complejos en el futuro y sirviendo de ejemplo para la comunidad de robótica aérea, que podrá emplearlo como punto de partida para otras aplicaciones. 

Respecto al bloque que se encarga de la planificación de la misión, el High-Level Planner, decir que hasta el momento ha demostrado tener la capacidad para general planes con sentido en las condiciones en las que se le ha puesto a prueba y que además ha demostrado ser capaz de recalcular dichos planes en línea como reacción a eventos imprevistos de diferente naturaleza. La solición alcanzada es capaz de planificar la misión teniento en cuenta resticciones impuestas como el tipo de cada ACW, la prioridad de cada una de las tareas y el nivel de autonomía de cada uno de los UAVs, siendo capaz de calcular planes para misiones formadas por una cantidad indefinida de tareas y ACWs. Respecto a la optimalidad de los planes generados por este bloque hay que decir que no se ha implementado ningún algoritmo que realice una búsqueda o aproximación del plan óptimo, sino que se ha diseñado una solución basada en una funcion de costes que se calcula para cada uno de los ACWs con cada una de las tareas. Sin embargo, este trabajo forma parte de un proyecto que tendrá aplicaciones reales en unas condiciones definidas. Por tanto, está justificado alejarse de análisis académicos en los que se busca aproximar el plan óptimo en misiones con un número indeterminado de tareas y de ACWs y analizar en su lugar al planificador en escenarios compuestos por pocas tareas y ACWs. La solución alcanzada, en este contexto, es una aproximación válida hacia un algoritmo de planificación que genere un plan próximo al óptimo.

\section{Future work}
\label{sec:FutureWork}
Como parte del trabajo futuro se realizará una validación en un entorno real con equivos reales de las técnicas desarrolladas en este trabajo. Además, el sistema desarrollado en este trabajo se empleará como punto de partida de una tesis doctoral en la que se tratará de pulir y mejorar el diseño del Agent Behaviour Manager, así como de desarrollar un algoritmo de planificación que genere una aproximación real al plan óptimo de cada situación. Para ello se introducirán al sistema algoritmos heurísticos aleatorios, así como la capacidad para aprender en tiempo real características como el consumo de la batería de los UAVs, anticipándose de esta forma a eventos imprevistos aplicando planes de contingencia, consiguiendo así una mayor robustez ante fallos y extendiendo la autonomía del sistema aún más.

Una primera mejora para el planificador respecto a la versión actual podría ser la incorporación de tareas de tipo Recarga que, en vez de ser solicitadas por operarios humanos como el resto de las tareas, serían incorporadas por el High-Level Planner a la cola de tareas, separando de esta forma las recargas de emergencia o las recargas que se ejecutan cuando un agente se encuentra ocioso, de las recargas llevadas a cabo como parte del plan. Implementar este cambio en el BT supondría modificar el Perform Task Tree para contemplar esta nueva tarea en el diseño del árbol, cambio que se podría llevar a cabo reutilizando y adaptando ligeramente los árboles empleados para las tareas de Inspection y Safety Monitoring aprovechando la reusabilidad de los BTs.

Además, durante la futura tésis, se pretende investigar el uso de tacnologías de realidad mixta también para aplicaciones de inspección con equipos multi-UAV, combinando vistas tomadas desde distintos puntos para recrear entornos visuales más completos para el operario, y mejorando la interacción hombre-máquina del sistema durante tareas colaborativas.

\endinput

%\chapter{References}
\label{ch:References}



\endinput

%

%%%%%%%%%%%%%%%%%%%%%%%%%%%%%%%%%%%%%%%%%%%%%%%%%%%%%%%%%%%%%%%%%%%%%%%%%%%%%%%
%%%%%%%%%%%%%%%%%%%%%%%%%%%%%%%%%%%%%%%%%%%%%%%%%%%%%%%%%%%%%%%%%%%%%%%%%%%%%%%
%%%%%%% Esto aún no lo he investigado, tengo que ver como va
%%%%%%%%%%%%%%%%%%%%%%%%%%%%%%%%%%%%%%%%%%%%%%%%%%%%%%%%%%%%%%%%%%%%%%%%%%%%%%%
%%%%%%%%%%%%%%%%%%%%%%%%%%%%%%%%%%%%%%%%%%%%%%%%%%%%%%%%%%%%%%%%%%%%%%%%%%%%%%%
%%%%%%% Apéndices
%%:Empezamos con los apéndices, que irían en uno o más ficheros. Es necesario incluir estos ficheros entre el entorno \begin{appendices}....\end{appendices} debido a que se ha deseado utilizar un formato diferente para el título de los apéndices, incluyendo la palabra apéndice, para la numeración de los apéndices, alfabético, y para las cabeceras de las páginas.
%
% \begin{appendices}
%
% \include{apendices/apendices} 
%
% \end{appendices}

%%%%%%%%%%%%%%%%%%%%%%%%%%%%%%%%%%%%%%
%%%%%%%%%%%%%%%%%%%%%%%%%%%%%%%%%%%%%%
%:Empieza todo lo que no constituye el cuerpo en si del libro. Todo lo que va detrás
\backmatter

%:Indice de figuras, coméntese las siguientes líneas si no se desea
\cleardoublepage
\phantomsection

%:Para añadir una línea en blanco en el TOC y separar esta lista
\addtocontents{toc}{\protect\mbox{}\protect\hspace*{0pt}\par}
\addcontentsline{toc}{listasb}{\listfigurename}
\pagestyle{especial}
\listoffigures

%:Indice de tablas, coméntese las siguientes líneas si no se desea
\cleardoublepage
\phantomsection
\addcontentsline{toc}{listasb}{\listtablename}
\pagestyle{especial}
\listoftables

%:Indice de Programas
\cleardoublepage
\phantomsection
\addcontentsline{toc}{listasb}{\lstlistlistingname}
\pagestyle{especial}
\lstlistoflistings

%%%%%%%%%%%%%%%%%%%%%%%%%%%%%%%%%%%%%%%%%%%%%%%%%%%%%%%%%%%%%%%%%%%%%%%%%%%%%%%
%:Bibliografía con biblatex
\nocite{*}
\cleardoublepage
\phantomsection
\addcontentsline{toc}{listasb}{\bibname}
\pagestyle{especial}

\bibliographystyle{IEEEtran}
%\bibliographystyle{amsplain} %flexbib amsplain alpha

%:Fichero con la bibliografía, BIBTEX
\bibliography{bibliography}

% Este fichero .bib se puede generar usando algún gestor de bibliografías. Se recomiendan dos:
% - Zotero
% - Mendeley (con licencia de la US)

%:Índice alfabético de palabras
%\cleardoublepage
%\phantomsection
%\addcontentsline{toc}{listasb}{\indexname}
%\chaptermark{\indexname}
%\printindex


%:Acrónimos
\cleardoublepage
\phantomsection
\addcontentsline{toc}{listasb}{\glossaryname}
\chaptermark{\glossaryname}
\printglossaries


\end{document}
